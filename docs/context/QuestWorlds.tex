% Enable hyperlinks
\setupinteraction
  [state=start,
  style=,
  color=,
  contrastcolor=]

% make chapter, section bookmarks visible when opening document
\placebookmarks[chapter, section, subsection, subsubsection, subsubsubsection, subsubsubsubsection][chapter, section]
\setupinteractionscreen[option=bookmark]

\setuppapersize[letter]
\setuppagenumbering[location={footer,middle}]
\setupbackend[export=yes]
\setupstructure[state=start,method=auto]

% use microtypography
\definefontfeature[default][default][script=latn, protrusion=quality, expansion=quality, itlc=yes, textitalics=yes, onum=no, lnum=yes, pnum=yes]
\definefontfeature[smallcaps][script=latn, protrusion=quality, expansion=quality, smcp=yes, onum=yes, pnum=yes]
\setupalign[hz,hanging]
\setupitaliccorrection[global, always]

\setupbodyfontenvironment[default][em=italic] % use italic as em, not slanted

\definefallbackfamily[mainface][rm][CMU Serif][preset=range:greek, force=yes]
\definefontfamily[mainface][rm][ebgaramond]
\definefontfamily[mainface][mm][Latin Modern Math]
\definefontfamily[mainface][ss][Latin Modern Sans]
\definefontfamily[mainface][tt][Latin Modern Typewriter][features=none]
\setupbodyfont[mainface,11pt]

\setupwhitespace[medium]
\setupindenting[no,none]

\setuphead[chapter, section, subsection, subsubsection, subsubsubsection, subsubsubsubsection][number=no]

\definedescription
  [description]
  [headstyle=bold, style=normal, location=hanging, width=broad, margin=1cm, alternative=hanging]

\setupitemize[autointro]    % prevent orphan list intro
\setupitemize[indentnext=no]

\setupfloat[figure][default={here,nonumber}]
\setupfloat[table][default={here,nonumber}]

\setupthinrules[width=15em] % width of horizontal rules

\setupxtable[frame=off]
\setupxtable[head][topframe=on,bottomframe=on]
\setupxtable[body][]
\setupxtable[foot][bottomframe=on]


\input{../../preamble.ctx}

\starttext
\completecontent

\chapter[title={Credits & Legal
Information},reference={credits-legal-information}]

\section[title={Legal Information},reference={legal-information}]

The {\em QuestWorlds} System Reference Document 0.51
(\quotation{QWSRD0.51}) describes the rules of {\em QuestWorlds}. You
may incorporate the rules as they appear in QWSRD0.51, wholly or in
part, into a derivative work, through the use of the {\em QuestWorlds}
Open Game License, Version 1.0. You should read and understand the terms
of that License before creating a derivative work from QWSRD0.51.

Thanks to Wizards of the Coast, the Open Source Initiative, and Creative
Commons for their work in creating the framework behind Open Source (and
in this case Open Game) licenses. You should be aware that the
{\em QuestWorlds} Open Game License for use of the {\em QuestWorlds}
system differs from the Wizards Open Game License and has different
terms and conditions.

\subsection[title={Using This License},reference={using-this-license}]

You should note that this is version of 0.51 of the {\em QuestWorlds}
System Reference Document. We expect to release revised versions of this
SRD, especially after development of Chaosium's upcoming
{\em QuestWorlds Core Book}. When we release the {\em QuestWorlds Core
Book} we will update the version designation to 1.0, indicating that the
SRD reflects the text published in that book. If you are developing
materials for {\em QuestWorlds} projects you may want to bear this in
mind. We will track any changes to the SRD at
{\em https://github.com/ChaosiumInc/QuestWorlds}.

Once we release SRD version 1.0 we expect that to be stable for some
time.

If you have questions about this license, please reach out to Moon
Design at licensing@chaosium.com.

\subsection[title={{\em QuestWorlds} Open Game License, Version
1.0},reference={questworlds-open-game-license-version-1.0}]

All Rights Reserved.

\startitemize[n,packed][stopper=.]
\item
  Definitions:
\stopitemize

\startitemize[a][left=(,stopper=),width=2.0em]
\item
  \quotation{Contributors} means the copyright and/or trademark owners
  who have contributed Open Game Content;
\item
  \quotation{Derivative Material} means copyrighted material including
  derivative works and translations (including into computer languages),
  potation, modification, correction, addition, extension, upgrade,
  improvement, compilation, abridgment, or other forms in which an
  existing work may be recast, transformed, or adapted;
\item
  \quotation{Distribute} means to reproduce, license, rent, lease, sell,
  broadcast, publicly display, transmit, or otherwise distribute;
\item
  \quotation{Open Game Content} means the {\em QuestWorlds} game,
  including the game mechanics and the methods, procedures, processes,
  and routines to the extent such content does not embody Prohibited
  Content and is an enhancement over the prior art and any additional
  content clearly identified as Open Game Content by the Contributor,
  and means any work covered by this License, including translations and
  derivative works under copyright law, but specifically excludes
  Prohibited Content;
\item
  The following items are hereby identified as \quotation{Prohibited
  Content}: All trademarks, registered trademarks, proper names
  (characters, deities, place names, etc.), plots, story elements,
  locations, characters, artwork, or trade dress from any of the
  following: any releases from the product lines of {\em Call of
  Cthulhu}, {\em Dragon Lords of Melniboné}, {\em ElfQuest},
  {\em Elric!}, {\em Hawkmoon}, {\em HeroQuest}, {\em Hero Wars},
  {\em King Arthur Pendragon}, {\em Magic World}, {\em Nephilim},
  {\em Prince Valiant}, {\em Ringworld}, {\em RuneQuest}, {\em 7th Sea},
  {\em Stormbringer}, {\em Superworld}, {\em Thieves' World},
  {\em Worlds of Wonder}, and any related sublines; the world and
  mythology of Glorantha; all works related to the Cthulhu Mythos,
  including those that are otherwise public domain; and all works
  related to {\em Le Morte d'Arthur}. This list may be updated in future
  versions of the License.
\item
  \quotation{Trademark} means the logos, names, marks, signs, mottos,
  and designs that are used by a Contributor to identify itself or its
  products or the associated products contributed to the
  {\em QuestWorlds} Open Game License by the Contributor;
\item
  \quotation{Use,} \quotation{Used,} or \quotation{Using} means to use,
  distribute, copy, edit, format, modify, translate, and otherwise
  create Derivative Material of Open Game Content;
\item
  \quotation{You} or \quotation{Your} means the licensee in terms of
  this agreement.
\stopitemize

\startitemize[n][start=2,stopper=.]
\item
  Grant: Except for material designated as Prohibited Content (see
  Section 1(e) above), the {\em QuestWorlds} System Reference Document
  is Open Game Content, as defined in the {\em QuestWorlds} Open Game
  License version 1.0, Section 1(d). No portion of this work other than
  the material designated as Open Game Content may be reproduced in any
  form without permission from Moon Design.
\item
  The License: This License applies to any work Using {\em QuestWorlds}
  Open Game Content published by Moon Design. You must affix a complete
  copy of this License to any {\em QuestWorlds} Open Game Content that
  You Use and include the Copyright Notice detailed in Section 7 in all
  appropriate locations. No terms may be added to or subtracted from
  this License except as described by the License itself. No other terms
  or conditions may be applied to any {\em QuestWorlds} Open Game
  Content distributed Using this License.
\item
  Offer and Acceptance: By Using the {\em QuestWorlds} Open Game Content
  You indicate Your acceptance of the terms of the {\em QuestWorlds Open
  Game License}.
\item
  Grant of License: Subject to the terms and conditions of this License,
  the Contributors grant You a perpetual, worldwide, royalty-free,
  non-exclusive license to Use the Open Game Content.
\item
  Representation of Authority to Contribute: If You are contributing
  original material as Open Game Content, You represent that Your
  contributions are Your original creation and/or You have sufficient
  rights to grant the rights conveyed by this License.
\item
  Copyright Notice: You must update the Copyright Notice portion of this
  License to include the current version of the text of the Copyright
  Notice of any {\em QuestWorlds} Open Game Content You are copying,
  modifying, or distributing.
\stopitemize

This work created using the {\em QuestWorlds} Open Game License.

{\em QuestWorlds} Open Game License v 1.0 © copyright 2020 Moon Design
Publications LLC.

{\em QuestWorlds} © copyright 2019--2020 Moon Design Publications LLC;
Author, original rules: Robin D. Laws; developed by Greg Stafford, Ian
Cooper, David Dunham, Mark Galeotti, Jeff Richard, Neil Robinson,
Roderick Robinson, David Scott, and Lawrence Whitaker.

QuestWorlds and the QuestWorlds logo are trademarks of Moon Design
Publications LLC. Used with permission.

\startitemize[n][start=8,stopper=.,width=2.0em]
\item
  Limitations on Grant: You agree not to Use any Prohibited Content,
  except as expressly licensed in another, independent Agreement with
  Moon Design. You agree not to indicate compatibility or
  co-adaptability with any Trademark or Registered Trademark in
  conjunction with a work containing {\em QuestWorlds} Open Game Content
  except as expressly licensed in another, independent Agreement with
  the owner of such Trademark or Registered Trademark.
\item
  Identification: If you distribute Open Game Content You must clearly
  indicate which portions of the work that you are distributing are Open
  Game Content.
\item
  Updating the License: Moon Design or its designated Agents may publish
  updated versions of the {\em QuestWorlds} Open Game License, including
  updates to the Prohibited Content list. Material published under any
  version of the License can continue to be published Using the terms of
  that version, but You agree to Use the most recent authorized version
  of this License for any new Open Game Content You publish or for
  revised or updated works with thirty percent (30\letterpercent{}) or
  more revised or new content, by total word count.
\item
  Use of Contributor Credits: You may not market or advertise the Open
  Game Content using the name of any Contributor unless You have written
  permission from the Contributor to do so.
\item
  Reputation: You must not copy, modify, or distribute Open Game Content
  connected to this License in a way that would be prejudicial or
  harmful to the honor or reputation of the Contributors.
\item
  Inability to Comply: If it is impossible for You to comply with any of
  the terms of this License with respect to some or all of the
  {\em QuestWorlds} Open Game Content due to statute, judicial order, or
  governmental regulation then You may not Use any Open Game Material so
  affected.
\item
  Termination: This License will terminate automatically if You fail to
  comply with all terms herein and fail to cure such breach within
  thirty (30) days of becoming aware of the breach.
\item
  Labeling: You must prominently display one of the following
  {\em QuestWorlds} logos on the front and back exterior and in the
  interior package, on the title page or its equivalent, of your Use of
  the Open Content. You are granted permission to reproduce the logo
  only for that purpose.
\stopitemize

{\externalfigure[Logos/QW-Stamp-Red.png][width=1in,height=1in]}
{\externalfigure[Logos/QW-Stamp-Black.png][width=1in,height=1in]}

\startitemize[n][start=16,stopper=.,width=2.0em]
\item
  Severability: If any provision of this License is held to be
  unenforceable, such provision shall be severed only to the extent
  necessary to make it enforceable.
\item
  Governing Law and Venue: The governing law for any disputes arising
  under this License shall be the laws of the State of Michigan, without
  reference to its choice of laws provisions. Venue is exclusively
  vested in the United States Federal District Court for the Eastern
  District of Michigan.
\stopitemize

\section[title={Credits},reference={credits}]

Original Rules: Robin D. Laws

Further Development: Greg Stafford, Ian Cooper, David Dunham, Mark
Galeotti, Jeff Richard, Neil Robinson, Roderick Robinson, David Scott,
Lawrence Whitaker

Additional Contributions: Paul Abertella, Shannon Appelcline, Simon
Bray, David Cake, Dave Camoirano, Melissa Camoirano, John Carnahan,
Charles Corrigan, David Dunham, Alex Ferguson, James Frusetta, Phil
Hibbs, Simon D. Hibbs, Jeff Kyer, Martin Laurie, Mark Leymaster, Julian
Lord, Rick Meints, Peter Metcalfe, Peter Nordstrand, Wesley Quadros,
Mikael Raaterova, Jamie Revell, Graham Robinson, Jonas Schiött, Gary
Sturgess, Ian Thomson, Nils Weinander

Chained Contests and Plot Edits from {\em Mythic Russia} © copyright
2006, 2010 Mark Galeotti; developed by Graham Robinson (for
\quotation{Chained Contests}) and added as Open Game Content here with
permission.

Material in {\em Section 1, Introduction} and {\em Section 2, Basic
Mechanics} © copyright 2018 Jonathan Laufersweiler and added as Open
Game Content here with permission.

Material in {\em Section 2, Basic Mechanics} © copyright 2020 Shawn
Carpenter and added as Open Game Content here with permission.

QuestWorlds SRD with annotations for individual contributions can be
found at GitHub:
\useURL[url1][https://github.com/ChaosiumInc/QuestWorlds/pulls]\from[url1].

Development of this version: Ian Cooper

Development Assistance for this version: Shawn Carpenter, Jonathan
Laufersweiler, James Lowder, Michael O'Brien, Jeff Richard, David Scott

Proofreading of this version: Martin Helsdon

\chapter[title={Introduction},reference={introduction}]

{\em QuestWorlds} is a roleplaying rules engine suitable for you to play
in any genre.

It is a traditional roleplaying game in that there is a GM and players.
The players play characters, each guided by the internal thoughts of
their character as to what decisions they make, and the GM plays the
world, including non-player characters (NPCs) and abstract threats.

It features an abstract, conflict-based resolution method and scalable,
customizable character descriptions. Designed to emulate the way
characters in fiction face and overcome challenges, it is suitable for a
wide variety of genres and play styles. It is particularly suited to
pulp genres (including their descendants, comic books) and cinematic,
larger-than-life action.

It is a rules-light system that facilitates beginning play easily, and
resolving conflicts in play quickly.

We refer to a rules-light but traditional roleplaying game as a
storytelling game, after Greg Stafford's definition in {\em Prince
Valiant}.

\section[title={Why QuestWorlds?},reference={why-questworlds}]

{\em QuestWorlds} is meant to facilitate your creativity, and then to
get out of your way.

It is well suited to a collaborative, friendly group with a high degree
of trust in each other's creativity. Characters in {\em QuestWorlds} are
described more in terms of their place in your imagination and the game
setting than by game mechanics.

If your group are often at odds and rely on their chosen rules kit as an
arbiter between competing visions of how the game ought to develop, or
use mechanical options to decide \quotation{what action to take,}
{\em QuestWorlds} is not a rules set that provides that structure. Make
sure to discuss with your group whether you are collectively on board
with trying a new play style dynamic, or if you would rather stick to
more structured systems.

\section[title={Version},reference={version}]

The first version of these rules {\em Hero Wars} was published in 2000
(ISBN 978-1-929052-01-1)

The second version {\em HeroQuest} was published in 2003 (ISBN
978-1-929052-12-7). We refer to this as {\em HeroQuest} 1e to
disambiguate.

The third version {\em HeroQuest}: Core Rules was published in 2009
(ISBN 978-0-977785-32-2). We refer to this as {\em HeroQuest} 2e.

{\em HeroQuest Glorantha} was published in 2015 (ISBN
978-1-943223-01-5). It is the version of the rules in {\em HeroQuest}
2e, presented for playing in Glorantha. We refer to this as
{\em HeroQuest} 2.1e.

{\em QuestWorlds} was published as a System Reference Document (SRD)
(this document) in 2020. The version of the rules here is slightly
updated, mainly to clarify ambiguities, from the version presented in
{\em HeroQuest} 2e and {\em HeroQuest} 2.1e. This makes this ruleset
{\em HeroQuest} 2.2e, despite the name change. However, to simplify we
identify this version as {\em QuestWorlds} 1e.

An Appendix lists changes in this version. As the SRD is updated we will
continue to track version changes there.

\section[title={Who Is This Document
For},reference={who-is-this-document-for}]

The primary audience for this document is game-designers who wish to
utilize the {\em QuestWorlds} rules framework to implement their own
game.

We also recognize that some people will use this document to learn about
the {\em QuestWorlds} system before purchasing it, and some players in
games where the GM has a rule book, may use this as a reference to help
understand the rules.

For that latter reason, we address the rules here to a player.

However, this remains a technical document with few examples, advice, or
other non-rules text to help you play your game, as such are beyond the
scope of this System Reference Document.

It is expected that the designers of games you play based on these rules
will include such guidance and context as is relevant to their game's
particular genre or setting, presented in a format better suited for
learning how to play.

\section[title={Numbering},reference={numbering}]

Sections within this document are numbered. This is for the benefit of
game designers and reviewers.

This does not imply that game designers need number the rules in their
own games.

Numbering however makes it easy to refer to rules in this document when
page numbers may vary by presentation format for the purposes of error
trapping or tracking changes. If you need to give us feedback about this
document, that will assist us.

\section[title={Participants},reference={participants}]

\subsection[title={Players},reference={players}]

You and your fellow players each create a Player Character (PC) to be
the \quotation{avatar} or \quotation{persona} whose role you will play
in the game. The PCs pursue various goals in an imaginary world, using
their {\bf abilities}, motivations, connections, and more to solve
problems and overcome {\bf story obstacles} that stand in their way.

When we say \quote{you} in this document we may mean the player or their
PC. Which should be clear from context, or explicitly noted.

\subsection[title={Game Master},reference={game-master}]

Your Game Master (GM) is the interface between your imagination and the
game-world in which the PCs have their adventures, describing the
people, places, creatures, objects, and events therein. Your GM also
plays the role of any Non-Player Characters (NPCs) with whom your PC
interacts in the course of your adventures.

We generally refer to the GM as \quote{your GM} in this document's
player-facing language. However, if you are the GM for a given game,
this naturally refers to you.

\chapter[title={Mechanics},reference={mechanics}]

In a {\em QuestWorlds} game, stories develop dynamically as you and your
GM work together to role-play the dramatic conflict between your group's
PCs in pursuit of their goals, and the challenges or threats that your
GM presents to stand in their way. Stories advance by two methods:
conflict and revelation. In conflict, your PC is prevented from
achieving their goals because there is something that must be
overcome---a {\bf story obstacle}---to gain a desired person, thing, or
even status---the {\bf prize}. In revelation, something must be
overcome---a {\bf story obstacle}---to learn a secret, uncover the past,
or reach understanding---the {\bf prize}.

Over the course of play, your GM will present various {\bf story
obstacles} as conflicts to the PCs, resulting in either {\bf victory} or
{\bf defeat} for your character, which determines whether or not you
gain the {\bf prize} you sought. These conflicts can represent any sort
of challenge you might face: fighting, a trial or debate, survival in a
harsh environment, out-wooing rival suitors, and so on.

Rather than mechanically addressing the individual tasks that make up
these conflicts, {\em QuestWorlds} usually assesses your overall
{\bf victory} or {\bf defeat} in a single {\bf contest} where you and
your GM make an opposed roll pitting your character's {\bf ability} vs
the {\bf resistance} the {\bf story obstacle} presents to you achieving
the {\bf prize}.

Whenever the GM presents a {\bf story obstacle} for you to overcome, you
should {\bf frame the contest} by describing what you are trying to
accomplish, the {\bf prize}, and which of your {\bf abilities} (see
§2.1) you want to use to achieve that {\bf prize}, and how.

Based on that {\bf framing} and other factors, your GM will assess what
{\bf resistance} the characters face.

You roll a twenty-sided die (D20) against your PC's {\bf ability}, and
your GM rolls a D20 against the {\bf resistance}. Your GM will assess
your overall {\bf victory} or {\bf defeat} in the contest based on the
{\bf success} or {\bf failure} of both rolls, and narrates the results
of your attempt to overcome the {\bf story obstacle} and gain the
{\bf prize} accordingly. The direction of the story changes, in either a
big or small way, depending on whether you gain the {\bf prize} or not.

We encourage your GM to work with your suggestions when narrating the
{\bf victory} or {\bf defeat}, but the final decision rests with them.

\section[title={Abilities},reference={abilities}]

Characters in {\em QuestWorlds} are defined by the {\bf abilities} they
use to face the challenges that arise in the course of their story.
Rather than having a standard list of attributes, skills, powers, etc.
for all characters, anything that you can apply to solve a problem or
overcome a {\bf story obstacle} could be one of your {\bf abilities}.
While your GM may provide some example {\bf abilities} to choose from
that connect your PC to a particular story or game world (whether
created by your GM or by the designer of a particular game), you get to
make up and describe most or all of your {\bf abilities}.

Some {\bf abilities} might be broad descriptions of your background or
expertise, like \quotation{Dwarf of the Chalk Hills} or
\quotation{Private Detective}---implying a variety of related
capabilities. Others might represent specific capabilities or assets
such as \quotation{Lore of the Ancients,} \quotation{Captain of the
Fencing Team,} or \quotation{The Jade Eye Medallion.}

Ultimately, {\bf abilities} are just names for the interesting things
your character can do.

\subsection[title={Flaws},reference={flaws}]

Your character may have one or more {\bf flaws}. A {\bf flaw} is an
{\bf ability} that you do not use to accomplish something, but instead
the GM uses to hinder you from accomplishing something, or invokes to
force you to act a certain way.

{\bf Flaws} may be psychological weaknesses such as
\quotation{Alcoholic} or \quotation{Heroin Addict,} or physical
weaknesses such as \quotation{One-Eye,} \quotation{Wheelchair-Bound} or
\quotation{Asthmatic.} A {\bf flaw} might also be a moral philosophy
such as \quotation{Code Against Killing,} \quotation{Pacifist,} or
\quotation{Radical Candor} that limits your behavior in some way. A
{\bf flaw} also might be a relationship such as a \quotation{Frail
Aunt,} \quotation{Single Dad} or \quotation{Blackmailed.}

Many {\bf flaws} describe attributes that can be viewed positively. By
making it a {\bf flaw} and not an {\bf ability} you are inviting your GM
to use it to make your life more difficult, not easier.

You should not use your {\bf flaw} to accomplish something; if you feel
that is likely, make it an {\bf ability} and flag to your GM that you
want them to draw on it as a {\bf flaw} at appropriate moments. In that
case, record both an {\bf ability} and a {\bf flaw} with the same name.

Ultimately, in {\em QuestWorlds} a {\bf flaw} is simply something that
you invite the GM to use to hinder or prevent your character doing
something. In return for the GM exercising the {\bf flaw} you gain
{\bf experience points} (see §8.1).

\subsection[title={Scores, Ratings and
Masteries},reference={scores-ratings-and-masteries}]

{\em QuestWorlds} {\bf abilities} are {\bf scored} on a {\bf rating} of
1--20, representing the {\bf target number} ({\bf TN}) you need to roll
or less to succeed on your roll during a {\bf contest} (see §2.3 for
more details).

Once your {\bf ability} passes 20, you would always be able to roll
under it on a D20. So to allow ability {\bf scores} to scale we use
tiers of capability we refer to as {\bf Mastery}. To reflect
{\bf abilities} (or {\bf resistances}) higher than 20, either
permanently through character advancement or temporarily with a
{\bf modifier} to a contest roll, note a {\bf mastery} for every 20
points in the ability, and treat what remains as the {\bf rating}. So,
for a {\bf score} of 27, we note one {\bf mastery} and a {\bf rating} of
seven, written as \quotation{7M}, and we write a {\bf score} of 21 as
1M.

The \quotation{M} after the {\bf rating} signifies {\bf mastery}. The
number in front of the M is the {\bf rating}, and represents the new
{\bf target number} you seek to roll or less. Whatever your roll, the
mastery then {\bf bumps} your {\bf result}. You {\bf bump} a
{\bf success} to a {\bf critical}, and {\bf bump} a {\bf failure} to a
{\bf success}. If you roll a {\bf critical} you can {\bf bump} down your
opponent. When both you and the resistance have {\bf masteries} they
cancel each other out.

Having a {\bf mastery} means that you {\bf succeed} most of the time and
{\bf critical} more often; you will only {\bf fail} when you roll a
{\bf fumble}, and have a higher chance of a {\bf critical} from rolling
under the {\bf TN}.

For example, Trevor Okafor is trying to hover a helicopter over a ravine
so that Bethany Ng can winch down to a stranded climber in high
cross-winds. The GM calls for a roll. Trevor Okafor has 31 in Pilot,
written as \quotation{11M}. Trevor's player rolls against a {\bf TN} of
11. They roll a 17 and fail, but Trevor's {\bf mastery} means the actual
{\bf result} is {\bf bumped} up to a {\bf success} on a 17. This beats
the GM's {\bf success} on a 13. Later Bethany Ng is trying to stabilize
the victim on the route back to hospital. Bethany has 27 in Medic,
written as \quotation{7M}. Bethany's player rolls against a TN of 7.
They roll a 4 and succeed, but the {\bf mastery} means that the actual
{\bf result} is bumped up to a {\bf critical} on a 4, which beats the
GM's {\bf success} on a 14.

Specific {\em QuestWorlds} games or genre packs may use other symbols
relevant to their setting or genre to denote {\bf mastery} instead of M.
If so, this should be clearly noted by their designers.

As a {\bf score} climbs, you may even gain multiple {\bf masteries} in
it. {\bf Mastery} tiers above one (representing an overall {\bf score}
or 41 or more) are marked with a number to the right of the M symbol.
Each successive {\bf score} increase over 20 becomes a new {\bf mastery}
tier. Thus, if you have 10M2, you have two {\bf masteries} and a
{\bf rating} of 10, (representing a total {\bf score} of 50). 10M3 means
that you have three {\bf masteries} and a {\bf rating} of 10, and so on.
Multiple {\bf masteries} result in multiple {\bf bumps} up, so with two
{\bf masteries} a critical {\bf bumps} to a success.

To simplify {\bf bumping}, when both you and the resistance have
{\bf masteries} they cancel out.

See §2.3.7 for more details on {\bf bumps}.

\subsubsection[title={No Relevant
{\bf Ability}},reference={no-relevant-ability}]

You may sometimes be faced with a {\bf story obstacle} for which you
have no relevant {\bf ability} whatsoever. In such cases, you may still
enter into conflict with the {\bf story obstacle} using a minimum base
{\bf target number} of 6 for your {\bf contest} roll. Like {\bf scores},
it may also be subject to {\bf modifiers}.

\subsubsection[title={Using Scores As
Thresholds},reference={using-scores-as-thresholds}]

{\em QuestWorlds} treats {\bf scores} as a measure of how effective you
are at solving problems with the {\bf ability}, and does not limit what
you can do with that {\bf ability}, provided your actions are credible
in genre. Where an important part of the genre is that certain uses of
the {\bf ability} are only available when you pass a threshold of
experience, often through overcoming {\bf story obstacles} to improve
the {\bf ability} in game, you may choose to set a threshold for those
{\bf abilities}. For example, a magic system might classify certain
supernatural effects as Apprentice, Journeyman, or Master level, and
require {\bf ratings} of 15, 5M, or 1M2 (respectively) in a relevant
{\bf ability} to even attempt them.

Such departures from abstraction should generally only be made where the
increased complexity they bring leads to rewarding choices in a key area
of interest to the setting or genre at hand. In most cases, you and your
GM can simply follow the fiction surrounding your {\bf ability} and its
context within the setting for guidance as to what applications of the
{\bf ability} are credible.

\section[title={Possessions and
Equipment},reference={possessions-and-equipment}]

Your character will generally be considered to have whatever equipment
is reasonably implied by your abilities. Having an \quotation{Athenian
Hoplite} {\bf ability} will mean that your character possesses bronze
armor, a shield, a spear, and a short-sword; while a \quotation{Country
Doctor} would be expected to have a well-stocked medical-bag and
possibly a horse & buggy in the right setting.

However, if you wish your character to possess something that is
particularly special, interesting, or unusual, you may also enumerate it
as a rated {\bf ability} in its own right, just like any other
{\bf ability} your character might use to solve a problem.

In play, the degree to which you can overcome {\bf story obstacles} with
your possessions depends not on any qualities inherent to the objects
themselves, but to the {\bf score} of your relevant {\bf ability}.
However the significance of various sorts of gear lies in the types of
actions you can credibly propose, and what their impact might reasonably
be. An \quotation{Invisibility Cloak} {\bf ability} implies very
different fictional capabilities than \quotation{Souped-up Muscle Car}
does.

Conversely, if in the course of play you find your character in a
situation without equipment essential to utilize an ability effectively,
or where your character's gear is poorly suited to the task at hand,
your GM may take that into account in assessing credibility-based
{\bf modifiers}.

\subsection[title={Wealth},reference={wealth}]

In {\em QuestWorlds}, wealth is treated as just another way to overcome
{\bf story obstacles}. Many characters may not even have an explicit
wealth {\bf ability}, with their wealth or assets instead implied by
{\bf abilities} representing their background, profession, or status.
Whether explicit or implied, the relevant {\bf score} is not an
objective measure of the size of your fortune, but instead indicates how
well you solve problems with money and resources.

\section[title={Contest Procedure},reference={contest-procedure}]

You choose an {\bf ability} relevant to the conflict at hand, describe
exactly what you are trying to accomplish, and how. Your GM may modify
these suggested actions to better fit the fictional circumstances, and
describes the actions of the NPCs or forces on the other side of the
conflict.

\subsection[title={Resolution Methods},reference={resolution-methods}]

The basic resolution methods are as follows:

\subsubsection[title={Assured Contest},reference={assured-contest}]

Some {\bf obstacles} don't require a roll to overcome. You'll just do it
and keep going, much as you get dressed in the morning or drive your car
to work. We call these kinds of contests {\bf assured} contests because
your {\bf victory} is assured. Your GM may want to describe your
{\bf victory} as a sweat inducing challenge for you, even though there
is no risk of {\bf defeat}, to highlight the heroic struggle of your PC
to beat the obstacle, nonetheless.

As your character advances, the challenges that qualify for assured
contests will become more complex. If you face a driving challenge, the
bar for assured will be much lower for a champion Formula 1 racer than a
typical commuter.

{\bf Assured contests} are the GM's primary tool to establish your
character's competence. This makes them one of the most powerful and
frequently used tools in a GM's tool chest. Remember, your GM doesn't
have to, and usually shouldn't, advise you you're involved in an
{\bf assured contest}, so it's best to treat all {\bf contests} as if
your skin is on the line.

Your GM may also use an {\bf assured contest} when there is no
interesting story branch from {\bf defeat}. If failing to open the
derelict spaceship's hatch means that the story of your exploration of
the ancient space hulk would end abruptly, your GM may choose to make it
an {\bf assured contest}. {\bf Assured contests} may be used to find
clues when your GM is running a mystery and correct application of one
of your {\bf abilities} should reveal the information and allow the
story to continue, over becoming mired due to a missed roll and missing
clue.

Sometimes your GM will decide potential complications could arise in
overcoming a {\bf story obstacle}. Or they may want to give you a
{\bf bonus} if you do particularly well. If so, they will call for you
to make a die roll even though your {\bf victory} is not in question.
Your GM will use your die roll {\bf result} to decide if any unforeseen
{\bf consequences} or {\bf benefits} arose from your actions.

If you roll a {\bf failure}, you still beat the obstacle, but you also
suffer an adverse {\bf consequence} (see §2.7.1). The nature of this
adversity is up to the GM. It will probably be a {\bf penalty} involving
the same tactic you used in this one (because you exhausted yourself,
sprained an ankle, embarrassed yourself in front of your peers, etc.) or
to the value of one of your relationships.

If you achieve a high {\bf result}, you'll receive a {\bf benefit} from
your effort (see §2.7.2). Again, this is up to the GM to define. It
could be a {\bf bonus} to the {\bf tactic} used in the {\bf contest}, or
to of one of your relationships, etc.

{\bf Fumbles} and {\bf criticals} always result in an unexpected
difficulty or reward!

An {\bf assured contest} can be summarized as follows:

\startitemize[n,packed][stopper=.]
\item
  You and your GM agree upon the terms of the {\bf contest}.
\item
  The GM may decide that you simply gain the {\bf victory} and there are
  no {\bf consequences} or {\bf benefits} beyond that.
\item
  If not the GM conducts a contest.
\item
  You roll a D20 vs your relevant {\bf ability}, while your GM rolls a
  D20 vs the {\bf resistance}.
\item
  Your GM compares the {\bf success} or {\bf failure} of the two rolls,
  and assesses any {\bf consequences} or {\bf benefits}.
\item
  Your GM then narrates how you obtained your {\bf victory} and any
  {\bf benefits} or {\bf consequences} that arose.
\stopitemize

\subsubsection[title={Simple Contest},reference={simple-contest}]

The {\bf simple contest} is {\em QuestWorlds}' primary resolution
mechanic for overcoming {\bf story obstacles}, and is used most often
where the outcome is uncertain. It also provides the foundation for
other types of uncertain {\bf contest}, including several {\bf long}
ones. As such, it receives both an overview of key concepts here as well
as a more detailed treatment in §4.

A {\bf simple contest} can be summarized as follows:

\startitemize[n,packed][stopper=.]
\item
  You and your GM agree upon the terms of the {\bf contest}.
\item
  You roll a D20 vs your relevant {\bf ability}, while your GM rolls a
  D20 vs the {\bf resistance}.
\item
  Your GM compares the {\bf success} or {\bf failure} of the two rolls,
  and assesses your overall {\bf victory} or {\bf defeat}.
\item
  Your GM then narrates the {\bf outcome} of the conflict as
  appropriate.
\stopitemize

If you enter into conflict with another player rather than a {\bf story
obstacle} presented by your GM, you both roll your relevant abilities
for the {\bf contest} instead of against a GM-set {\bf resistance}, and
your GM interprets the {\bf results}, as described above.

\subsection[title={Framing the Contest},reference={framing-the-contest}]

\subsubsection[title={Contest Framing
Overview},reference={contest-framing-overview}]

When a conflict arises during the game, you and your GM start by clearly
agreeing on:

\startitemize[packed]
\item
  What goal you are trying to achieve. We call this the {\bf prize}.
\item
  What the {\bf story obstacle} is you are trying to overcome.
\item
  What {\bf tactic} you are using to overcome it.
\stopitemize

This process is called {\bf framing the contest}.

\subsubsection[title={Conflict: Goals vs
Obstacles},reference={conflict-goals-vs-obstacles}]

{\bf Contests} in {\em QuestWorlds} don't simply tell you how well you
performed at a particular task: they tell you whether or not you
overcame a {\bf story obstacle}, which moves the story in a new
direction. Unlike some other roleplaying games, a {\bf contest} in
{\em QuestWorlds} does not resolve a task, it resolves the whole
{\bf story obstacle}.

If you need secret records which are stored in a vault within a
government compound, your goal is to get the information---while the
fact that it is secured against your access is a {\bf story obstacle}
you must overcome to attain that goal. Overcoming that {\bf story
obstacle} may involve many possible tasks, evading guards, lock-picking,
forging credentials, etc.---but the {\bf contest} doesn't address those
individually. The {\bf contest} is framed around the entire conflict
against the {\bf story obstacle} as a whole.

In a fight, your {\bf story obstacle} may be the opponents themselves,
who you are fighting to capture or kill. Just as often you are seeking
another goal and you might just as easily attain it by incapacitating or
evading your foes. In this case, beating the enemy is a task, not the
{\bf story obstacle}. For example, if an {\bf ally} has been accused of
treason by the King, your goal could be to prove the {\bf ally's}
innocence. The power of the King threatening your {\bf ally} is a
{\bf story obstacle} to be overcome, and a trial by combat could be a
{\bf contest} to resolve the conflict with an {\bf ability} like
\quotation{Knight Errant.}

In a court trial, your goal is likely a particular verdict, while the
{\bf story obstacle} might be the opposing lawyer, an unjust law, or
even the justice system itself. In this case, jury selection, a closing
argument, revelatory evidence, or legal procedural challenges are tasks,
not the entire {\bf story obstacle}. The overall conflict encompasses
all those things.

A conflict to overcome a {\bf story obstacle} moves the story forward
when it is resolved. If it is merely a step toward resolving a
{\bf story obstacle} it is a task and not a conflict. While those
component tasks may be interesting parts of narrating {\bf tactics} and
{\bf results}, your GM should be sure to look for the {\bf story
obstacle} in conflict when framing a {\bf contest}.

If there is no {\bf story obstacle} to your actions, your GM should not
call for a {\bf contest} but simply let you narrate what you do,
provided that seems credible.

For example, you are traveling from one star system to another. In the
next star system you hope to confront the aged rebel who holds
long-forgotten secrets that could bring freedom to the galaxy. Your GM
feels there is no useful {\bf story obstacle} for you to {\bf contest}
against, and so lets you describe heading down to the spaceport to
secure a ship, meeting the captain and crew of your vessel, and
traveling to the next world. Your GM encourages you to summarize what
happens quickly so you can get to the meeting with the old rebel. Your
GM knows that will be the real {\bf story obstacle}: convincing the old
rebel to part with their secrets.

\subsubsection[title={Tactics},reference={tactics}]

You either choose an {\bf ability} that represents any \quote{key
moment} in overcoming that {\bf story obstacle}, or a broad
{\bf ability} that lets you overcome the whole {\bf story obstacle}. We
call this choosing a {\bf tactic}.

The {\bf rating} from your {\bf ability score}, after adding any
{\bf augment} (see §2.5) or {\bf modifiers} (see §2.4), relevant
{\bf consequences} or {\bf benefits} (see §2.7) is your {\bf target
number}.

Your {\bf tactic} might describe your using an {\bf ability} that helps
you overcome a task within the {\bf story obstacle}: sneaking past the
guards, picking the locks, choosing the right jury or skewering your
opponent with your foil. Or, your {\bf tactics} might describe using a
broad ability like \quotation{Ninja}, \quotation{Lawyer}, or
\quotation{Fencer} to overcome all those challenges that might form part
of the {\bf story obstacle}. Either way, if you succeed at that roll,
you overcome the whole {\bf story obstacle}. Or by failing at that roll,
you fail to overcome the {\bf story obstacle}, not just fail at one
task.

When deciding on your {\bf tactic}, focus on how your unique abilities
would help you overcome the {\bf story obstacle}. This as the \quote{key
moment} where we focus on your PC. Use this moment to reveal your PC's
strengths to the group.

Your GM will determine if your {\bf tactic} passes a {\bf credibility
test}. If you try to jump a 100 meter gap or run faster than a speeding
car, your action is not credible and your GM will ask you to choose a
different {\bf tactic}.

Credibility depends on the genre, as what is not credible in a gritty
police procedural might be in pulp, where you might be able to leap from
a bridge onto a speeding train. If in dispute, your GM should discuss
with the group whether they consider your {\bf tactic} credible for the
genre.

{\bf Extraordinary abilities} in some genres give you the capability to
do the incredible. For example in a superhero genre you might fly or be
invulnerable to bullets, while in a fantasy genre you might hurl magical
lightning bolts. A genre pack for the game should help define what
incredible {\bf tactics} are allowed for that game as part of an
{\em Extraordinary Powers Framework}.

The GM can narrate the remaining tasks that make sense of the story
depending on your {\bf success} with that roll, or have them occur
\quote{off-stage} for speed. Think of the way TV or Cinema often cuts to
the key moment of drama in a break-in, over showing us the whole heist
from beginning to end.

\subsubsection[title={No Repeat
Attempts},reference={no-repeat-attempts}]

A {\bf contest} represents all of your attempts to overcome a {\bf story
obstacle}. If you lose it means that no matter how many times you tried
to solve the problem, you finally had to give up. You can try again only
if you use a new {\bf tactic} to overcome the {\bf story obstacle}.

\subsection[title={Rank},reference={rank}]

Your GM uses a {\bf rank} when choosing to adjust a {\bf target number}.
That adjustment may come from an {\bf augment} (see §2.5),
{\bf modifier} (see §2.4), {\bf benefits} or {\bf consequences} (see
§2.7). Your GM also uses the same {\bf rank} when determining the
{\bf resistance}.

Bonuses to the dice roll use the scale of values: +3, +6. +9, +M,
+M2\crlf
Penalties to the dice roll use the scale of values: −3, −6, −9, −M, −M2

You should be able to memorize these values in play, and just go up or
down the scale, instead of translating a {\bf rank} to a value. For
convenience the following table shows the scale of {\bf ranks}.

\startplacetable[title={2.3.3.1 RANKS TABLE}]
\startxtable
\startxtablehead[head]
\startxrow
\startxcell[align=middle] Rank \stopxcell
\startxcell[align=middle] Value \stopxcell
\stopxrow
\stopxtablehead
\startxtablebody[body]
\startxrow
\startxcell[align=middle] 5 \stopxcell
\startxcell[align=middle] +M2 \stopxcell
\stopxrow
\startxrow
\startxcell[align=middle] 4 \stopxcell
\startxcell[align=middle] +M \stopxcell
\stopxrow
\startxrow
\startxcell[align=middle] 3 \stopxcell
\startxcell[align=middle] +9 \stopxcell
\stopxrow
\startxrow
\startxcell[align=middle] 2 \stopxcell
\startxcell[align=middle] +6 \stopxcell
\stopxrow
\startxrow
\startxcell[align=middle] 1 \stopxcell
\startxcell[align=middle] +3 \stopxcell
\stopxrow
\startxrow
\startxcell[align=middle] 0 \stopxcell
\startxcell[align=middle] 0 \stopxcell
\stopxrow
\startxrow
\startxcell[align=middle] −1 \stopxcell
\startxcell[align=middle] −3 \stopxcell
\stopxrow
\startxrow
\startxcell[align=middle] −2 \stopxcell
\startxcell[align=middle] −6 \stopxcell
\stopxrow
\startxrow
\startxcell[align=middle] −3 \stopxcell
\startxcell[align=middle] −9 \stopxcell
\stopxrow
\startxrow
\startxcell[align=middle] −4 \stopxcell
\startxcell[align=middle] −M \stopxcell
\stopxrow
\stopxtablebody
\startxtablefoot[foot]
\startxrow
\startxcell[align=middle] −5 \stopxcell
\startxcell[align=middle] −M2 \stopxcell
\stopxrow
\stopxtablefoot
\stopxtable
\stopplacetable

\subsection[title={Resistance},reference={resistance}]

Your GM chooses a {\bf resistance} to represent the difficulty of the
{\bf story obstacle}.

When setting {\bf resistances} it is important to understand that whilst
traditional roleplaying games simulate an imaginary reality,
{\em QuestWorlds} emulates the techniques of fictional storytelling.

Understanding this distinction will help you to play the game in a
natural, seamless manner.

For example, let's say that your GM is playing a game inspired by
fast-paced, non-fantastic, martial arts movies in a contemporary
setting. You are running along a bridge, pacing a hovercraft piloted by
the main bad guy. You want your character, Joey Chun, to jump onto the
hovercraft and punch the villain's lights out.

In a traditional, simulative game, your GM would determine how hard this
is based on the physical constraints you've already described. In doing
so, they would come up with imaginary numbers and measurements. Your GM
would have to work out the distance between bridge and hovercraft.
Depending on the rules set, they might take into account your relative
speed to the vehicle. Then they would use whatever resolution mechanic
the rules provide them with to see if Joey succeeds or fails. If you
blow it, your GM will probably consult the falling rules to see how
badly you injure yourself (if you land poorly), or the drowning rules,
if you end up in the river.

In {\em QuestWorlds}, your GM starts not with the physical details, but
with the proposed action's position in the storyline. They consider a
range of narrative factors, from how entertaining it would be for you to
{\bf succeed}, how much {\bf failure} would slow the pacing of the
current sequence, and how long it has been since you last scored a
thrilling {\bf victory}. If, after this, they need further reference
points, your GM can draw inspiration more from martial arts movies than
the physics of real-life jumps from bridges onto moving hovercraft.
Having decided how difficult the task ought to be dramatically, your GM
will then supply the physical details as color, to justify their choice
and create suspension of disbelief---the illusion of authenticity that
makes us accept fictional incidents as credible on their own terms. If
they want Joey to have a high chance of {\bf success}, your GM describes
the distance between bridge and vehicle as impressive (so it feels
exciting if you make it) but not insurmountable (so it seems believable
if you make it).

In other words, in {\em QuestWorlds} your GM will pick a
{\bf resistance} based on dramatic needs and then justify it by adding
details into the story.

Your GM determines the {\bf resistance} from a {\bf base resistance}
modified by a {\bf rank} depending on their view of how difficult the
obstacle is for you. Increasing {\bf ranks} make it harder to succeed,
and decreasing {\bf ranks} easier.

By default, the {\bf base resistance} starts at 14. The {\bf rank} never
reduces the resistance value below 6.

All {\bf contests} use the {\bf base resistance} + {\bf rank}, except
for {\bf contests} to determine {\bf augments}. {\bf Augmenting} always
faces the {\bf base resistance}.

\subsection[title={Die Rolls},reference={die-rolls}]

To determine how well you use an {\bf ability}, roll a 20-sided die
(D20). At the same time, your GM rolls for the {\bf resistance}.

Compare your rolled number with your {\bf Target Number} to determine
the {\bf result}, a level of {\bf success} or {\bf failure} for the roll
(not the {\bf contest} as a whole).

\startitemize[packed]
\item
  {\bf Critical}: If the die roll is equal to the {\bf TN} (even when
  the {\bf TN} is 20), you succeed brilliantly. This is the best
  {\bf result} possible.
\item
  {\bf Success}: If the die roll is less than the {\bf TN}, you succeed,
  but there is nothing remarkable about the success.
\item
  {\bf Failure}: If the die roll is greater than the {\bf TN} but not a
  {\bf fumble}, you fail. Things do not happen as hoped.
\item
  {\bf Fumble}: If the die roll is 20, you fumble (except when the
  {\bf TN} is 20, when it is a {\bf critical}). You fail miserably. This
  is the worst {\bf result} possible.
\stopitemize

Note that whatever your {\bf result} the {\bf outcome} will depend on
comparing your roll with your opponent's. So you might {\bf succeed},
but still lose the {\bf prize}. At the same time, your GM should take
into account your {\bf result} when narrating the {\bf outcome}, and not
use your incompetence as a reason you failed to gain the {\bf prize} if
you succeeded, instead focusing on the {\bf resistance}'s superiority
despite your {\bf success}.

\subsection[title={Outcome},reference={outcome}]

Your roll and that of your GM's roll are compared to determine your
overall {\bf outcome} which will be either {\bf victory} or {\bf defeat}
for the {\bf contest} as a whole.

If you have a better {\bf result} than the GM, then you have a
{\bf victory} and you gain the {\bf prize} set out when the
{\bf contest} was framed.

If you have a worse {\bf result}, then you are {\bf defeated} and do not
gain the {\bf prize}.

If you both have the same {\bf result}, the higher roll wins.

If your rolls tie, then it is a standoff.

A {\bf critical} is a better {\bf result} than a {\bf success} which is,
in turn, a better {\bf result} than a {\bf failure}, which is a better
{\bf result} than a {\bf fumble}.

\subsubsection[title={Narrating
Outcomes},reference={narrating-outcomes}]

You GM narrates the contest {\bf outcome}. Their narration should take
into account the {\bf prize} and the {\bf tactics} used by each side.
Your GM may invite you to contribute more detail on your actions as part
of that narration, if they wish. But the GM is the final arbiter of the
how the story progresses as a result of the rolls---provided they
respect the {\bf outcome} in which you win or lose the {\bf prize}.

Your GM should bear in mind your {\bf result} when describing the
outcome. For example, if you {\bf succeeded}, but the {\bf resistance}
{\bf succeeded} better, the GM should describe your actions as
successful, but the {\bf resistance} as doing better. If your
{\bf result} was a {\bf critical} and the {\bf resistance's}
{\bf result} was a {\bf fumble}, your GM should describe a crushing
{\bf defeat} in which your adversary is clearly outclassed.

The GM is narrating a car chase through the busy streets of New Los
Angeles. The PCs are trying to catch the demon-worshipper Ath'Zul who
has stolen The Eye of Lorus from a museum. Some examples of how the GM
might interpret {\bf outcomes} as follows:

\startitemize[packed]
\item
  PC {\bf Success} ({\bf Better Roll}) vs.~Ath'Zul {\bf Success}:
  Ath'Zul tries to shake the PCs, his hover bike weaving in and out of
  traffic, but the PCs are always on his tail, and catch him at the
  lights on Bradbury Junction.
\item
  PC {\bf Success} vs.~Ath'Zul {\bf Failure}: Ath'Zul tries to shake the
  PCs, his hover bike weaving in and out of traffic, but the PCs force
  him off the road, where his bike loses repulsor lift and halts.
\item
  PC {\bf Success} vs.~Ath'Zul {\bf Fumble}: Ath'Zul tries to shake the
  PCs, his hover bike weaving in and out of traffic, but he crashes into
  a parked car, spilling Ath'Zul and the stolen artefact over the road.
\stopitemize

Your GM should avoid robbing your PC of competence by describing your
{\bf defeat} as due to your incompetence when you may have rolled a
{\bf critical} or a {\bf success}.

\subsubsection[title={Confusing Ties},reference={confusing-ties}]

Your GM will describe most tied {\bf outcomes} as inconclusive
standoffs, in which neither of you gets what you wanted.

In some situations, ties become difficult to visualize. Chief among
these are {\bf contests} with binary {\bf outcomes}, where only two
possible results are conceivable.

Your GM can either change the situation on such a tie, introducing a new
element that likely renders the original {\bf prize} irrelevant to both
participants, or they can resolve the tie in your favor as a
{\bf victory}.

\subsection[title={Bumps},reference={bumps}]

A {\bf bump} affects the degree of {\bf success} or {\bf failure} of
your die roll. A {\bf bump} up improves your {\bf result} by one step,
changing a {\bf fumble} to a {\bf failure}, a {\bf failure} to a
{\bf success}, or a {\bf success} to a {\bf critical}. {\bf Bump} ups
come from two sources: {\bf masteries} and {\bf story points} (applied
in that order). A {\bf bump} down reduces the {\bf result} by one step,
changing a {\bf critical} to a {\bf success}, a {\bf success} to a
{\bf failure}, or a {\bf failure} to a {\bf fumble}. {\bf Bump} downs
come from one source: {\bf masteries}.

{\bf Bumps} always affect {\bf results} not {\bf outcomes}, although the
outcome could change as an effect of gaining a different {\bf result}.

\subsubsection[title={Bump Up with
Mastery},reference={bump-up-with-mastery}]

If you're engaged in a {\bf contest} against a {\bf resistance}, and you
have a {\bf target number} of 10M versus a {\bf resistance} of 10, you
enjoy an advantage. You get a {\bf bump} to your die roll from that
{\bf mastery}.

You get one {\bf bump} up for each level of {\bf mastery} your PC has
greater than your opponent's. So against a {\bf resistance} of 14 a PC's
{\bf target number} of 7M is treated as 7 vs.~14 but we {\bf bump} the
{\bf result} one step in the favor of the PC; a PC's {\bf ability} of
3M2 is treated as 3 vs.~14 but we {\bf bump} the {\bf result} two steps
in favor of the PC.

Opposed {\bf masteries} cancel out, each contestant reducing their
{\bf score} by the same number of {\bf masteries} until only one or
neither of them has {\bf masteries}. If you have two {\bf masteries},
then you enjoy the same great advantage over an opponent with a single
{\bf mastery} as someone with one {\bf mastery} has over an opponent
with no {\bf masteries}. If you have an advantage of two or more
{\bf masteries} over an opponent, you can pretty much count on pounding
them into the dust.

This allows {\em QuestWorlds} to represent large differences in
{\bf ability} or {\bf resistance}.

If the {\bf resistance} is one or more levels of {\bf mastery} greater
than you, your GM can {\bf bump} up their roll.

\subsubsection[title={Bump Up with Story
Points},reference={bump-up-with-story-points}]

You can spend a {\bf story point} to {\bf bump} up any {\bf result} by
one step. You can decide to use a {\bf story point} for a {\bf bump}
after the die roll {\bf results} are calculated (including any
{\bf bump} ups from {\bf masteries}).

You can only spend one {\bf story point} per roll.

\subsubsection[title={Bump Down with
Mastery},reference={bump-down-with-mastery}]

If you have a {\bf critical} and still have one or more {\bf bumps} left
from your advantage over your opponent, you can use them to {\bf bump}
down your opponent, since you cannot get a {\bf result} better than a
critical. {\bf Bump} down your opponent's result for each mastery
remaining. So against a {\bf resistance} of 14, if you have an
{\bf ability} of 7M, you have one {\bf bump} of the result. If you
{\bf critical} and the resistance gets a {\bf success}, then as you
cannot {\bf bump} up from your critical, you instead {\bf bump} the
resistance down to a {\bf failure.}

If the {\bf resistance} has unspent {\bf bumps} then the GM can
{\bf bump} you down if they have a {\bf critical}.

\subsubsection[title={Bump Up with Negative
Mastery},reference={bump-up-with-negative-mastery}]

If you are facing a {\bf resistance} that has negative {\bf mastery},
your GM should treat it as an {\bf assured contest}. If the GM wants you
to roll, to find out how well you did, your roll is {\bf bumped} up by
the {\bf resistance}.

\section[title={Modifiers},reference={modifiers}]

Your {\bf score} represents a general {\bf ability} to succeed in the
narrative, but modifiers reflect specific conditions that may make it
easier or harder to overcome particular {\bf story obstacles}. They are
applied to your {\bf ability} to get a final {\bf target number}
({\bf TN}).

Positive modifiers are called {\bf bonuses}; negative modifiers are
called {\bf penalties}.

{\bf Bonuses}, may raise your {\bf ability} high enough to gain a
{\bf mastery}, in which case you get the {\bf bumps} up or down that a
{\bf mastery} would normally supply.

{\bf Penalties}, may lower an {\bf ability} to the point where it loses
one or more {\bf masteries}. In this case, you lose the {\bf bumps} up
or down you would normally get.

Your GM should only use modifiers to alter your {\bf target number} due
to unusual circumstances you helped to create, or have some control
over. If an unusual situation applies to a {\bf resistance}, the GM
should choose a {\bf resistance} that reflects that. Modifiers never
apply to the {\bf resistance}.

If {\bf penalties} reduce your {\bf target number} to 0 or less, any
attempt to use it automatically {\bf results} in {\bf failure}. You must
find another way to achieve your aim.

\subsection[title={Stretches},reference={stretches}]

When you propose an action using an {\bf ability} that seems completely
inappropriate, your GM rules it impossible. If you went ahead and tried
it anyway, you'd automatically fail---but you won't, because that would
be silly.

In some cases, though, your proposed match-up of action and
{\bf ability} is only somewhat implausible. A successful attempt with it
wouldn't completely break the illusion of fictional reality---just
stretch it a bit.

Using a somewhat implausible {\bf ability} is known as a {\bf stretch}.
If your GM deems an attempt to be a {\bf stretch}, the PC suffers a −3,
−6, −9 {\bf penalty}, or a {\bf bump} down, to their {\bf target
number}, depending on how incredible the {\bf stretch} seems to the GM
and other players. Your GM should {\bf penalize} players who try to
create a \quote{do anything} {\bf ability} that they then {\bf stretch}
to gain from raising fewer {\bf abilities} in advancement to ensure
balance with other PCs.

A default {\bf stretch penalty} should be −6.

The definition of {\bf stretch} is elastic, depending on genre.

Your GM should not impose {\bf stretch penalties} on action descriptions
that add flavor and variety to a scene, but do not fundamentally change
what you can do with your {\bf ability}. These make the scene more fun
but don't really gain any advantage.

\subsection[title={Situational
Modifiers},reference={situational-modifiers}]

Your GM may also impose {\bf modifiers} when, given the description of
the current situation, believability demands that you should face a
notable {\bf bonus} or {\bf penalty}. Your GM should choose
{\bf modifiers} of +6, +3, --3, or --6. {\bf Modifiers} of less than 3
don't exert enough effect to be worth the bother. Those higher than 6
give the situational {\bf modifier} a disproportionate role in
determining {\bf outcomes}.

During a {\bf long contest}, they should typically last for a single
{\bf round}, and reflect clever or foolish choices.

\section[title={Augments},reference={augments}]

You may sometimes face {\bf contests} where more than one {\bf ability}
may be applicable to the conflict at hand. In such cases, you may
attempt to use one {\bf ability} to give a supporting bonus to the main
ability you are using to frame the {\bf contest}. This is called an
{\bf augment}. For example, if your character has the {\bf abilities}
\quotation{The Queen's Intelligencer} and \quotation{Master of
Disguise}, you might use the latter to {\bf augment} the former when
infiltrating a rival nation's capitol. Similarly, a character with
\quotation{Knight Errant} and \quotation{My Word is my Bond}
{\bf abilities} might use one to {\bf augment} the other when in
conflict with a {\bf story obstacle} the character has sworn to
overcome.

Abilities that represent special items, weapons, armor, or other
noteworthy equipment can be a common source of {\bf augments}. However,
this grows tired if over-used and you should try and restrict repeated
use of equipment in this way to {\bf contests} where they are
particularly interesting or apropos.

{\bf Augments} can also come from other characters' {\bf abilities} if
one character uses an {\bf ability} to support another's efforts rather
than directly engaging in the {\bf contest}. {\bf Augments} can even
come from outside resources like support from a community, see §8, or
other circumstantial help.

If you have a good idea for an {\bf augment}, propose it to your GM
while the {\bf contest} is being framed. When making your proposal,
describe how the {\bf augmenting ability} supports the main one in a way
that is both {\em entertaining} and {\em memorable}. Don't just hunt for
mechanical advantage, show your group more about your PC when you
{\bf augment}, their attitudes, passions, or lesser known
{\bf abilities}. If you are {\bf augmenting} with a {\bf broad ability}
like \quotation{Fool's Luck}, be prepared to describe the unlikely
events that tilt the scales in your favor. Your GM will decide whether
the {\bf augment} is justified and can refuse boring and uninspired
attempts to {\bf augment}, where you are just looking for a bonus to
your roll and not adding to the story.

You may only use one of your own {\bf abilities} to {\bf augment} the
{\bf ability} you are using in the {\bf contest}, and you may not use an
{\bf ability} to {\bf augment} itself. You may not use a {\bf breakout}
to augment it's parent {\bf keyword}, or another {\bf breakout} from the
parent {\bf keyword}. However, {\bf augments} from other players
supporting you can add together with your own, along with other
{\bf modifiers}, including those from {\bf benefits} and from {\bf plot
augments}.

If your GM accepts your {\bf augment} proposal, it will be resolved by
one of the methods below. The main {\bf contest} then proceeds as
normal, with any bonus from the {\bf augment} added onto the {\bf score}
of the {\bf ability} chosen when {\bf framing the contest}. The
{\bf augment} remains in effect for the duration of the {\bf contest}.

To grant an {\bf augment} to yourself, or another PC, in an upcoming
{\bf contest}, your GM decides how uncertain the benefit of your
{\bf ability} is to the contest. In some genres, certain abilities, such
as magic in a fantasy setting, may always be uncertain.

If use of the {\bf ability} to augment seems unlikely to fail, your GM
may treat it as an {\bf assured contest}. As with any {\bf assured
contest} GM might still ask you to roll if there is a risk that the
{\bf augment} results in a {\bf penalty} to other {\bf abilities} such
as resources or relationships.

If your GM decides it is uncertain whether your {\bf ability} can
{\bf augment} this contest, you engage in a {\bf simple contest} against
the {\bf base resistance} before the main contest begins to determine
whether the {\bf augment} attempt results in an advantage. Frame this
{\bf augment contest} with your GM, making it clear how your supporting
{\bf ability} will achieve the goal of making your {\bf ability} in the
main {\bf contest} more effective. {\bf Augment contests} may not
themselves be {\bf augmented}, and if your {\bf augment} attempt ends in
{\bf defeat}, you may not make another attempt at an {\bf augment} for
the main {\bf contest}.

If you are victorious in the {\bf augment contest}, your GM will award a
{\bf bonus} of +3 to the {\bf ability} used in the main {\bf contest}.
If your description of how you were using the {\bf augmenting ability}
was particularly entertaining, your GM may increase the {\bf bonus} to
+6.

\section[title={Flaws},reference={flaws-1}]

During play your GM may decide that your {\bf flaw} has been triggered.
A {\bf flaw} might apply to the {\bf tactic} you are using in upcoming
{\bf contest}, when it is called a {\bf hindrance} (see §2.6.1).
Alternatively a {\bf flaw} might simply come into play when you want to
describe your PC acting in a certain way, and your GM feels that one of
your {\bf flaws} could prevent this, or your GM feels that a situation
raises a challenge that means one of your {\bf flaw} means would lead to
you responding in a certain way. (see §2.6.2).

\subsection[title={Hindrance},reference={hindrance}]

if you describe a {\bf tactic} for a {\bf contest} that is in conflict
with a {\bf flaw}, your GM may decide to impose a {\bf penalty} called a
{\bf hindrance} against you in the upcoming {\bf contest}. Your GM may
also use an {\bf ability} on your character sheet against you in this
way too, if appropriate. This may be the case for relationships you
have, philosophies you espouse, or groups you belong to. Your GM should
trigger a {\bf hindrance} from a {\bf keyword} more frequently than from
a stand-alone {\bf ability} that is not a {\bf flaw}. Your GM may treat
their use of an {\bf keyword} as a {\bf flaw} as a {\bf stretch} when
invoking it.

Your GM should follow a similar approach to {\bf augments} when applying
a {\bf hindrance} (see §2.5). They should ask themselves if it is
{\em fresh}, {\em interesting} or {\em illuminates character}. In a
movie of book would your {\bf flaw} be prominent here?

If your GM feels that there is no uncertainty as to whether the
{\bf flaw} applies to your {\bf tactic} in the contest they apply a
{\bf penalty} of −3 or −6 depending on how serious a handicap the
{\bf flaw} is. (This is, in effect a reverse {\bf assured contest} in
the GM's favor). You may also suggest to the GM that you feel the
{\bf flaw} would apply without a {\bf contest}.

If your GM feels that it is uncertain as to whether the {\bf flaw}
hinders you, or you are able to overcome it, and you agree that you wish
to try, treat it as a {\bf simple contest}. Roll the rating of your
{\bf flaw} against the {\bf base resistance}. On a {\bf victory}, you
receive a {\bf penalty} of −3 or −6 depending on how significant a
{\bf hindrance} your GM feels the flaw is to your {\bf tactic}. On a
{\bf defeat}, you overcome the {\bf flaw}. You may want to describe how
you overcome your {\bf flaw} to use your {\bf tactic}.

When you experience a {\bf penalty} due to a flaw, you gain an
{\bf experience point} (see §8.1).

\subsection[title={Act according to your
flaw},reference={act-according-to-your-flaw}]

At times the direction of the story you are all telling may place your
PC in situations when it seems likely they would act according to their
{\bf flaw}. The addict may reach for drink or drugs following an
emotional setback, a lust for vengeance may come between your PC and
showing mercy, prejudices or bigotry may prevent your from seeing others
positively.

If you chose to act according to your {\bf flaw} there is no contest,
simply describe your character behaving as the {\bf flaw} dictates. This
might result in a {\bf hindrance} to further actions (see §2.6.1)

If you wish to act against your {\bf flaw}, your {\bf tactic} must pass
a {\bf credibility test} as to how you try overcome your {\bf flaw} in
this instance. In effect, pick an {\bf ability} to resist the {\bf flaw}
with. Then you must obtain a {\bf victory} in a {\bf simple contest}
against your {\bf flaw}. On a {\bf victory} you may act in a way that
contradicts your {\bf flaw}.

If you submit to your {\bf flaw}, your GM might impose a {\bf hindrance}
on further actions (see §2.6.1). You should not contest this
{\bf hindrance} unless the situation is not related to the one which
triggered your {\bf flaw} in this instance, or significant time has now
passed.

Your GM may impose a {\bf penalty} against an {\bf ability} if you gain
the {\bf victory} against your {\bf flaw} representing your struggle
against your inner nature, violating dearly held principles, or letting
down dependents. This is often true where the GM invokes a flaw from a
{\bf keyword}. For example, if you had they {\bf keyword}
\quotation{Gangster} and decide to inform on a fellow mobster, your GM
might invoke the {\bf flaw} of \quotation{Code of Silence} even if it is
not a {\bf breakout} under you {\bf keyword}; this is particularly
appropriate where facts such as the \quotation{Code of Silence} have
been established in game. Even if you overcome your {\bf flaw}, and
inform on your fellow mobster, the GM might still impose a {\bf penalty}
on use of the {\bf keyword} to interact with your crime family for
having breached the \quotation{Code of Silence.}

Similarly, your GM might give you a {\bf bonus} for acting according to
your {\bf flaw}. representing the sacrifices you have made for
dependents or a temporary boost from satisfying your inner demons. For
example, if your superhero \quotation{Speedster} goes to see the
premiere of his partner's new play, instead of heading to the docks to
stop Dr.~Squid's shipment of Vibrium, your GM might award you a
{\bf bonus} to your relationship to your partner.

If your GM compels you to act according to your {\bf flaw}, you gain an
{\bf experience point} (see §8.1).

\section[title={Benefits and
Consequences},reference={benefits-and-consequences}]

{\bf Contests}, in addition to deciding whether you overcome a
{\bf story obstacle}, carry additional {\bf consequences}.

Your GM may simply determine these from what makes fictional sense,
given the agreed {\bf prize} for the {\bf contest}, as described above.
Optionally your GM may impose {\bf consequences} or provide
{\bf benefits} if they desire ongoing {\bf penalties} or {\bf bonuses}.
Your GM should always respond to the flow of the story, if narrative
consequences are enough, they should not reach for additional mechanical
{\bf bonuses} or {\bf penalties}. Your GM should use mechanical
{\bf bonuses} or {\bf penalties} where it strains credibility that there
is no ongoing consequence or benefit from the outcome of the contest.

In a fight, it may strain credibility that a defeat does not leave you
impaired for further physical activity. In a display or oratory before
the assembled townsfolk, it may strain credibility if they would not
later act according to your rousing words. In a romance, it may strain
credibility of the wonderful date night, does not improve your chances
of taking your relationship to the next level.

\subsection[title={Consequences},reference={consequences}]

After a {\bf contest}, you may suffer {\bf consequences}: literal or
metaphorical injuries.

\startitemize[packed]
\item
  In a fight or test of physical mettle, you wind up literally wounded.
\item
  In a social contest, you suffer damage to your reputation.
\item
  If commanding a war, you lose battalions, equipment, or territories.
\item
  In an economic struggle, you lose money, other resources, or
  opportunities.
\item
  In a morale crisis, you may suffer bouts of crippling self-doubt.
\stopitemize

The GM is the arbiter of when a {\bf consequence} should be applied.

Your GM may assign a penalty to reflect this {\bf consequence}. Your GM
should assign a {\bf penalty} that corresponds to a {\bf rank}: −3, −6,
−9, −M, or −M2. The {\bf rank} will depend on how severe they feel the
{\bf consequences} are.

If your opponent defeats you, your GM may use the difference between
your {\bf result} and the {\bf resistance's} result to determine the
{\bf rank}. If your opponent had a {\bf better success} it is one
{\bf rank} or −3. If you were one level of result different, such as you
{\bf failed} and they {\bf succeeded}, or you {\bf succeeded} and they
rolled a {\bf critical}, then it is two {\bf ranks} or −6, and so on,
with your rolling a {\bf fumble} and the GM rolling a {\bf critical} the
largest {\bf rank} at 4 for a −M penalty.

If you {\bf defeat} your opponent, your GM may still decide that you
suffer a {\bf consequence}, representing fatigue, exhaustion,
disapproval or other expenditure of resources on earning the
{\bf prize}.

\startitemize[packed]
\item
  In a fight, you are left bruised and battered.
\item
  In a social contest, you sacrifice the trust of a marginalized group.
\item
  If commanding a war, you must sacrifice some of your forces for
  victory.
\item
  In an economic struggle, you take significant losses to win market
  share.
\item
  In a morale crisis, your resolve alienates the cowardly.
\stopitemize

On a {\bf better roll} you might suffer a {\bf consequence} of
{\bf rank} 2 or −6, and on a better {\bf result} you might suffer a
{\bf consequence} of {\bf rank} 1 or −3.

\subsubsection[title={Ending a PC's
story},reference={ending-a-pcs-story}]

Your GM should not impose a narrative {\bf consequence} on your PC that
takes them permanently out of the game, such as by death, without
discussion. Some games allow characters to be taken out of the story by
the result of a dice roll, but QuestWorlds is a co-operative
storytelling game where a failed dice roll should not automatically
remove a character from play. However, you, or the GM, might feel that
your PC's story has come to an end with this failure, and you can
consent to that outcome. Usually, your GM should refrain from suggesting
this option unless the story itself suggests it.

A story-ending {\bf outcome} may not just be death. It can include
anything that takes the PC out of play, such as exile, dismissal from
the secret agency, a broken heart. In some cases the ending to your PCs
story could be ambiguous, allowing the PC to return at a future point
when the story makes their salvation possible.

Your GM might declare that the stakes of a particular {\bf contest}
place a PC at risk of this being a story ending moment, before the dice
are rolled. This may be important for credibility in the story that the
group is telling, In this case there should be an option for the PC to
avoid, or backdown from a {\bf contest}, that has a risk of ending their
story.

\subsection[title={Benefits},reference={benefits}]

Just as when you can experience ongoing ill effects from a
{\bf contest}, you can gain ongoing benefits from a {\bf contest}.

\startitemize[packed]
\item
  In a fight or test of physical mettle, your workout leaves you sharp
  for the next encounter.
\item
  In a social contest, you gain confidence and admiration from your
  triumph.
\item
  If commanding a war, you gain strategic advantage over your enemy.
\item
  In an economic struggle, your profits can be re-invested, or you drive
  competitors into the ground.
\item
  In a morale crisis, you are buoyed up by success, nothing can stop you
  now.
\stopitemize

Remember that the {\bf benefit} does not have to be directly related to
the {\bf ability} used. Look to the goal of the {\bf contest}. The
abilities or situation should reflect the {\bf story obstacle} that was
overcome and the {\bf tactic} used to overcome it.

\startitemize[packed]
\item
  In a fight or test of physical mettle, your triumph has everyone
  rallying to your cause.
\item
  In a social contest, you win powerful {\bf allies} who will strengthen
  you in your fight against your enemies.
\item
  If commanding a war, you pillage the enemy city and enrich your army.
\item
  In an economic struggle, you gain status as one of the wealthy elite.
\item
  In a morale crisis, your rallied troops strengthen your army.
\stopitemize

The GM is the arbiter of when a {\bf benefits} should be applied.

Your GM may assign a {\bf bonus} to reflect this {\bf benefit}. Your GM
should assign a {\bf bonus} that corresponds to a {\bf rank}: +3, +6,
+9, +M, or +M2. The {\bf rank} will depend on how great they feel the
{\bf benefits} are.

If you win the {\bf prize}, your GM may choose to use the difference
between your {\bf result} and their result to determine the {\bf rank}.
If you had a {\bf better success} than your opponent it is one
{\bf rank} or +3. If you were one level of result different, such as you
{\bf succeeded} and they {\bf failed}, or you rolled a {\bf critical}
and they {\bf succeeded}, then it is two {\bf ranks} or +6, and so on,
with your rolling a {\bf critical} and the GM rolling a {\bf fumble} the
largest {\bf rank} at 4 for a +M bonus.

If you lost the {\bf prize}, your GM may still decide that you gain a
{\bf benefit}, representing learning, gratitude, or resolve developed
from losing the {\bf prize}.

\startitemize[packed]
\item
  In a fight or test of physical mettle, you learn your opponent's
  weaknesses.
\item
  In a social contest, many feel sympathy for you though they cannot
  support you.
\item
  If commanding a war, you win the trust of your soldiers through shared
  suffering.
\item
  In an economic struggle, your organization becomes leaner and fitter.
\item
  In a morale crisis, you reflect on your failure and gain new inner
  strength.
\stopitemize

On a {\bf worse roll} you gain a {\bf benefit} of {\bf rank} 2 or +6,
and on a worse {\bf result} you might gain a {\bf benefit} of {\bf rank}
1 or +3.

\subsection[title={Recovery and
Healing},reference={recovery-and-healing}]

{\bf Consequences} lapse on their own with the passage of time. Your GM
will determine when the {\bf consequences} have faded, and you should
ask about whether they still apply at each new game session. The worse
the {\bf penalty}, the longer it may last, though the GM may reduce its
{\bf rank} with time, as you recover or heal. However, you'll often want
to remove them ahead of schedule, with the use of {\bf abilities}.

\subsubsection[title={Healing Abilities},reference={healing-abilities}]

The {\bf ability} used to bring about recovery from a {\bf consequence}
must relate to the type of harm.

You can heal physical injuries with medical or extraordinary
{\bf abilities}.

You can remove mental traumas, including those of confidence and morale,
with mundane psychology or through {\bf extraordinary abilities}. You
might also remove them through a dramatic confrontation between the
victim and the source of the psychic injury.

You use social abilities to heal social injuries. You probably have to
make a public apology of some sort, often including a negotiation with
the offended parties and the payment of compensation, either in
disposable wealth or something more symbolic.

You can fix damage to items and equipment with some sort of repair
{\bf ability}. If you want to fix an extraordinary item, you may require
genre-specific expertise: a broken magic ring may require a ritual to
reforge.

Your GM should almost always resolve healing attempts as {\bf simple
contests}. An exception might be a medical drama, in which surgeries
would comprise the suspenseful set-piece sequences of the game, and your
GM might chose a {\bf long contest}.

\subsubsection[title={Healing
Resistances},reference={healing-resistances}]

The {\bf resistances} to remove a states of adversity is the {\bf base
resistance} modified by the {\bf rank} of the {\bf consequence}. So if
you were suffering from a {\bf rank} 2 consequence, of −6, you modify
the {\bf base resistance} by a {\bf rank} 2 modifier of +6.

Your GM can more easily calculate this as the {\bf base resistance} with
a {\bf bonus} that is equal to and opposite your {\bf penalty}. So if
you have a {\bf penalty} of −3, it is {\bf base resistance} +3; if you
have a {\bf penalty} of −6, it is {\bf base resistance} +6 and so on.

When you make a successful healing attempt, you remove the
{\bf penalty}.

\subsection[title={Waning Benefits},reference={waning-benefits}]

Just as you recover from {\bf consequences} with time, or through
healing, so {\bf benefits} fade with time.

At the beginning of a session, especially when a significant period of
game-world time passes between the conclusion of one session and the
beginning of the next, the GM may declare that all {\bf benefits} have
expired or waned. A waning benefit may reduce its {\bf rank} with time,
as the effect fades. You are no longer charged with the confidence of
your recent victory, the fans have forgotten your last concert, or the
people of the village have started to think once again about the
day-to-day struggle of their lives not how the stranger helped them. An
expired benefit no longer gives you a {\bf bonus}, your past victories
no longer bring you solace, your fickle fans have moved on to the latest
sensation.

\subsection[title={Multiple Benefits And
Consequences},reference={multiple-benefits-and-consequences}]

A PC may apply {\bf bonuses} from multiple {\bf benefits} to a single
{\bf contest}, or apply {\bf penalties} from multiple {\bf consequences}
to a single {\bf contest}. {\bf Benefits and consequences} may cancel
each other out.

Because it is confusing to track both {\bf benefits and consequences}
against the same {\bf ability} your GM may simply rule that one cancels
the other out. This is particularly true of social {\bf contest}s where
a moment of shame can erase your previous triumphs, or your confidence
eroded by a {\bf failure}. Physical benefits may cancel out, flushed
with victory you may be able to ignore pain, but it may defy credibility
for wounds to be healed by an athletic performance.

Your GM may simply rule that {\bf benefits} and {\bf consequences}
cancel out, or they may take the difference between the two benefits and
create a new one. For example if you have a +6 bonus from impressing the
crowd with your previous performance in the dance {\bf contest}, but
then suffer an injured ankle with a {\bf penalty} of −3, your GM may
rule that your twisted ankle cancels out your energy from the last
performance, or your GM might rule that your success sees you through
the pain, but you are now only +3 to impress the crowd.

Your GM may prefer to cancel out in {\bf ranks} so that if you have a
rank 4 benefit from your popularity with the village following saving
their holy idol, giving you a +M bonus, but you make a minor social gaff
at the mayor's daughter's wedding of {\bf rank} 1, you drop one
{\bf rank} to 3, and a +9 bonus, over reducing +M to 17. This keeps the
numbers used for {\bf bonuses} and {\bf penalties} consistent, at the
cost of having to track or figure out the {\bf rank} of the bonus.

\section[title={Resistance
Progression},reference={resistance-progression}]

Your GM may decide that {\bf resistance} to your actions gets harder, as
the campaign progresses. This reflects the trope of the type of
challenges you face getting tougher as you improve.

Your GM should adopt a strategy that mimics a TV show where the
{\bf resistance} does not increase during a season of the show, allowing
our protagonists to get more competent as the show progresses towards
its climax. In the next season though the {\bf resistance} usually goes
up, and the writers reflect this with more challenging opposition in the
new season of the show. At the same time, the opposition that was tough
in the first season, now become mooks that can be easily dispatched to
show the increased competence of the protagonists.

In that case your GM should increment the {\bf base resistance} by +3,
+6 or +9 for the next campaign you play with the same characters. The
size of the change should reflect the increase in your previous
{\bf abilities} in the last campaign. For example, if in the last season
you increased your {\bf occupation keyword} by +6, your GM may decide to
increase the {\bf resistance} by +3 or +6 to reflect the more
challenging opposition in the new campaign. The GM should consider
triggering {\bf resistance progression} when your PCs find it difficult
to earn {\bf experience points} because they too regularly outclass even
the climatic encounters (the boss monsters) of their game.

Your GM should also take into account that the opposition you were
improving with respect to the previous season should now be considered
more-easily defeated mooks, and use lower {\bf scores} for them when
they appear in the story or even allow them to be taken out with an
{\bf assured contest}.

\subsection[title={No Progression},reference={no-progression}]

Your GM may also decide that the {\bf resistances} do not get harder as
the campaign progresses, reflecting the PCs {\bf ability} to disregard
minor challenges, and simply choose harder {\bf resistances} to
challenge the players and allow them to earn {\bf experience points}.

\section[title={Combined Abilities},reference={combined-abilities}]

On certain occasions your GM may rule that you can only hope to achieve
the {\bf prize} by using two disparate {\bf abilities}. When this
occurs, average your two {\bf ability scores}, then apply any modifiers,
to arrive at your {\bf TN}.

Combining your abilities, rather than using the best one and
{\bf augmenting} it with other, is always a disadvantage. Your GM should
only require combined {\bf ability} use when story logic absolutely
demands that you face a lower chance of {\bf success}, because you have
to do two things at once.

\section[title={Mismatched and Graduated
Goals},reference={mismatched-and-graduated-goals}]

Sometimes, the two sides in a {\bf contest} may have goals that do not
directly conflict one another. A huntsman pursues a nurse, who is trying
to escape through the forest with two small children. The huntsman wants
to kill the nurse. The nurse wants to save the children.

When encountering {\bf mismatched goals}, your GM should determine
whether the mismatch is complete, or partial.

In a {\bf complete mismatch}, neither side is at all interested in
preventing the other's goal. A {\bf complete mismatch} does not end in a
{\bf contest}; your GM asks what you are doing, and then describes each
participant succeeding at their goals.

In most instances, the {\bf contest} goals are not actually
{\bf mismatched}, but {\bf graduated}. You have both a {\bf primary} and
a {\bf secondary} goal. In this case, your GM frames the {\bf contest},
identifying which goal is which. To achieve both, you must get a higher
{\bf result} than your opponent, such as {\bf success} vs.~{\bf failure}
or a {\bf critical} vs {\bf success}. On a better roll alone, such as
{\bf success} vs.~{\bf success} or {\bf failure} vs {\bf failure}, your
GM may present you with the choice of which objective you obtain, where
that choice illuminates your PC's priorities.

\section[title={Mobs, Gangs, and
Hordes},reference={mobs-gangs-and-hordes}]

Sometimes you will face large numbers of opponents. Your GM can treat
many as one. Your GM divides the number of opponents by the number of
contesting PCs. Your GM then treats each of these sections of the crowd
as a single opponent with one {\bf score}. Their numbers are factored
into the {\bf score} your GM assigns to them.

If in doubt, your GM should think of the {\bf resistance} that would be
dramatically appropriate for a single opponent and then adjust it with a
{\bf bonus} of +3, +6 or +9 depending on how outnumbered you are. No
more than six foes can typically contend with you in a physical
confrontation, or two in a social one, or they tend to get in each
other's way.

When the mob loses an exchange, your GM describes individuals within it
as being hurt or falling away. When it wins, describe them overwhelming
you, or swelling in numbers.

\section[title={Ganging Up},reference={ganging-up}]

Sometimes you may outnumber your opponent. As above, if in doubt, the GM
should think of the {\bf resistance} that would be dramatically
appropriate for a one-on-one confrontation and then adjust it with a
{\bf penalty} of −3, −6, or −9 depending on how significantly you
outnumber them. As above, note that unless your opponent is
extraordinarily large, you cannot confront them physically with more
than about six people (include {\bf followers}) or socially with about
two people (again include {\bf followers}) or people just get in each
other's way.

\section[title={Mass Effort},reference={mass-effort}]

Clashes of massive forces resolve like any other {\bf contest},
{\bf simple} or {\bf long}. These include:

\startitemize[packed]
\item
  Military engagements
\item
  Corporate struggles for market share • Building competitions
\item
  Efforts to spread a faith or ideology • Dance competitions
\stopitemize

If you are not participating in the {\bf contest} and have no stake in
its {\bf outcome}, then your GM doesn't bother to run a {\bf contest}.
The GM just chooses an {\bf outcome} for dramatic purposes.

Otherwise, your GM will start by determining your degree of influence
over the {\bf outcome}. They are either:

\startitemize[packed]
\item
  Determining factors: The success of the effort depends mostly on your
  choices and successes. For example, you might be a military leader
  facing a force of roughly equal potency. As all else is equal, the
  better general will win the day. In this instance, your {\bf tactic}
  should be a relevant leadership {\bf ability}.
\item
  Contributors: One of the forces enjoys a clear advantage over the
  others, but your efforts may tip the balance in favor of a chosen
  side. Your GM will give you a {\bf TN} to roll against that represents
  the strength of your force, but you can {\bf augment} that {\bf TN}
  with an appropriate leadership {\bf ability}.
\item
  Acted Upon: You have little influence over the {\bf outcome}, but are
  stuck in the middle of the conflict and must struggle to prosper
  within it. The GM predetermines the {\bf outcome} of the overall
  competition on dramatic grounds. To determine your fate in the battle,
  you {\bf contest} against a {\bf resistance} determined by the GM,
  derived from the overall battle {\bf outcome}.
\stopitemize

\chapter[title={Character Creation},reference={character-creation}]

The first step in creating your character is to come up with a concept
that fits in with the genre of game that your GM intends to run. With
that, you can assign {\bf abilities}, {\bf scores} for those
{\bf abilities}, and if required {\bf flaws}.

In addition, you will want to give your character a name, and provide a
physical description. We recommend focusing on three physical things
about your PC that others would immediately notice, over anything more
detailed.

Your GM should not use this method for creating NPCs. NPCs do not
require definition via {\bf abilities} and {\bf keywords}. Instead, your
GM simply describes the NPC, and picks an appropriate {\bf resistance}
in any contest with them, based on their feeling for what would be
{\bf credible} for that NPC. If in doubt the GM just uses the {\bf base
resistance} for a mook, with a suitably higher {\bf rank} for a boss.
The design intent is to remove the need for the GM to prepare stat
blocks, making improvisation of NPCs easier, and shifting focus to the
NPCs personality or role in the story instead.

\section[title={As-You-Go Method},reference={as-you-go-method}]

\startitemize[n,packed][stopper=.]
\item
  Choose a {\bf concept}. Your {\bf concept} is a brief phrase, often
  just a couple of words that tells the GM and other players what you do
  and how you act. Start with a noun or phrase indicating your
  {\bf occupation keyword} or area of expertise, and modify it with an
  adjective suggesting a {\bf distinguishing characteristic}, a
  personality trait that defines you in broad strokes:
\stopitemize

\startitemize[packed]
\item
  haughty priestess
\item
  hotshot lawyer
\item
  noble samurai
\item
  remorseful assassin
\item
  sardonic ex-mercenary
\item
  slothful vampire
\item
  naive warrior
\stopitemize

\startitemize[n,packed][start=2,stopper=.]
\item
  Now provide your character with a name.
\item
  If the series uses other {\bf keywords}, such as those for culture or
  religion, you may gain one for free.
\item
  When events in the story put you in a situation where you want to
  overcome a {\bf story obstacle}, make up an applicable {\bf ability}
  on the spot. The first time you use an {\bf ability} (including the
  two you start play with: {\bf distinguishing characteristic} and
  {\bf occupational keyword}), assign a {\bf score} to it. This may be a
  {\bf breakout ability} from a {\bf keyword}. You are restricted to
  only one {\bf sidekick}.
\item
  If you want, describe {\bf flaws}.
\item
  Once you have 12 {\bf abilities} (including the two for character
  concept), and up to three {\bf flaws} you are done creating your
  character.
\stopitemize

\section[title={Assigning Ability
Scores},reference={assigning-ability-scores}]

You have now defined your {\bf abilities}. These tell everyone what you
can do. Now assign numbers to each {\bf ability}, called {\bf scores},
which determine how well you can do these things.

Assign a starting {\bf score} of 17 to the {\bf ability} you find most
important or defining. Although most players consider it wisest to
assign this {\bf scores} to their {\bf occupational keyword}, you don't
have to do this. Assign a {\bf score} of 17 to your {\bf distinguishing
characteristic}.

All other {\bf abilities} start at a {\bf score} of 13.

A {\bf breakout} from a {\bf keyword} starts at +1. In some cases, you
may treat your {\bf distinguishing characteristic} as a {\bf breakout
ability} from a {\bf keyword} in this case, treat it as a +4.

Now spend up to 20 points to increase any of your various {\bf scores},
including {\bf keywords}. Each point spent increases a {\bf score} by 1
point. You can't spend more than 10 points on any one {\bf ability}.

Some genre packs may require you to have additional {\bf keywords} that
reflect the setting. These additional {\bf keywords} come from the 12
{\bf abilites} allowance, so in many genres you will have fewer wildcard
{\bf abilities} but better fit the setting.

\section[title={Keywords},reference={keywords}]

You may build your PC around one or more {\bf keywords}. A {\bf keyword}
gives you a package deal: you get a number of {\bf abilities} by
selecting a pre-existing character concept, which the player then
modifies.

{\bf Keywords} are best suited for use as the PC's {\bf occupation}.

In certain genres, you may require multiple {\bf keywords}: for example,
one for {\bf occupation}, another for species or culture, and perhaps a
third for religious affiliation.

Here are two ways to handle {\bf keywords}. If in doubt, choose
Umbrella.

{\bf Keywords as Packages}: Treat {\bf keywords} simply as shorthand for
a package of {\bf abilities}. These can be increased together during
character creation, but are too unrelated to increase together during a
game. You are still free to use the {\bf keyword} as an {\bf ability},
and in fact may prefer to write only the specific {\bf abilities}
they've improved on their character sheet.

{\bf Keywords as an Umbrella}: Treat {\bf keywords} both as raisable
{\bf abilities} and as a collection of more specific {\bf abilities}.
This approach keeps the character sheet from getting too cluttered but
encourages specialization. If your character is particularly good at an
aspect of that keyword, you create a {\bf breakout ability} under the
{\bf keyword} at a {\bf bonus} from the {\bf score} of the {\bf keyword}
you write these specialized {\bf breakout abilities} under the
{\bf keyword}, along with how much they've improved from the
{\bf keyword}:

Detective 17 Forensics +2 Handgun +1

In this example, whilst the {\bf score} for most {\bf contests} in which
Detective was an appropriate {\bf tactic} would be 17, for contests
involving Forensics it would be 19, and for those involving firing a
handgun it would be 18.

In some settings, an {\bf ability} may be listed in more than one of a
PC's {\bf keywords}. Choose only one to detail it under.

\section[title={Flaws},reference={flaws-2}]

You may assign up to three {\bf flaws} to their PC. Common flaws
include:

\startitemize[packed]
\item
  Personality traits: surly, petty, compulsive.
\item
  Physical challenges: blindness, lameness, diabetes.
\item
  Social hurdles: outcast, ill-mannered, hated by United supporters.
\stopitemize

{\bf Flaws} are assigned a {\bf score} equivalent to your
{\bf abilities}. The first {\bf flaw} is rated at the highest
{\bf ability}, the second shares the same {\bf score} as the
second-highest {\bf ability}, and the third equals the lowest
{\bf ability}.

Certain {\bf keywords} include {\bf flaws}. {\bf Flaws} gained through
{\bf keywords} do not count against the limit of three chosen
{\bf flaws}. All {\bf flaws} after the third are given the same
{\bf score} as the third {\bf ability}. You may designate {\bf flaws}
from {\bf keywords} as your first or second-ranked {\bf flaw}.

\section[title={Advanced Character
Creation},reference={advanced-character-creation}]

{\em QuestWorlds} offers two advanced methods of character creation:
prose and list.

\subsection[title={The List Method},reference={the-list-method}]

This is like the As-You-Go method (see §3.1) but you spend all their
points before the game begins. This is possible with the As-You-Go
method as well, but the list method allows you to signal what they want
the game to be about from the abilities you pick, as opposed to reacting
to material once the game begins.

\subsection[title={The Prose Method},reference={the-prose-method}]

This is the most different method as you write a piece of prose and then
pull {\bf abilities} from that. Its intent is to emulate a character
description in fiction, and indeed PCs can be built by copying text from
a story and then identifying {\bf keywords}. It is the least
\quote{fair} of the character creation options.

\section[title={List Method},reference={list-method}]

\startitemize[n,packed][stopper=.]
\item
  Choose a {\bf concept}. Your {\bf concept} is a brief phrase, often
  just a couple of words that tells the GM and other players what you do
  and how you act. Start with a noun or phrase indicating your
  {\bf occupation keyword} or area of expertise, and modify it with an
  adjective suggesting a {\bf distinguishing characteristic}, a
  personality trait that defines you in broad strokes:
\stopitemize

\startitemize[packed]
\item
  haughty priestess
\item
  hotshot lawyer
\item
  noble samurai
\item
  remorseful assassin
\item
  sardonic ex-mercenary
\item
  slothful vampire
\item
  naive warrior
\stopitemize

\startitemize[n,packed][start=2,stopper=.]
\item
  Now provide the character with a name.
\item
  Note their {\bf occupation}, which is usually a {\bf keyword}. You
  probably already picked this when you came up with your character
  concept.
\item
  If the series uses other {\bf keywords}, such as those for culture or
  religion, you may have one of them for free.
\item
  Pick 10 additional {\bf abilities}, describing them however the player
  wants. Only one of these {\bf abilities} may be a {\bf sidekick}.
\item
  If you want, describe up to 3 {\bf flaws}.
\stopitemize

\section[title={Prose Method},reference={prose-method}]

You write a paragraph of text like you would see in a story outline,
describing the most essential elements of your character. Include
{\bf keywords}, personality traits, important possessions,
relationships, and anything else that suggests what you can do and why.
The paragraph should be about 100 words long.

Compose the description in complete, grammatical sentences. No lists of
{\bf abilities}; no sentence fragments. Your GM may choose to allow
sentences like the previous one for emphasis or rhythmic effect, but not
simply to squeeze in more cool things you can do.

Once your narrative is finished, convert the description into a set of
{\bf abilities}. Mark any {\bf keywords} with double underlines. Mark
any other word or phrase that could be an {\bf ability} with a single
underline. Then write these {\bf keywords} and {\bf abilities} on your
character sheet.

There is no limit to the number of {\bf abilities} you can gain from a
single sentence, as long as the sentence is not just a list of
{\bf abilities}. If your GM decides a sentence is just a list, they may
allow the first two {\bf abilities}, or they may tell the player to
rewrite the sentence. Note, however, that you cannot specify more than
one {\bf sidekick} in your prose description.

\chapter[title={Simple Contests},reference={simple-contests}]

{\bf Simple contest}s are the default resolution method for all
{\bf story obstacles}.

\section[title={Simple Contest},reference={simple-contest-1}]

\subsection[title={Procedure},reference={procedure}]

\startitemize[n,packed][stopper=.]
\item
  Your GM {\bf frames the contest}.
\item
  You choose a {\bf tactic}, and figure your PC's {\bf target number}
  ({\bf TN}) using the {\bf score} and any {\bf modifiers}. The PCs
  {\bf TN} is the {\bf score} of their {\bf ability}, plus or minus
  {\bf modifiers} the GM may give you.
\item
  Your GM determines the {\bf resistance}. If two PCs contend, your
  opponent figures their {\bf TN} as described in step 2.
\item
  Roll a D20 to determine your {\bf success or failure}, then apply any
  {\bf bumps}. Your GM does the same for the {\bf resistance}. Compare
  your rolled number with your {\bf TN} to see how well you succeeded or
  failed with your {\bf ability}. Remember to apply any {\bf bumps} from
  {\bf masteries} or {\bf story points}.
\item
  Determine {\bf victory} or {\bf defeat}. Award {\bf experience points}
  if appropriate (see §8.1).
\item
  Describe the {\bf outcome} based on the {\bf story obstacle}.
\stopitemize

\section[title={Group Simple Contest},reference={group-simple-contest}]

In the {\bf group simple contest}, multiple participants take part in a
{\bf simple contest}. Each of you in your group conducts an individual
{\bf simple contest} against the GM, and the {\bf outcomes} for each
side are collated to determine the victor.

A {\bf group simple contest} may pit all of you against a single
{\bf resistance}, representing one {\bf story obstacle}. Alternatively,
a {\bf group simple contest} may be a series of paired match-ups between
two groups of contestants. If you are forced to participate in more than
one {\bf contest}, then you face the standard multiple opponent
{\bf penalties}.

\subsection[title={Procedure},reference={procedure-1}]

\startitemize[n,packed][stopper=.]
\item
  Your GM {\bf frames the contest}.
\item
  You choose a {\bf tactic}, and figure your PC's {\bf target number}
  ({\bf TN}) using the {\bf score} and any {\bf modifiers}. Your
  {\bf TN} is the {\bf score} of their {\bf ability}, plus or minus
  {\bf modifiers} the GM may give you.
\item
  Your GM determines the {\bf resistance}. If two PCs contend, your
  opponent figures their {\bf TN} as described in step 2.
\item
  For each of your group, roll a D20 to determine your {\bf success or
  failure}, then apply any {\bf bumps}. Your GM does the same for the
  {\bf resistance}. Compare your rolled number with your {\bf TN} to see
  how well you succeeded or failed with your {\bf ability}. Remember to
  apply any {\bf bumps} from {\bf masteries} or {\bf story points}.
\item
  The side with the highest number of {\bf victories} is the overall
  victor in the {\bf contest}. Award {\bf experience points} if
  appropriate (see §8.1).
\item
  Describe the {\bf outcome} based on the agreed {\bf prize}.
\stopitemize

It is possible that you suffer a {\bf defeat}, even though your side
gains the {\bf victory}. It is possible that, as a result, that your GM
will suffer a {\bf consequence of defeat} (see §2.7) related to your
{\bf defeat}, even though your side won. If your side loses, then you
may suffer both a {\bf consequence of defeat} for your own individual
{\bf contest}, and a {\bf consequence of defeat} for the overall
{\bf contest}. That may simply be a worsening of the {\bf consequence of
defeat}.

\chapter[title={Long Contests},reference={long-contests}]

Most conflicts should be resolved simply and quickly, using the
{\bf simple contest} rules.

However, every so often, your GM wants to draw out the resolution,
breaking it down into a series of smaller actions, increasing the
suspense you feel as you wait to see if they {\bf succeed} or
{\bf fail}.

Think of the different ways a film director can choose to portray a
given moment, depending on how important it is to the story, and how
invested they want us to feel in its {\bf outcome}.

For example, there are two ways to shoot a scene in which a thief breaks
into the bank to steal the contents of the safe.

The action can be portrayed quickly, cutting to a moment with the thief,
their ear pressed against the safe trying to get the tumblers to fall
into place. Then they sigh with relief, open the safe, and get whatever
is inside. In this instance, the story is about what happens after the
thief gets what's in the safe, not about what might happen to them if
they fail.

Another film might instead choose to make the bank robbery a pivotal
turning point in the story, if not its climactic moment. It would spend
many scenes building up to the safe-cracking sequence: obtaining the
plans of the bank, learning the movements of the guards, crawling
through the air conditioning ducts, sliding past the motion sensors and
pressure plates, and finally cracking the safe itself. All of these
scenes would be {\bf rounds} of a {\bf long contest}.

Remember that {\em QuestWorlds} uses conflict resolution. If you want to
describe how you overcome a sequence of {\bf story obstacles} to
overcome the {\bf resistance} then your GM should use a {\bf long
contest}, if you just want to move on to the next scene, use a
{\bf simple contest}.

Even a movie driven by action and suspense will typically include only a
handful of these set-piece sequences. They need the rest of their
running time to build up to their big moments, to make us care about the
characters, and to give us quiet moments to contrast with the
white-knuckle parts.

So pacing may always trump your desire to work through the sequence of
events, as your GM may wish to resolve this conflict quickly. This is
especially true if only one player is involved.

Your GM may be tempted, to adjudicate every fight with a {\bf long
contest}, because fights seem like they should be played out
blow-by-blow. They should resist this temptation, as fights are often
repetitive trading of blows that can drag when everyone repeats actions
from {\bf round} to {\bf round}. Only use {\bf long contests} for fights
where the PCs want to do more than slug it out toe-to-toe with their
opponents until only one is left standing.

There are three types of {\bf long contest}. Your GM should choose
{\bf ONE} to use with their campaign: {\bf scored contest},
{\bf extended contest}, or {\bf chained contest}.

\section[title={No Nesting},reference={no-nesting}]

Your GM should never \quotation{nest} one {\bf long contest} inside
another. If a {\bf long contest} is in progress and you want to perform
an action your GM should treat it as an {\bf unrelated action}, or
disallow it completely during the current {\bf contest}.

\section[title={Scored Contest},reference={scored-contest}]

{\bf Scored contests} are longer and more dramatic than {\bf simple
contests}. Your GM uses {\bf scored contests} when the {\bf outcome} of
the struggle is important, to generate suspense for you, or when your GM
want a back-and-forth struggle. It is something you and your GM should
visualize and describe.

A {\bf scored contest} consists of one or more {\bf rounds}, in which
you perform actions that are similar to {\bf simple contests}. However,
actions and {\bf rounds} do not decide the {\bf outcome} of the whole
{\bf contest}, only who gains or loses {\bf resolution points} at that
time. In a {\bf scored contest} there is no distinction between
aggressor and defender, each {\bf round} represents attempts by both
parties to overcome their opponent. Your GM should determine who has the
initiative to describe what they are doing for any {\bf exchange}, based
on their interpretation of the flow of events. If in doubt your GM
should defer to you over your opponent to describe what you do in the
{\bf round}, and describe the NPC reacting to that.

\subsection[title={Procedure},reference={procedure-2}]

\startitemize[n,packed][stopper=.]
\item
  Your GM {\bf frames the contest}.
\item
  You choose a {\bf tactic}, and figure your PC's {\bf target number}
  ({\bf TN}) using the {\bf score} of your {\bf ability}, plus or minus
  {\bf modifiers} the GM may give you.
\item
  Your GM determines the {\bf resistance}. If two PCs contend, your
  opponent figures their {\bf TN} as described in step 2.
\item
  Carry out one or more {\bf rounds}, repeating as necessary.
  \startitemize[n,packed][stopper=.]
  \item
    A {\bf scored contest} unfolds as a series of {\bf simple contests}.
    At the end of each {\bf simple contest}, the winner scores a number
    of {\bf resolution points (RPs)} to their tally, which varies
    between 1 and 5, depending on the {\bf result}. Tied {\bf results}
    leave the score unchanged.
  \item
    Your GM decides which opponent has the initiative and describes what
    they are trying to do to achieve the {\bf prize}, the
    \quote{aggressor}. The \quote{defender} describes how they counter
    the aggressor's attempt to seize the {\bf prize}. If it is not
    obvious from the unfolding narrative, your GM should choose your PC
    as the \quote{aggressor}.
  \item
    Conduct a {\bf simple contest} as normal, but once the {\bf outcome}
    has been determined, it becomes a number of {\bf resolution points}
    scored by the winning side.
  \item
    The number of {\bf resolution points} the winner garners at the end
    of each {\bf round} depends on the difference in their
    {\bf results}, (see below).
  \item
    The first to accumulate a total of 5 {\bf resolution points} wins;
    their opponent is knocked out of the {\bf contest} and loses the
    {\bf prize}.
  \stopitemize
\item
  Determine the {\bf scored contest outcome} based on {\bf rising
  action} or {\bf climax} (below). Award {\bf experience points} if
  appropriate (see §8.1).
\item
  Determine {\bf benefits} or {\bf consequences}.
\item
  Describe the {\bf outcome} based on the {\bf story obstacle}.
\stopitemize

Unlike in an {\bf extended contest} (see below), where you usually take
part in two {\bf exchanges} with your opponent per {\bf round} (one in
which you choose the {\bf AP bid}, and one in which your opponent does),
here you and your opponent engage in a single {\bf exchange} per
{\bf round} (in which whoever the GM determines has initiative describes
an action to seize the {\bf prize} and their opponent how they intend to
stop them).

\subsubsection[title={Resolution Points},reference={resolution-points}]

\startitemize[packed]
\item
  If you have a better roll on the same result you score 1
  {\bf resolution point}.
\item
  You score two {\bf resolution points} for one level of difference,
  such as {\bf success} vs.~{\bf failure}, or {\bf critical}
  vs.~{\bf success}.
\item
  You score three {\bf resolution points} for two level of difference
  such as {\bf success} vs {\bf fumble}, or {\bf critical}
  vs.~{\bf failure}
\item
  You score four {\bf resolution points} for three levels of difference,
  which is a {\bf critical} vs.~{\bf fumble}.
\stopitemize

You can summarize this as: one more {\bf resolution point} than the
levels of difference between the {\bf results}.

Your {\bf resolution point} score tells you how well you're doing,
relative to your opponent, in the ebb and flow of a fluid, suspenseful
conflict. If you're leading your opponent by 0--4, you're giving them a
thorough pasting. If you're behind 4--0, you're on your last legs, while
your opponent has had an easy time of it. If you're tied, you've each
been getting in some good licks.

In a fight, scoring 1 {\bf RP} might mean that you hit your opponent
with a grazing blow, or knocked him into an awkward position.

Scoring 2 {\bf RPs} might mean a palpable hit, most likely with
bone-crunching sound effects.

A 3 {\bf RP} hit sends them reeling, and, depending on the realism level
of the genre, may be accompanied by a spray of blood.

However, the exact physical harm you've dished out to them remains
unclear until the {\bf contest's} end. When that happens, the real
effects of your various {\bf victories} become suddenly apparent.
Perhaps they stagger, merely dazed, up against a wall. Maybe they fall
over dead.

In a debate, a 1 {\bf RP} might occasion mild head nodding from
spectators, or a frown on your opponent's face.

2 {\bf RPs} would occasion mild applause from onlookers, or send a flush
to your opponent's face.

On 3 {\bf RPs}, your opponent might be thrown completely off-track, as
audience members wince at the force of your devastating verbal jab.

In interpreting the individual {\bf simple contests} within a
{\bf scored contest}, your GM is guided by two principles:

\startitemize[n,packed][stopper=.]
\item
  No consequence is certain until the entire {\bf scored contest} is
  over.
\item
  When a character scores points, it can reflect any positive change in
  fortunes, not just the most obvious one.
\stopitemize

\subsection[title={Scored Contest
Outcomes},reference={scored-contest-outcomes}]

In a {\bf scored contest} the contestant that is the first to gain a
total of 5 {\bf resolution points} gains the {\bf prize}.

Your GM may treat the difference in {\bf resource points} as a measure
of the magnitude of your {\bf victory} or {\bf defeat}. A 5--0
{\bf outcome} is far more decisive than a 5--4 {\bf outcome} for
example.

Your GM may apply {\bf consequences} and {\bf benefits} as they see fit.
The {\bf scale} of those {\bf consequences} and {\bf benefits} may be
guided by the difference in {\bf resolution points} between the two
sides.

\subsection[title={Parting Shot},reference={parting-shot}]

In the {\bf round} immediately after you take an opponent out of the
{\bf contest}, you may attempt to gain another {\bf prize} from your
opponent suffers by engaging in a {\bf parting shot}. This is an attempt
(metaphoric or otherwise) to kick your opponent while he's down:

\startitemize[packed]
\item
  Striking an incapacitated enemy
\item
  Attacking a retreating army
\item
  Attacking one more punitive rider to a legal settlement
\item
  Demanding additional money from a business partner
\item
  Delivering one last humiliating insult
\stopitemize

You should agree an additional {\bf prize} that you desire beyond the
stakes agreed at the beginning of the contest. If you succeed in a
{\bf parting shot} you will also gain that prize. Your GM should agree
that the additional {\bf prize} makes sense as an opportunity brought
about by your opponent's {\bf defeat}.

Your GM should not use a {\bf parting shot}.

The {\bf parting shot} is another simple contest against your
{\bf defeated} opponent. The {\bf ability} you use must relate to the
consequences the opposition will suffer, but needn't be the same one you
used to win the {\bf contest}. If the loser is a PC they use a suitable
{\bf ability} to resist; otherwise the GM rolls a suitable
{\bf resistance} value.

If you succeed in your {\bf parting shot} roll, you gain the additional
{\bf prize}.

However, if your opponent succeeds, they take the number of
{\bf resolution points} they would, in a standard {\bf round}, score
against you, and instead subtracts them from the number of
{\bf resolution points} scored against them in the {\bf round} that
removed them from the {\bf contest}. If the revised total is now less
than 5 {\bf RPs}, they return to the {\bf contest}, and may re-engage
you. Your GM describes this as a dramatic turnaround, in which your
overreaching has somehow granted them an advantage allowing them to
recover from their previous misfortune. The provisional consequences
they suffered now go away, and are treated as a momentary or seeming
disadvantage.

Where it makes sense, unengaged PCs may attempt {\bf parting shots}
against opponents taken out of the {\bf contest} by someone else. You
may not revive your teammates by using your lamest abilities to make
{\bf parting shots} on them; this, by definition, does not pass a
{\bf credibility test}.

\subsection[title={Asymmetrical Round},reference={asymmetrical-round}]

You may choose to briefly suspend your attempt to best your opponent in
a {\bf scored contest}, in order to do something else. An instance where
you are trying to do something else and your opponent is trying to win
the {\bf contest} is called an {\bf asymmetrical round}.

In an {\bf asymmetrical round}, you do not score {\bf RPs} against your
opponent if you win the {\bf round}. Instead, you succeed at whatever
else you were doing. You still lose {\bf RPs} if you fail. Often you
will be using an {\bf ability} other than the one you've been waging the
{\bf contest} with, one better suited to the task at hand. This becomes
additionally dangerous when the {\bf score} associated with your
substitute {\bf ability} is significantly lower than the one used for
the rest of the {\bf contest}.

In addition to secondary objectives, as in the above example, you may
engage in {\bf asymmetrical round} to grant {\bf augments} (see above)
to yourself or others.

\subsection[title={Disengaging},reference={disengaging}]

You can always abandon a {\bf contest}, but, in addition to failing at
the {\bf story obstacle}, you may also suffer negative consequences. In
a {\bf contest} where your opponent intends to harm you, you will always
suffer negative consequences if you withdraw, unless you successfully
disengage.

To disengage, you make an {\bf asymmetrical round}, using the
{\bf ability} relevant to the {\bf contest} you're trying to wriggle out
of.

If you fail, your effort is wasted and the score against you increases,
as it would have during a normal {\bf round}. If you succeed, you escape
the clutches, literal or metaphorical, of your opponent, without further
harm from a {\bf contest} during the {\bf rising action}. In a
{\bf climactic} scene, however, {\bf RPs} scored during {\bf contests}
you disengaged from are still taken into account when determining
{\bf consequences}. In the case of a {\bf group contest},
{\bf consequences} against you are determined as soon as you disengage.

\section[title={Group Scored Contest},reference={group-scored-contest}]

{\bf Group scored contests} proceed as a series of {\bf scored contests}
between pairs of PC and opponents, interwoven so that they happen nearly
simultaneously.

As in a {\bf scored contest} between a single PC and an opponent, only
one {\bf simple contest} per pair of adversaries occurs each
{\bf round}. Usually the PCs make up one team, and their antagonists the
other.

A {\bf group scored contest} continues until one side has no active
participants. If you {\bf defeat} your opponent you can pair with a new
opponent. The new opponent might be unengaged, but might also be engaged
in an existing pairing. When you pair with a new opponent, you begin a
new {\bf contest}, even if your opponent is already engaged in a
{\bf contest}. Alternatively, if you are unopposed, you may choose to
{\bf assist}. Of course, you may be later engaged by an opponent who
becomes free yourself.

You may lose some pairings amongst the PCs, but still win if the last
participant standing is a PC; otherwise if the last participant belongs
to the opposition you lose.

\subsection[title={Procedure},reference={procedure-3}]

\startitemize[n,packed][stopper=.]
\item
  Your GM {\bf frames the contest}.
\item
  You choose a {\bf tactic}, and figure your PC's {\bf target number}
  ({\bf TN}) using the {\bf score} of your {\bf ability}, plus or minus
  {\bf modifiers} the GM may give you.
\item
  The GM determines the {\bf resistance}. If two PCs contend, your
  opponent figures their {\bf TN} as described in step 2.
\item
  The PCs to choose their opponents in order of their {\bf TN} where it
  makes sense. Otherwise your GM will allocate opponents to you
  dependent on what makes narrative sense.
\item
  Establish an order of the paired {\bf contests}. There is no
  significant advantage to going first, but use your group's {\bf TN}s
  from highest to lowest if no other option presents itself.
\item
  For each pairing your GM carries out one {\bf round}. Then they repeat
  by carrying out more {\bf rounds} in order, as necessary. The
  {\bf group scored contest} ends as soon as there are no active
  participants on one side of the conflict. The side with one or more
  participants left standing wins.
  \startitemize[n,packed][stopper=.]
  \item
    A {\bf group scored contest} unfolds as a series of {\bf simple
    contests}. At the end of each {\bf simple contest}, the winner
    scores a number of {\bf resolution points (RPs)} to their tally,
    which varies between 1 and 5, depending on the result. Tied results
    leave the score unchanged.
  \item
    Your GM decides which opponent in a pair has the initiative and
    describes what they are trying to do to achieve the {\bf prize}, the
    \quote{aggressor}. The \quote{defender} describes how they counter
    the aggressor's attempt to seize the {\bf prize}. If it is not
    obvious from the unfolding narrative, your GM should choose your PC
    as a the \quote{aggressor}.
  \item
    Conduct a {\bf simple contest} as normal, but once the {\bf outcome}
    has been determined, it becomes a number of {\bf resolution points}
    scored by the winning side.
  \item
    The number of {\bf resolution points} the winner garners at the end
    of each {\bf round} depends on the difference between their
    {\bf results}, (see §5.1.2).
  \item
    The first to accumulate a total of 5 points wins; their opponent is
    knocked out of the {\bf contest}.
    \startitemize[n,packed][stopper=.]
    \item
      As one of a pair is eliminated from the {\bf group scored
      contest}, their victorious opponents may then move on to engage
      new targets, starting new {\bf contests}, which are then added to
      the end of the existing sequence.
    \item
      If participating in multiple pairings, each pairing is the first
      to 5 points, points already scored do not count.
    \stopitemize
  \stopitemize
\item
  Award {\bf experience points} if appropriate (see §8.1).
\item
  Describe the {\bf outcome} based on the {\bf story obstacle}.
\stopitemize

\subsection[title={Group Scored Contest
Outcomes},reference={group-scored-contest-outcomes}]

In a {\bf group scored contest} the team that has the last undefeated
contestant gains the {\bf prize}.

Your GM may decide that you suffer individual {\bf consequences} or gain
individual {\bf benefits} from the {\bf outcome} of the contests you
pursued, regardless of whether your team won or lost. You might be on
the winning team, but lost your individual {\bf contest} and suffer a
{\bf penalty} to ongoing actions as a result. Alternatively, you might
be on the losing team, but win your individual {\bf contest} and gain a
{\bf benefit} as a result. If you lost, your {\bf benefit} should not be
the {\bf prize} but instead reflect a side-affect of your individual
triumph. Similarly, if your team won, your {\bf consequence} should not
limit the {\bf prize} which your team one, but should reflect a
side-affect of your individual loss.

See §5.1.3 for a discussion of {\bf consequences} and {\bf benefits} in
{\bf scored contests}.

\subsection[title={Unrelated Actions},reference={unrelated-actions}]

If you are not currently enmeshed in a {\bf round}, either after a
successful disengagement, or after winning a {\bf round}, you may take
actions within the scene that do not directly contribute to the
{\bf defeat} of the other side. These {\bf unrelated actions} may grant
an {\bf augment} to yourself or to a teammate. You may achieve a
secondary story objective. This resembles an {\bf asymmetrical round},
except that, as you are not targeted by any opponents, there is no
additional risk.

\subsection[title={Assists},reference={assists}]

You may take an {\bf unrelated action} to grant an {\bf assist} to a
teammate enmeshed in a {\bf round}. {\bf Assists} are subject to the
same restrictions as {\bf augments}: they must be both credible and
interesting.

Your first {\bf assist} faces a {\bf moderate resistance}. Each
subsequent {\bf assist} attempt to the same beneficiary, steps up by one
step on the {\bf scale}:+3, +6, +9, +M, +M2. The {\bf resistance}
escalation occurs even when another PC steps in to make a subsequent
{\bf assist}. This escalation allows the occasional dramatic rescue but
makes it difficult for players to prolong losing battles to excruciating
length. Your GM should make it seem credible by justifying the
increasing {\bf resistances} with descriptions of ever-escalating
countermeasures on the part of the opposition.

Your GM may adjust the starting {\bf resistance} up or down by one step
to account for campaign credibility or other dramatic factors. If an
{\bf assist} as proposed seems too improbable or insufficiently useful,
your GM should collaborate with you to propose alternate suggestions
which would face {\bf moderate resistance}.

The {\bf assist} alters the score against your teammate according to the
{\bf outcome} of a {\bf simple contest}.

If you have the same result, but a better level of {\bf success}, you
reduce the {\bf resolution points} by 1; if you have one level of
difference, such as {\bf critical} vs a {\bf success} or a {\bf success}
vs a {\bf failure} you reduce then {\bf resolution points} by 2; if you
have two levels of difference, such as a {\bf critical} vs a
{\bf failure} or a {\bf success} vs a {\bf fumble} you reduce the reduce
the {\bf resolution points} by 3; if you have three levels of difference
from a {\bf critical} vs.~a {\bf fumble} you reduce the {\bf resolution
points} by 4.

If you lost the contest, you will worsen your allies position.

If you have the same result, but a lower level of {\bf success} you
increase the {\bf resolution points} by 1; if you have one level of
difference, such as a {\bf success} vs a {\bf critical} or a
{\bf failure} vs a {\bf success} you increase the {\bf resolution
points} by 2; if you have two levels of difference, such as a
{\bf failure} vs a {\bf critical} or a {\bf fumble} vs.~a {\bf success}
you increase the {\bf resolution points} by 3. If you {\bf fumble}
against a {\bf critical} you increase the {\bf resolution points} by 4.

If you fail with a lesser success, your GM may interpret the actions as
a distraction that allows your opponent advantage. If you fail, your GM
may interpret your actions as backfiring and making inflicting harm.

Scores can never be reduced below 0.

\subsection[title={Followers},reference={followers}]

You may choose to have your {\bf followers} take part in {\bf group
scored contests} in one of three ways: as full contestants, as secondary
contestants, or as supporters.

{\bf Contestant}: The {\bf follower} takes part in the {\bf contest} as
any other PC would. You roll for your {\bf followers} as you would their
main characters. However, your {\bf followers} are removed from the
{\bf contest} whenever 3 {\bf resolution points} are scored against them
in a given {\bf round}.

{\bf Secondary contestant}: To act as a secondary contestant, your
{\bf follower} must have an {\bf ability} relevant to the {\bf contest}.
The {\bf follower} sticks by your side, contributing directly to the
effort: fighting in a battle, tossing in arguments in a legal dispute,
acting as the ship's navigator, or whatever. Although you describe this,
you do not roll for the {\bf follower}. Instead, you may, at any point,
shift any number of {\bf resolution points} to a {\bf follower} acting
as a secondary contestant. Followers with 3 or more {\bf resource
points} lodged against them are removed from the scene.

{\bf Supporter}: Your {\bf follower} is present in the scene, but does
not directly engage your opponents. Instead they may perform
{\bf assists} and other {\bf unrelated actions}.

{\bf Followers} acting in any of these three capacities may be removed
from the {\bf contest} by otherwise unengaged opponents. To remove a
{\bf follower} from a scene, an opponent engages your {\bf follower} in
a {\bf simple contest}. Your GM sets the {\bf resistance}, or if it is
another PC's {\bf follower} they determine the relevant {\bf ability} of
the {\bf follower} engaging yours. On any failure, your {\bf follower}
is taken out of the {\bf contest}.

Your GM determines any long-term implications for the follower being
removed from the contest. Whilst your GM should not end your character's
story without consent, such as via death, they may choose to end the
story of a follower in such circumstances, viscerally demonstrating the
threat that the PCs face.

\subsection[title={Risky Gambits},reference={risky-gambits}]

During a {\bf scored contest}, you can attempt to force a conflict to an
early resolution by making a {\bf risky gambit}. If you win the
{\bf round}, you lodge an additional 1 {\bf resolution point} against
your opponent. However, if you lose the {\bf round}, your opponent
lodges an additional 2 {\bf resolution points} against you.

If both contestants engage in a {\bf risky gambit}, the winner lodges an
additional 2 {\bf resolution points} against the loser.

\subsection[title={Defensive Responses},reference={defensive-responses}]

In a {\bf scored contest}, you can make a {\bf defensive response},
lowering the number of {\bf resolution points} lodged against you in a
{\bf round}. If you win the {\bf round}, the number of {\bf resolution
points} you lodge against your opponent decreases by 1. If you lose,
your opponent lodges 2 fewer {\bf resolution points} against you. The
total number of {\bf resolution points} assigned by a {\bf round} is
never less than 0; there is no such thing as a negative {\bf resolution
point}.

\subsection[title={Joining Scored Contests in
Progress},reference={joining-scored-contests-in-progress}]

When you wish to join a {\bf scored contest} in progress, you and your
GM should discuss whether you accept the current framing. If so, you can
participate. In a {\bf scored contest}, you simply select an opponent
and enter into a new {\bf round}. If you want to achieve something other
than the goal established during framing, you may instead perform
{\bf unrelated actions}, including {\bf assists} and {\bf augments}.

\subsection[title={Switching Abilities},reference={switching-abilities}]

You may describe an action in a {\bf scored contest} that is not covered
by the {\bf ability} that you started the contest with. There are two
possibilities here: either you are trying to provide color to your
actions in the {\bf round}, without seeking to gain advantage, or you
are seeking to gain advantage over your opponent with a novel
{\bf tactic}. In the former case, you can continue to use the
{\bf ability} you started the contest with, as you should not be
penalized for wanting to enhance the contest with colorful or
entertaining descriptions. In the latter case you should switch
{\bf abilities}, and your GM must decide if the {\bf resistance} changes
because of your new {\bf ability}. Your GM is encouraged to reward
{\bf tactics} that exploit weaknesses that have been identified in the
story so far with a lower {\bf resistance}. Sometimes your GM may
respond with a higher {\bf resistance} because your {\bf tactic} looks
less likely to succeed due to conditions already established in the
story.

\section[title={Extended Contest},reference={extended-contest}]

{\bf Extended contests} are longer and more dramatic than {\bf simple
contests}. Your GM uses {\bf extended contests} when the {\bf outcome}
of the struggle is important, to generate suspense for the players, or
when they want a back-and-forth struggle. It is something you and your
GM should visualize and describe.

An {\bf extended contest} consists of one or more {\bf rounds}, in which
you perform actions that are similar to {\bf simple contests}. However,
actions and {\bf rounds} do not decide the {\bf outcome} of the whole
{\bf contest}, only who gains or loses {\bf advantage points (AP)} at
that time. You take actions in turn, an {\bf exchange}, losing and
gaining the advantage, until either you or your opponent runs out of
{\bf advantage points} and is {\bf defeated}.

\subsection[title={Procedure},reference={procedure-4}]

\startitemize[n,packed][stopper=.]
\item
  Your GM {\bf frames the contest}.
\item
  You choose a {\bf tactic}, and figure your PC's {\bf target number}
  ({\bf TN}) using the {\bf score} of your {\bf ability}, plus or minus
  {\bf modifiers} the GM may give you. Figure your starting
  {\bf advantage point (AP)} total (see §5.3.2.1).
\item
  The GM determines the {\bf resistance}. The GM opposes the PC with a
  {\bf resistance}---the harder the task or tougher the opponent, the
  higher the {\bf resistance}. The GM figures starting {\bf APs} for the
  {\bf resistance} (see §5.3.2.1).
\item
  Carry out one or more {\bf rounds}, repeating as necessary.
  \startitemize[n,packed][stopper=.]
  \item
    Each {\bf round} consists of two {\bf exchanges}: an action and
    immediate response.
    \startitemize[n,packed][stopper=.]
    \item
      You describe your action towards the desired {\bf prize} and bid
      {\bf APs} *(see §5.3.2.2).
    \item
      Roll a die to determine your {\bf result}, then apply any
      {\bf bumps}. Your GM does the same.
    \item
      Compare the results of the two {\bf results} to determine who
      loses {\bf AP}; only when you have a {\bf critical} can you gain
      {\bf AP} from your opponent. (see §5.3.2.3)
    \item
      If either contestant reaches 0 {\bf advantage points} or fewer,
      the contest is over.
    \item
      The GM then hazards a number of {\bf APs} for the
      {\bf resistance}.
    \item
      Roll a die to determine your {\bf result}, then apply any
      {\bf bumps}. Your GM does the same.
    \item
      Compare the results of the two {\bf results} to determine who
      loses {\bf AP}; only when you have a {\bf critical} can you gain
      {\bf AP} from your opponent. (see §5.3.2.3).
    \item
      If either contestant reaches 0 {\bf advantage points} or fewer,
      the contest is over.
    \stopitemize
  \stopitemize
\item
  Award {\bf experience points} if appropriate (see §8.1).
\stopitemize

\subsection[title={Advantage Points},reference={advantage-points}]

\subsubsection[title={Starting AP
Totals},reference={starting-ap-totals}]

You describe your action towards the desired {\bf prize} and what
{\bf ability} you use. The {\bf ability} used in the contest can be
varied, but {\bf APs} are always calculated on the first {\bf ability}
that you use in a contest. That {\bf ability} must be used in the first
{\bf round}. Figure your starting {\bf advantage point (AP)} total using
the {\bf TN}, including all {\bf modifiers} and {\bf augments}. The
{\bf AP} include +20 for each level of {\bf mastery}, and can also be
increased by {\bf followers}.

The GM figures starting {\bf APs} for the {\bf resistance} from the
{\bf resistance} {\bf TN}.

\subsubsection[title={Bidding Advantage
Points},reference={bidding-advantage-points}]

You gamble a number of your {\bf APs} in an attempt to reduce your
opponent's {\bf AP}, but if you fail the attempt you lose the {\bf AP}.

You describe your action towards the desired {\bf prize}, what
{\bf ability} you use, and how much risk you take. \quotation{I want to
climb straight up to that outcrop, taking chances if needed.} You can
specify your {\bf AP bid}; if you do not, your GM determines this based
on the amount of risk you are taking.

The size of the {\bf bid} mirrors how bold and risky your character's
action is. Extreme or aggressive actions mean a high {\bf AP bid}, and
cautious actions require less. If you describe an all-out offensive with
your sword cutting vicious arcs, you need to bid a lot of {\bf APs}; if
you say that you are circling your foe cautiously, a low {\bf bid} is in
order. Your GM will look at the level of risk you are taking, and may
suggest that you change your {\bf bid} to better match your actions. If
you do not declare a {\bf bid} before rolling the die, your GM will
decide how many points are {\bf bid} (using 3 as a default), with
riskier actions calling for higher {\bf AP bids}.

\subsubsection[title={Losing Advantage
Points},reference={losing-advantage-points}]

The number of advantage points lost by a contestant is a multiplier of
their bid. Determine the multiplier used as follows:

\startitemize[packed]
\item
  On a tie, both contestants lose ½x bid. Round up.
\item
  If the loser had the same result, but a worse roll, they lose ½x bid.
  Round up.
\item
  If the loser had one level of difference, such as {\bf success} vs
  {\bf failure} or {\bf failure} vs {\bf fumble}, they lose 1x their
  bid.
\item
  If the loser had two levels of difference, such as {\bf success} vs
  {\bf fumble}, they lose 2x their bid.
\item
  if the loser has three levels of difference, a {\bf critical} vs
  {\bf fumble}, they lose 3x their bid.
\item
  If the winner has a {\bf critical}, the {\bf APs} lost by the loser
  are gained by the winner---a transfer.
\stopitemize

\subsubsection[title={Followers and Advantage
Points},reference={followers-and-advantage-points}]

{\bf Followers} can act in different ways during a {\bf contest},
{\bf augmenting} you with their {\bf abilities} or allowing you to use
one of your {\bf abilities} as if it were your own. Alternatively, a
{\bf follower} with a relevant {\bf ability} or {\bf keyword} can simply
add their {\bf APs} to the PC's at the beginning of the {\bf contest}.

Remember to figure any {\bf modifiers} into your {\bf follower's}
{\bf ability} before adding it to your starting {\bf AP} total.

Neither you nor the GM makes rolls for {\bf followers}. Instead, their
actions are subsumed into yours. The {\bf follower's} relevant
{\bf ability} or {\bf keyword} is used solely as a source of
{\bf advantage points}.

You can assign your {\bf followers} to someone else, although you may
have to succeed at a contest to persuade a reluctant follower to go
along.

\subsubsection[title={Advantage Point
Knowledge},reference={advantage-point-knowledge}]

Once your opponent has won or lost {\bf APs} during the current contest,
you can ask the GM what the opposition's {\bf AP} total is. This is
where the element of skill comes in. When choosing how many {\bf APs} to
stake, you must weigh the effect they want to gain if you succeed versus
the risk you face if the action fails.

\subsubsection[title={Advantage Point
Recalculation},reference={advantage-point-recalculation}]

{\bf Advantage points} are only relevant for the length of a particular
{\bf contest}. Your PC does not have any until the next {\bf extended
contest} begins, when you calculate them all over again.

\subsection[title={Extended Contest
Outcomes},reference={extended-contest-outcomes}]

When your GM determines {\bf consequences} and {\bf benefits} they can
use the final {\bf AP} totals

In a {\bf group extended contest} the side that has the last undefeated
contestant gains the {\bf prize}.

Your GM may apply {\bf consequences} and {\bf benefits} as they see fit.
The {\bf scale} of those {\bf consequences} and {\bf benefits} may be
guided by the difference in {\bf action points} between the two sides.

\subsection[title={Parting Shot},reference={parting-shot-1}]

When you {\bf defeat} an opponent in an {\bf extended contest}, you can
act again immediately to try to make their {\bf consequences} more
severe. This is called a {\bf parting shot}.

In the {\bf round} immediately after you take an opponent out of the
{\bf contest}, you may attempt to gain another {\bf prize} from your
opponent suffers by engaging in a {\bf parting shot}. This is an attempt
(metaphoric or otherwise) to kick your opponent while he's down:

\startitemize[packed]
\item
  Striking an incapacitated enemy
\item
  Attacking a retreating army
\item
  Attacking one more punitive rider to a legal settlement
\item
  Demanding additional money from a business partner
\item
  Delivering one last humiliating insult
\stopitemize

You should agree an additional {\bf prize} that you desire beyond the
stakes agreed at the beginning of the contest. If you succeed in a
{\bf parting shot} you will also gain that {\bf prize}. Your GM should
agree that the additional {\bf prize} makes sense as an opportunity
brought about by your opponent's {\bf defeat}.

Your GM should not use a {\bf parting shot}.

You once again {\bf bid} {\bf AP} and use an appropriate {\bf ability}
against your opponent. Your GM must agree that the size of your {\bf AP}
{\bf bid} is sufficient to gain the additional {\bf prize}. The greater
the {\bf prize} the more risk that failure will bring them back into the
contest, and so the higher the {\bf bid} must be. If you succeed, their
{\bf AP} will decrease; their {\bf outcome} may or may not change, but
they cannot finish the {\bf round} by taking an action against you.

{\bf Parting shots} are risky; if you fail, an {\bf AP} transfer might
bring your opponent back into the {\bf contest}. Your stumble can give
them an opening that they can exploit in an effort to snatch
{\bf victory} from the jaws of {\bf defeat}.

\subsection[title={Desperation Stake},reference={desperation-stake}]

You can stake more {\bf advantage points} than you currently have, to a
maximum of your starting {\bf AP} total. This allows you to attempt a
{\bf desperation stake} even when you are within a single {\bf AP} of
{\bf defeat}. Your GM can never stake more {\bf advantage points} than
they have.

\subsection[title={Asymmetrical
Exchange},reference={asymmetrical-exchange}]

If you are engaged, you may choose to briefly suspend your attempt to
best your opponent in an {\bf extended contest}, in order to do
something else. An instance where you are trying to do something else
and your opponent is trying to win the {\bf contest} is called an
{\bf asymmetrical exchange}.

In an {\bf asymmetrical exchange}, you do not score {\bf APs} against
your opponent if you win the {\bf exchange}. Instead, you succeed at
whatever else you were doing. You still lose {\bf AP} if you fail. Often
you will be using an {\bf ability} other than the one you've been waging
the {\bf contest} with, one better suited to the task at hand. This
becomes additionally dangerous when the {\bf score} associated with your
substitute {\bf ability} is significantly lower than the one used for
the rest of the {\bf contest}.

In addition to secondary objectives, as in the above example, you may
engage in {\bf asymmetrical exchange} to grant {\bf augments} (see §2.5)
to yourself or others.

\subsection[title={Switching
Abilities},reference={switching-abilities-1}]

You can usually switch freely from one {\bf ability} to another in the
middle of an {\bf extended contest}. It makes sense to do so if you
think a different {\bf ability} will yield an advantage.

Your {\bf AP} total stays the same when you change your {\bf ability},
so it makes sense to start the contest with your best {\bf ability}
(appropriate to your goal, of course). If this seems odd, remember that
{\bf advantage points} measure advantage---how well the character is
doing in the contest at the current moment. They do not measure
proficiency; that is what the {\bf target number} is for.

When you switch {\bf abilities}, your {\bf prize} does not change, just
the means by which you pursue it.

\subsection[title={Disengaging},reference={disengaging-1}]

To disengage from an {\bf extended contest} when your opponent is
actively trying to keep you in the conflict, use an {\bf asymmetrical
exchange} (see ±5.3.6). You use an {\bf ability} relevant to your
attempt to disengage; the opponent counters with the {\bf resistance}
or, if a PC, an appropriate {\bf ability}. If the GM attempts to
disengage, they use the {\bf resistance} to do so. These {\bf abilities}
may or may not be those used in the main {\bf contest}.

On any {\bf victory}, you are able to leave the {\bf contest}.

If you withdraw from a {\bf group extended contest} and later decide to
rejoin it (or are forced to), you rejoin with the {\bf advantage point}
total you had when you left. If you can show how your leaving and
returning substantially changes the situation, the GM may restore some
of your {\bf AP}---for example, if you leave a street fight to get your
{\bf followers} from a nearby tavern. Leaving a {\bf contest} just to
pick up a weapon or catch your breath is an {\bf unrelated action}, and
does not change your {\bf advantage points}.

\subsection[title={AP Gifting},reference={ap-gifting}]

If you are uninvolved in the contest you can also increase a
participant's {\bf AP} total. First, agree the {\bf tactic} you are
using to help the engaged participant. Second, figure your {\bf APs}
from that ability (see ±5.3.2.1). You {\bf bid} a number of {\bf APs}
which may not exceed your {\bf target number}. The {\bf resistance} is
twice the {\bf bid}.

On a {\bf victory} you transfer that number of the participant. On a
{\bf defeat} you transfer that number to the participant's opponent.

\subsection[title={Edges and Handicaps},reference={edges-and-handicaps}]

Your GM may want rules to represent opponents who strike rarely but with
great effect or who strike often but with little impact per blow. The
first quality can be represented with an {\bf edge}; the second, with a
{\bf handicap}. {\bf Edges} and {\bf handicaps} are designated using ^
(^5, for example), {\bf handicaps} with a minus sign (--^5).

{\bf Edges} and {\bf handicaps} affect only the {\bf advantage points
bid} in an {\bf extended contest}. Your {\bf edge} is added to your
{\bf AP bid} when your opponent must {\bf lose} or {\bf transfer APs}.
Your {\bf handicap} is subtracted from your bid when your opponent
{\bf loses} or {\bf transfers APs}. A contestant's {\bf edge} or
{\bf handicap} never affects his {\bf AP} when he defends, only when he
is attacking.

Most GMs find {\bf edges} and {\bf handicaps} more trouble than they're
worth, and depict these phenomena with description alone. Earlier books
made more extensive use of {\bf edges} and {\bf handicaps} to represent
the quality of equipment carried by the PCs. For example, your suit for
chainmail might be ^4 and your sword ^3. In games where restricted
access to equipment is a significant part of the setting and your GM
wants to use extended contests it may make sense to use them, otherwise
we recommend ignoring them.

\section[title={Group Extended
Contests},reference={group-extended-contests}]

When an {\bf extended contest} involves three or more contestants, it is
a {\bf group extended contest}. The conflict is often between two
groups; each side wants to knock the other out of the contest by
reducing all of its opponents to 0 or fewer {\bf APs}.

Sometimes a contest will be a free-for-all involving three or more
groups.

\subsection[title={Procedure},reference={procedure-5}]

\startitemize[n,packed][stopper=.]
\item
  Your GM {\bf frames the contest}.
\item
  You choose a {\bf tactic}, and figure your PC's {\bf target number}
  ({\bf TN}) using the {\bf score} of your {\bf ability}, plus or minus
  {\bf modifiers} the GM may give you. Figure your starting
  {\bf advantage point (AP)} total (see §5.3.2.1).
\item
  The GM determines the {\bf resistance}. The GM opposes the PC with a
  {\bf resistance}---the harder the task or tougher the opponent, the
  higher the {\bf resistance}. The GM figures starting {\bf APs} for the
  {\bf resistance} (see §5.3.2.1).
\item
  You describe your action towards the desired {\bf prize} and bid
  {\bf APs}* (see §5.3.2.2).
\item
  The GM describes actions for the resistance and bids {\bf APs} (see
  §5.3.2.2).
\item
  The GM determines the order of action from highest {\bf bid} to
  lowest: a {\bf bid} of 20 {\bf APs} goes before a {\bf bid} of 5
  {\bf APs}. (In case of a tie, the contestant whose {\bf bid} is the
  most daring goes first.)
  \startitemize[n,packed][stopper=.]
  \item
    In order of action
    \startitemize[n,packed][stopper=.]
    \item
      Decide if you want to defer your action, you can jump back into
      the order at any point.
    \item
      Roll a die to determine your {\bf result}, then apply any
      {\bf bumps}. Your GM does the same.
    \item
      Compare the results of the two {\bf results} to determine who
      loses {\bf AP}; only when you have a {\bf critical} can you gain
      {\bf AP} from your opponent. (see §5.3.2.3)
    \item
      If either contestant reaches 0 {\bf advantage points} or fewer,
      the contest is out of the contest.
    \stopitemize
  \stopitemize
\item
  When all characters still in the contest have completed their action
  the {\bf round} ends and a new one begins.
\item
  When one side has reduced all of its opponents to 0 or fewer {\bf APs}
  the contest ends.
\stopitemize

If your chosen opponent is knocked out before your PC acts, the GM
decides if you can change your declared action. If you defer your action
and jump back in, the GM decides if you can change your declared action.

\subsection[title={Group Extended Contest
Outcomes},reference={group-extended-contest-outcomes}]

In a {\bf group extended contest} the side that has the last undefeated
contestant gains the {\bf prize}.

Your GM may decide that you suffer individual {\bf consequences} or gain
individual {\bf benefits} from the {\bf outcome} of the contests you
pursued, regardless of whether your team won or lost. You might be on
the winning team, but lost your individual {\bf contest} and suffer a
{\bf penalty} to ongoing actions as a result. Alternatively, you might
be on the losing team, but win your individual {\bf contest} and gain a
{\bf benefit} as a result. If you lost, your {\bf benefit} should not be
the {\bf prize} but instead reflect a side-affect of your individual
triumph. Similarly, if your team won, your {\bf consequence} should not
limit the {\bf prize} which your team one, but should reflect a
side-affect of your individual loss.

See §5.3.3 for a discussion of {\bf consequences} and {\bf benefits} in
{\bf scored contests}.

\subsection[title={Second Chance},reference={second-chance}]

If your PC falls to 0 or fewer {\bf advantage points} in a standard
{\bf extended contest}, you are {\bf defeated}. In a {\bf group extended
contest}, however, you can try a {\bf second chance} to stay in the
{\bf contest}. A {\bf second chance} represents the knack to come back
when your opponent turns away to gloat or deal with the other player
characters. A character may only attempt one {\bf second chance} in any
{\bf contest}.

To attempt a {\bf second chance}, you must be free from attention by the
opposition. You must spend a {\bf story point}. This does not provide a
{\bf bump} up on the roll to come; it is the cost of performing a
{\bf second chance}. You can use a relevant {\bf ability} in a
{\bf simple contest} against the number of {\bf APs} your PC is below 0.
Even if you succeed, a {\bf consequences} applies: take a --6 to further
actions in this contest.

If you win the {\bf simple contest}, you rejoin the contest with a
positive {\bf AP} total. Your new total is a 1/4 of your original
{\bf AP} total at the outset of the {\bf contest}, round up.

Your GM should not use a {\bf second chance} for the {\bf resistance}.

Your GM may decide to impost a {\bf consequence} on you, even if you are
later victorious in a contest, or your team wins the prize, that
represents the adversity you suffered that brought you initially to
defeat.

\subsection[title={AP Lending},reference={ap-lending}]

{\bf AP lending} is a common and important option in {\bf extended
contests}. You can transfer some or all of your {\bf advantage points}
to another PC engaged in a {\bf group extended contest} on your side.
With more {\bf advantage points}, they can stay in the {\bf contest} for
longer, or make larger {\bf bids} without driving themselves to
{\bf defeat}.

You cannot lend {\bf advantage points} to yourself.

If a {\bf follower's AP} are already included in your {\bf AP} total,
the {\bf follower} cannot lend them to you.

Use an {\bf unrelated action} and describe what your character is trying
to do to improve the position of the target. For example, your PC might
throw them a weapon, jeer at an opponent, or simply shout words of
encouragement. Then, state the number of {\bf AP} you are trying to
{\bf lend}. (The GM may suggest a higher or lower {\bf bid} based on the
action you describe.) This determines the {\bf resistance} you face in a
{\bf simple contest}. You lost the {\bf APs} whether or not you
{\bf succeed} in the contest.

\section[title={Chained Contest},reference={chained-contest}]

{\bf Chained contests} do not defer {\bf consequences} to the end of the
{\bf contest}, instead your GM applies the {\bf consequences} to the
loser in the {\bf contest} immediately following a {\bf round}. This
leads to a grittier feel to the {\bf contest}, but at the price of a
death spiral: once you lose the {\bf consequences} make it likelier that
you will lose again. {\bf Chained contests} tend to produce the most
extreme {\bf outcomes}, as participants tend to accumulate significant
{\bf consequences}.

To run an {\bf chained contest} your GM runs a {\bf simple contest} as
normal, and then applies {\bf consequences} (see §2.6) with immediate
effect.

You decide if you wish to continue the {\bf chained contest}, and your
GM makes a similar determination for the {\bf resistance}. Both you and
your GM then express your intent. If your or your GM wishes to continue,
play out another {\bf simple contest}. If you, or your GM, wishes to
{\bf disengage}, then on a {\bf victory} you leave the contest, without
inflicting consequences on the opposition. If both you and the GM wish
to leave the contest, then you both {\bf disengage}, and the contest
ends. Award {\bf experience points} if appropriate (see §8.1).

You can switch {\bf abilities} within the contest, provided your GM
agrees that the new {\bf ability} represents a suitable {\bf tactic} to
obtain the {\bf prize}.

If the {\bf penalties} suffered by one contestant reduce their
{\bf ability} below 0, they must concede the contest, If it makes sense,
your GM may allow you to continue by switching {\bf abilities}. The GM
may decide that accumulated {\bf penalties} apply to the new ability
used in the contest---this may still force you to concede.

A {\bf chained contest} can benefit from using the difference between
{\bf results} as a {\bf rank} when your GM assesses the {\bf benefits
and consequence} (see §2.7.2).

{\bf Chained contests} are asymmetric, in that they accumulate
{\bf consequences} for the loser, and not {\bf benefits} for the winner.
Your GM may decide to award {\bf benefits} to a victorious PC after the
{\bf contest} ends (see §2.6).

\subsection[title={Group Chained
Contest},reference={group-chained-contest}]

In a {\bf group chained contest} opponents pair off and fight a series
of {\bf chained contest} {\bf rounds} with each other.

Your GM should determine the order of action, but as all rounds
represent actions by both aggressor and defender there is no advantage
to be obtained by going first. If there are surplus characters on your
side, you may engage an already engaged opponent in a second
{\bf contest}; your GM may choose to apply a {\bf penalty} to them as
they are already engaged with one opponent. Alternatively you may choose
to {\bf augment} an existing player character, reflecting aiding them in
their fight instead.

\subsection[title={Group Chained Contest
Outcomes},reference={group-chained-contest-outcomes}]

In a {\bf group chained contest} the side that has not conceded gains
the {\bf prize}.

Individual {\bf consequences} or {\bf benefits} will have already been
determined by the {\bf chained contest} {\bf outcomes} on each
{\bf round}.

\subsection[title={Followers in a Chained
Contest},reference={followers-in-a-chained-contest}]

{\bf Followers} may augment your character in a {\bf chained contest}.

In addition, if you suffer a defeat in a round of a {\bf chained
contest} you may transfer that {\bf outcome} to a follower, but they
suffer a {\bf state of adversity} one level worse than you would do, so
marginal becomes minor etc., and the {\bf follower} is removed from the
{\bf contest}.

\section[title={Extended vs Scored Contests vs Chained
Contests},reference={extended-vs-scored-contests-vs-chained-contests}]

We recommend that your GM chooses ONE form of {\bf long contest} only,
and stick to it, within a given campaign of {\em QuestWorlds}. If in
doubt, use a {\bf scored contest} by default. We also recommend that
game designers choosing to use {\em QuestWorlds} as the basis of their
own game, choose ONE form of {\bf long contest} to include. This
document is comprehensive to allow designers and GMs to choose.

{\bf Scored contests} have the advantage of speed and simplicity.
{\bf Extended contests} have the advantage of each {\bf exchange}
allowing both parties to take turns acting, over your GM adjudicating
who has the initiative; the bidding system also adds drama. {\bf Chained
contests} offer the benefit of grittier exchanges where the
{\bf outcomes} of each {\bf round} have impact, as opposed to being
\quote{cosmetic} until the end of the {\bf contest}.

{\bf Scored contests} require more interpretation by your GM, to
determine who has the initiative and describe the nature of the next
{\bf round}. {\bf Extended contests} drama comes at the cost of
increased complexity, and some harder to interpret corner cases.
{\bf Chained contests} create a death spiral which can be hard to break
out of.

\section[title={Extremely Long
Contests},reference={extremely-long-contests}]

There's no particular time scale associated with {\bf contests}. But
some {\bf contests} may by their very nature be a drama that can't be
resolved at one point in the narrative. Examples include political
campaigns, construction projects, or seductions. These can be resolved
by {\bf long contests} where each {\bf round} is conducted at an
appropriate moment, rather than sequentially. Your GM will need to keep
track of the {\bf resolution or advantage points} and the
{\bf resistance}, though this might change as the context changes (a
civil war started by the players could impede their castle-building
plans). The challenges of each round will vary, and you may use a
different {\bf ability} or {\bf augment} in the next exchange.

\chapter[title={Relationships},reference={relationships}]

Abilities may represent your relationship to NPCs.

\section[title={Supporting
Characters},reference={supporting-characters}]

Many relationships connect you to NPCs controlled by the GM.

When you try to use one of these relationships to solve a problem, your
{\bf tactic} is your relationship {\bf ability}. You can't simply go to
the {\bf supporting character} you have a relationship with, stick them
with the problem, and expect to see it solved.

If you succeed, the {\bf supporting character} helps you solve the
problem. If you fail, they don't. As with any {\bf ability}, you must
still specify how the NPC goes about overcoming the {\bf story
obstacle}. Calls on relationships are almost always {\bf simple
contests}.

In crucial situations, it may seem dramatically inappropriate for you to
solve a problem indirectly, by working through others. Your GM can
expose the {\bf supporting character} to serious risk. If the character
dies or otherwise suffers a change of status that renders them useless
to you, you permanently lose the relationship {\bf ability}.

Before putting {\bf supporting characters} at serious risk, your GM
should make sure the players understand the magnitude of the possible
consequences.

When {\bf supporting characters} undertake significant risk, the
{\bf supporting character} may suffer a {\bf consequence of defeat}
commensurate with the level of the {\bf defeat} in the {\bf contest}. Or
it may simply be your relationship that is damaged or destroyed.

\section[title={Allies},reference={allies}]

An {\bf ally} is a character of roughly the same level of accomplishment
as you, often in the same or a similar line of work. For every favor you
ask of them they'll ask one of you. These reciprocal favors will be
roughly equivalent in terms of risk, time commitment, resistance class,
and inconvenience.

\section[title={Patrons},reference={patrons}]

{\bf Patrons} enjoy greater access to assets than you, either through
personal ownership (as in a Merchant Prince) or authority (as in the
governor of a province). They may lend you advice or provide you with
assets but are too busy and important to personally perform tasks for
you. They may hire you to do jobs, or issue orders within a command
structure to which you both belong.

When you roll your {\bf patron} relationship, your GM adjusts the
resistance class depending on what you have done for them lately.

\section[title={Contacts},reference={contacts}]

A {\bf contact} is a specialist in an {\bf occupation}, skill, or area
of expertise. {\bf Contacts} provide your information and perform minor
favors, but will expect information or small favors from you in return.

You can describe a {\bf contact} as being a particular individual, or as
a group of similar individuals.

\subsection[title={Occupational
Contacts},reference={occupational-contacts}]

Any {\bf occupational keyword} can be treated as a source of
{\bf contacts}. However, using an {\bf occupational keyword} as a source
of {\bf contacts} will always be a {\bf stretch}. To more reliably draw
on particular {\bf contacts} associated with your occupation, you should
take an explicit ability. Use a {\bf breakout ability} if you are using
{\bf umbrella keywords}.

\section[title={Followers},reference={followers-1}]

A {\bf follower} is a {\bf supporting character} that travels with you
and contributes on a regular basis to your success.

There are two types of followers: {\bf sidekicks} and {\bf retainers}.

{\bf Followers} need not be people, or even sentient beings: you can
write up a spirit guardian, trusty robot, or companion animal as a
{\bf follower}.

\subsection[title={Sidekick},reference={sidekick}]

A {\bf sidekick} is a {\bf supporting character} under your control.
Most of the time they stay at your side to render assistance, but they
can also go off and perform errands or missions on their own.

You should give your {\bf sidekick} a name. You should, when asked,
explain how the {\bf sidekick} came to be your {\bf follower}, and why
they continue in that role.

{\bf Sidekicks} start with three {\bf abilities}, one rated at 16 and
the others at 13. Any of these {\bf abilities} may be a {\bf keyword}.
At least one of them should indicate a {\bf distinguishing
characteristic}.

If the sidekick is nonhuman or a member of an unusual culture, one of
its three starting {\bf abilities} must be its species or culture
{\bf keyword}.

Once you have determined the {\bf sidekick's} base {\bf abilities}, they
allocate 15 additional points between three of them, spending no more
than 10 on any one {\bf ability}.

You can improve these {\bf abilities} through the expenditure of
{\bf experience points}.

You may use any of your {\bf sidekick's abilities} as your own. The
{\bf sidekick} can go off and do things without you.

\subsection[title={Replacing Lost
Sidekicks},reference={replacing-lost-sidekicks}]

As a {\bf consequence of defeats} in which they participated,
{\bf sidekicks} can be killed or leave your service permanently.

Defeat in physical {\bf contests} can lead to literal death.
Metaphorical deaths from non-violent {\bf contests} indicate they break
up with you. The {\bf sidekick} may angrily withdraw from your service,
but is more likely to sorrowfully retire. You may be able to bring a
{\bf sidekick} back from metaphorical death by overcoming {\bf story
obstacles}.

If you lose a {\bf sidekick}, you may create a new one without needing
to spend a {\bf experience point}. You must explain how the new
{\bf sidekick} has come to be your new {\bf follower}.

You may find it convenient to promote {\bf retainers} to {\bf sidekick}
status, giving them names and personalities, with a sudden improvement
in {\bf abilities} and {\bf scores} to match.

\subsection[title={Retainers},reference={retainers}]

A {\bf retainer} is a more or less anonymous servant or helper. You may
specify a single {\bf retainer}, or, where appropriate to your character
concept, an entire staff of them.

Like any other {\bf ability}, a {\bf retainer} {\bf ability} allows you
to overcome relevant {\bf story obstacles} by engaging in a
{\bf contest}. To model the contribution of {\bf retainers}, when you
are acting, you can use them to {\bf augment} your {\bf ability}. Your
GM can rule that {\bf consequences} apply to {\bf retainers}.

{\bf Retainers} generally regard you with all the affection and loyalty
due to an employer or master. If you treat them more poorly than is
expected for their culture, your GM should increase the {\bf resistance
class} of attempts to make use of their talents.

If you lose {\bf retainers} for any reason, you can replace them simply
by providing a convincing explanation of how you go about it.

\section[title={Relationships as
Flaws},reference={relationships-as-flaws}]

Certain relationships with {\bf supporting characters} act as
{\bf flaws}. They impose obligations on you, prompting your GM to
present you with {\bf story obstacles} you have no choice but to
overcome.

\subsection[title={Dependents},reference={dependents}]

A {\bf dependent} is a person, usually a family member or loved one, who
requires your aid and protection. Your GM should periodically create
storylines in which your {\bf dependent} is endangered.

Rather than taking a {\bf dependent} as a {\bf flaw}, you may find it
more fruitful to specify the nature of your relationship as an
{\bf ability}, such as {\em Love for Wife} or {\em Love for Son}.

\subsection[title={Adversaries},reference={adversaries}]

An {\bf adversary} is a rival, enemy or other individual who can be
relied upon to periodically disrupt your plans.

The {\bf adversary's} goals are probably the opposite of yours, although
they could be a bitter rival within the same community, organization, or
movement.

To treat an {\bf adversary} as an {\bf ability}, rather than a
{\bf flaw}, describe your emotional response to them. Examples:
{\em Hates Leonard Crisp}, {\em Fears the Electronaut}, {\em Sworn
Vengeance Against Heimdall}. That way, you still inspire your GM to add
the plot elements you desire, but can use your antipathy toward the
enemy to {\bf augment} your {\bf target number}s against them.

\chapter[title={Story Points},reference={story-points}]

{\em QuestWorlds} design favors pulp stories and cinematic action.
{\bf Story points} mirror the ability of heroes in these genres to
\quotation{cheat death}, or \quotation{escape with one bound}.

Normally, your GM should ensure that {\bf defeat} takes the story for
your PC in an interesting new direction. Unlike some games, where your
goal is to win against challenges set by the GM, in a storytelling game
your goal is to tell a good story together. Just as in fiction the
protagonist can suffer all sorts of reversals, so in a storytelling
game, your PC should suffer all sorts of adversities before they triumph
(or meet their tragic end). As a result, we recommend against the
tendency to \quote{buy off {\bf defeat}} with {\bf story points} in the
middle of the story. Instead, use {\bf story points} when {\bf defeat}
would damage the conception of the character that you have been building
during the story, or lead to an unsatisfactory climax to the story.

Your GM should push the story in an interesting new direction on
{\bf defeat} not send it to a dead end. If there is no interesting
branch from {\bf defeat} they should consider an {\bf assured contest}
instead.

In other genres, it may feel less appropriate that you can \quote{cheat
certain death.} For those genres you can simply drop {\bf story points}
without impacting the game.

In games with a strong player vs.~player element, your GM should
dispense with {\bf story points} as they become disruptive if used
against each other.

\section[title={Story Point Pool},reference={story-point-pool}]

At the beginning of play, your GM will create a {\bf story point pool}
for your group. The {\bf story point pool} has one {\bf story point} per
PC. During play you can {\bf burn} one or more points from this
{\bf pool}, after which it is lost. When you {\bf burn} {\bf story
points} you can edit the story in your PC or group's favor. You can
either do this to improve your {\bf result} via a {\bf bump} (see §7.1)
or to introduce a helpful fact into the world via a {\bf plot edit} (see
\$7.2).

You can decide to spend {\bf story points} at any time. You do not
agreement from the other players to do so.

Your GM may feel that the genre they are playing requires a greater pool
of {\bf story points} as in that genre the heroes never seem to lose.
You can create a larger {\bf story point pool} to reflect this but
beware that failure and reversals of fortune for the hero are part of
most {\em interesting} stories. It can be unwise for the GM to remove
all sense of threat from the players by giving them a {\bf story point
pool} from which they can {\bf bump} any roll, or remove all branches of
the story that stem from {\bf defeat}.

\subsection[title={Refreshing Story
Points},reference={refreshing-story-points}]

Because you burn a {\bf story point} to use it, your {\bf story point
pool} may become exhausted. The GM has three choices for
{\bf refreshing} your {\bf story point pool:}

\startitemize[packed]
\item
  The {\bf story point pool refreshes} at the beginning of every session
  of play.
\item
  The {\bf story point pool refreshes} whenever your PCs engage in
  genre-appropriate downtime. Usually the GM plays this out as a
  montage, asking your character to describe genre appropriate
  activities in this time period. For example: in a police procedural
  series, the PCs might gather at a cop bar to drink and talk about
  their personal problems; in a series about high-school paranormal
  investigators they might gather in the school library to chill with
  their mentor, the librarian, and talk about teenage problems.
\item
  The {\bf story point pool refreshes} whenever the GM deems it
  necessary, based on their desire to allow you to edit the upcoming
  story.
\stopitemize

Ultimately your GM is always the arbiter of when and how the {\bf story
point pool refreshes}. On a refresh your {\bf story points} pool resets
to one {\bf story point} per PC.

\subsection[title={Story Point Pool
Summary},reference={story-point-pool-summary}]

To summarize:

\startitemize[packed]
\item
  At the beginning of a session you have 1 {\bf story points} per PC in
  the pool.
\item
  During the session you may {\bf burn} one {\bf story point} to
  {\bf bump} a PC's roll (see §7.1),
\item
  During the session you may {\bf burn} one or more {\bf story points}
  on a {\bf plot edit}.
\item
  {\bf Story points} that are burned are lost from the {\bf story point
  pool}.
\item
  The GM decides on the conditions to a refresh a {\bf story point
  pool}.
\item
  The {\bf story point pool} refreshes to 1 {\bf story point} per PC in
  the pool.
\stopitemize

\section[title={Bump with Story
Points},reference={bump-with-story-points}]

You can burn a {\bf story point} to gain a {\bf bump} (see §2.3.7).

\section[title={Plot Edits},reference={plot-edits}]

{\em QuestWorlds} is a co-operative game, and you may create details
about the setting as the normal part of narration. Your GM should allow
this, as long as they do not break credibility. So, you may describe
your PC walking over to the pot of soup bubbling on the fire, swiping a
drink from the tray the waiter is carrying at the governor's ball, or
taking the monorail to the next city to continue your investigation.
Your GM should allow these additions without interruption, providing it
does not confer significant advantage to your PC. Mostly this will be
using elements that have already been established as part of the
setting.

A {\bf plot edit} is a more significant moment of good fortune that you
wish to narrate, that provides advantage to your PC. You are not just
describing something that is plausible in the environment, but something
whose existence aids you in overcoming {\bf story obstacles} or
uncovering secrets.

A {\bf plot edit} might be thought of as \quote{fate} or \quote{luck.}

Burning {\bf story points} for a {\bf plot edit} allows you to modify
the setting or environment in your PC's favor. The chance encounter in
the street with an NPC, favorable weather, car keys in the sun visor,
the forthcoming eclipse, the wind that fills the sails.

Your GM is the arbitrator of whether a {\bf plot edit} is allowed. It
should not suspend the disbelief of the other players in the game or
setting or hamper their enjoyment. It should not derail or short-circuit
the game's entertainment. The {\bf plot edit} should, by contrast, be
something that enhances the story for all the players.

The cost, in {\bf story points}, of a {\bf plot edit}, is given by the
following table.

\startplacetable[title={7.3.1.1 PLOT EDIT TABLE}]
\startxtable
\startxtablehead[head]
\startxrow
\startxcell[align=middle,width={0.31\textwidth}] Level \stopxcell
\startxcell[align=middle,width={0.06\textwidth}] Cost \stopxcell
\startxcell[align=middle,width={0.31\textwidth}] Impact \stopxcell
\startxcell[align=middle,width={0.31\textwidth}] Example \stopxcell
\stopxrow
\stopxtablehead
\startxtablebody[body]
\startxrow
\startxcell[align=middle,width={0.31\textwidth}] Marginal \stopxcell
\startxcell[align=middle,width={0.06\textwidth}] 1 \stopxcell
\startxcell[align=middle,width={0.31\textwidth}] A substantive change
that does not alter the situation but offers an alternate avenue for
resolution \stopxcell
\startxcell[align=middle,width={0.31\textwidth}] The gate guard at the
secret government facility tonight is an old war buddy established by
the PC in a prior scene and cemented as a relationship \stopxcell
\stopxrow
\startxrow
\startxcell[align=middle,width={0.31\textwidth}] Minor \stopxcell
\startxcell[align=middle,width={0.06\textwidth}] 2 \stopxcell
\startxcell[align=middle,width={0.31\textwidth}] A substantive change
that does not flow from previously established facts in the story. A
{\em deus ex machina} change \stopxcell
\startxcell[align=middle,width={0.31\textwidth}] The XO of the Patrol
ship is an old drinking buddy of your PC, a fact not previously
established in play \stopxcell
\stopxrow
\stopxtablebody
\startxtablefoot[foot]
\startxrow
\startxcell[align=middle,width={0.31\textwidth}] Major \stopxcell
\startxcell[align=middle,width={0.06\textwidth}] 3 \stopxcell
\startxcell[align=middle,width={0.31\textwidth}] A stroke of good
fortune that is unrelated to prior events and resolves a conflict or
reveals a secret \stopxcell
\startxcell[align=middle,width={0.31\textwidth}] The vampire has failed
to notice the approaching sun rise, which disintegrates them just as
they are about to drain the incapacitated PC \stopxcell
\stopxrow
\stopxtablefoot
\stopxtable
\stopplacetable

\chapter[title={Experience},reference={experience}]

During a session of play your character will have the chance to learn
from experience or overcoming personal obstacles. When your character
learns, they gain a {\bf experience points}. {\bf Experience points} can
be used to improve your character.

\section[title={Earning Experience
Points},reference={earning-experience-points}]

You gain one {\bf experience points} for any of the following:

\startitemize[packed]
\item
  When your {\bf outcome} for a {\bf contest} is a {\bf defeat}.
\item
  Your GM uses a {\bf flaw} or other {\bf ability} against you in a
  contest with you (see §2.6). This happens either when the story forced
  you to confront a {\bf flaw}, or the GM gave you a {\bf hindrance}
  (see §3.4), if the {\bf hindrance} results in a {\bf penalty}.
\stopitemize

You do not gain {\bf experience points} if any of the following apply:

\startitemize[packed]
\item
  You only gain an {\bf experience point} for each of your
  {\bf abilities} or {\bf flaws} once in a session of game play.
\item
  You do not get {\bf experience points} for an {\bf augment}, {\bf AP
  gifting} or {\bf assist}.
\item
  You do not gain an {\bf experience point} from an {\bf assured
  contest}, even if you roll to determine {\bf benefits} or
  {\bf consequences}.
\stopitemize

You can gain a maximum of five {\bf experience points} in any one
session. Once you have earned five {\bf experience points}, you cannot
gain further {\bf experience points} in that session.

\subsection[title={Experience on
Defeat},reference={experience-on-defeat}]

Awarding {\bf experience points} on {\bf defeat} is a self-correction
mechanism.

\startitemize[packed]
\item
  It slows your advance if your PC regularly outclass the
  {\bf resistance}. This pushes your GM to introduce threats that
  {\bf credibly} present a greater threat to your PC.
\item
  If you regularly buy off {\bf defeat} with {\bf story points} you will
  find it harder to advance. In {\em QuestWorlds} your GM should provide
  an entertaining story branch on defeat; you should not need to buy
  {\bf defeat} off, unless it damages your character conception or is
  the climax.
\stopitemize

If the GM finds that the PCs are no longer regularly earning
{\bf experience points} they can consider using {\bf resistance
progression} (see §2.8) to increase the {\bf base resistance} so that
more {\bf contests} will feature a high enough resistance to earn
{\bf experience points}.

\section[title={Improving Your
Character},reference={improving-your-character}]

When you accumulate 10 {\bf experience points}, you can buy an
{\bf advance}. An {\bf advance} allows you to select two of the
following. You cannot choose an element more than once.

\startitemize[packed]
\item
  +9 to a standalone {\bf ability} or breakout {\bf ability}; or +6 to a
  {\bf keyword}.
\item
  +6 to a standalone {\bf ability} or breakout {\bf ability}; or +3 to a
  {\bf keyword}.
\item
  a new standalone {\bf ability} at 13; or a new breakout {\bf ability}
  at + 1.
\item
  a new standalone {\bf ability} at 13.
\item
  Turn a stand-alone {\bf ability} into a {\bf keyword} by adding a new
  +1 breakout {\bf ability} to it.
\item
  Drop a {\bf flaw}, or turn it into an {\bf ability} if story
  appropriate and agreed with the GM.
\stopitemize

In addition, if you have less than three {\bf flaws}, you may add
another, provided it fits the story, when you take an {\bf advance}.

In some genres you may wish to maintain a tally of the total
{\bf experience points} earned as a measure of your reputation.

\subsection[title={Catch-Ups},reference={catch-ups}]

To encourage well-rounded characters, a package deal, called a
{\bf catch-up}, becomes available whenever you acquire via improvement a
new {\bf mastery} in one of your {\bf abilities} ({\bf keyword} or
stand-alone). Any time one of your {\bf scores} crosses a {\bf mastery}
threshold (i.e.~20 -> 21, 40 -> 41, etc). you may also improve up to
three {\bf abilities} or {\bf keywords} of your choice by three points
each, as long as the chosen {\bf abilities} are currently rated five or
more points lower than your newly adjusted {\bf scores} in the raised
{\bf ability} that triggered the {\bf catch-up}.

You may not increase the bonus of {\bf breakout abilities} under a
{\bf keyword} with a {\bf catch-up}, nor does net effective value of a
breakout {\bf ability} crossing a {\bf mastery} threshold trigger a
{\bf catch-up}. Only a {\bf keyword}'s base {\bf scores} is considered
in this context.

\subsection[title={Rate of Advancement},reference={rate-of-advancement}]

We assume an average earning rate of two {\bf experience points} per
session. This would lead to you gaining an advance every five sessions.
If your rate is lower than one {\bf experience point} a session, your GM
should choose one of these options:

\startitemize[packed]
\item
  Provide more credible threats
\item
  Use {\bf resistance progression}
\item
  Reduce the cost of an {\bf advance} to five {\bf experience points}.
\stopitemize

\subsection[title={Directed
Improvements},reference={directed-improvements}]

On occasion your GM may increase one of your {\bf abilities}, by +3, +6
or +9, or give you a new {\bf ability}, usually rated at 13. These are
called {\bf directed improvements}.

{\bf Directed improvements} are usually rewards for overcoming
particularly important or dramatic {\bf story obstacles}.

Your GM will tend to use them to raise {\bf abilities} that would
otherwise fall behind, but should increase due to story logic, or
introduce new {\bf abilities} for the same reason.

Your GM might give you a new {\bf flaw} to represent a story outcome
from a contest, that leads you with a hindrance to future action. If you
have three or more {\bf flaws} you can ask your GM to drop one in favor
of the new {\bf flaw}, if you it seems story appropriate.

\subsection[title={Timing of
Improvements},reference={timing-of-improvements}]

Your improvements happen immediately, when you cross the threshold to
buy an {\bf advance}, or a GM awards you a {\bf directed improvement}.

\section[title={Milestone
Improvements},reference={milestone-improvements}]

Your GM may decide that they do not want to track {\bf experience
points} earned during a game. In this case they may switch to
{\bf milestone improvement}.

Under {\bf milestone improvements} the GM simply declares that your PCs
have reached a point in the story where we should see them improve their
{\bf abilities} and award you an {\bf advance} (see §8.2).

Your GM should not use both {\bf experience points} and {\bf milestone
improvements} but choose one. If in doubt, choose {\bf experience
points} as the default. {\bf Milestone improvements} do not naturally
balance against the {\bf resistance} and the GM may need to use
{\bf resistance progression} to continue to up the threat level against
your PCs (see §2.8).

\chapter[title={Community Resources and
Support},reference={community-resources-and-support}]

Some series revolve around the relationship between a band of
influential figures and the community they protect. In defense of the
community, they can {\bf bolster}, expend, and juggle its various
{\bf resources}.

These rules allow your GM to track the rise and fall of the fortunes of
your community, and your impact on them.

If your GM intend to play a game centered around a community, you should
have a relationship {\bf ability} to that community.

It is possible that you have relationships with other communities that
are not the focus of play. Treat these relationships as {\bf abilities}
that you can call on, but your GM should not track these communities
with these rules. Your GM should pick the level of community that
provides the greatest dramatic potential from its competition for
{\bf resources}, friendly or otherwise, with its rivals.

Some campaigns do not center on a community, with the adventurers being
footloose wanderers. In that case, even if you have community
{\bf abilities}, your GM will not track any community. Before you decide
this though, consider where your PCs might turn for help, succor, or
aid. Is there somewhere in the campaign defined as a place of refuge and
safety for you. It may well be that there is a community, the bar where
other footloose adventurers all meet, who will help each other out in a
tight spot for example, that your GM can model.

\section[title={Community Design},reference={community-design}]

\subsection[title={Defining Resources},reference={defining-resources}]

Communities have {\bf resources} that your GM defines. Your PC can try
to draw on their community's {\bf resources} to use them as {\bf bonus}.
If your community is in difficulty, a strained {\bf resource} might act
as a {\bf penalty}. Your GM should focus on no more than five or so
broadly-labeled {\bf resource} types, so that the PCs can care about
(and have a chance of successfully managing) all of them.

Most communities have variants of the following {\bf resources}, perhaps
with more colorful names:

\startitemize[packed]
\item
  Wealth --- the capacity of the community to provide financial help,
  whether counted primarily in dollars, credits, or cattle
\item
  Diplomacy --- the relationships with other groups though which a
  community can obtain favors, while minimizing the cost of its
  reciprocal obligations
\item
  Morale --- the community's resolve to achieve its goals, and
  willingness to follow the directives of its leaders
\stopitemize

The following abilities might appear, depending on setting:

\startitemize[packed]
\item
  Military --- its capacity to defend itself from outside threats, and
  to aggressively achieve its own aims through force of arms (for
  settings where communities of the size you're tracking field their own
  armed units)
\item
  Magic --- the capability of a community to perform supernatural acts
  (for fantasy worlds)
\item
  Technology --- its access to specialized, rare or secret devices or
  scientific knowledge not shared by its rivals (for post- apocalyptic
  or SF worlds)
\stopitemize

Similar communities in the genre, should have the same set of
{\bf resources}.

\subsection[title={Assigning Ranks},reference={assigning-ranks}]

Your GM distributes the following {\bf ranks} between the five
{\bf resources}: +M, +9, +6 +3 and 0. Note that the size of the group
doesn't affect the {\bf ranks}.

\subsection[title={Community
Questionnaires},reference={community-questionnaires}]

Your GM may create a questionnaire that asks the players to make choices
about the priorities of their community. The responses to each question
should be multiple-choice. Each choice you make adds points to a score
for each {\bf resource} type. Points are awarded according to what the
answer reveals about the community's relative priorities. An answer may
give points to more than one {\bf resource}.

You can choose your answers by consensus, majority vote, or take turns.

When you're done, rank the {\bf resources} in the order of the scores.
Assign the high {\bf ranks} to the highest {\bf scores} and the lowest
to the low.

A questionnaire also introduces your setting in a punchy, interactive
format, and tailors the community to the players' desires, increasing
their investment in it.

\section[title={Drawing on Resources},reference={drawing-on-resources}]

You can use community {\bf resources} as a {\bf bonus} to your
{\bf abilities} after convincing the community to let you expend
precious assets. This requires a preliminary {\bf contest} using a
social {\bf ability}, most likely your community relationship. Your GM
will use a {\bf moderate resistance} as the baseline, with higher
{\bf resistance}s when your proposals seem selfish or likely to fail,
and lower ones when everyone but the dullest dolt would readily see
their collective benefits. Your GM may increase {\bf resistance}s if
your group draws constantly on community {\bf resources} without
replenishing them.

The lobbying effort and the actual resource use require framing, a clear
description of what you are doing, and other details to bring them to
fictional life. You cannot use {\bf resource abilities} directly, but as
an {\bf bonus} to your own {\bf abilities}.

Use of community {\bf resources} should pass the threshold for being
{\em memorable} and {\em entertaining}. Normally there should be a clear
benefit to the community, or risk to the community. The PC's actions
should be in support of the community, not themselves. Community
involvement becomes part of the story. A certain amount of routine
support for your character is assumed; a {\bf bonus} implies that the
community is expending abnormal effort on your behalf, that will cost
the community itself.

\subsection[title={Resource Depletion},reference={resource-depletion}]

Unlike character abilities, each use of community {\bf resources}
temporarily {\bf depletes} it. Regardless of {\bf outcome} a
{\bf resource} drops a {\bf rank} when used.

Your GM decides when a {\bf resource} is restored to its original value.
Your GM should decide what the credible interval is for the community to
recover from the expenditure of effort. At that point, your GM restores
the {\bf bonus} for the {\bf resource}.

You might chose to use a {\bf resource} when it is already depleted, in
which case you use its lower {\bf rank}. Your GM may use this to
represent attrition to your community from a continued struggle. A
{\bf resource} that is depleted enough, may become a {\bf penalty}.

Threats to community {\bf resources} act as a spur to PC action. Your GM
may rule that the {\bf outcome} from a {\bf contest} where you did not
use the {\bf resource} may still deplete the rank of a community
{\bf resource}.

\subsection[title={Required Resource
Use},reference={required-resource-use}]

As part of your GM's setting design, they may specify that certain
actions in a setting always require the use of a community
{\bf resource}. Because the {\bf resource} use is obligatory, it need
not meet the usual criteria for entertainment value.

\subsection[title={Resource as a
Penalty},reference={resource-as-a-penalty}]

A {\bf resource's rank} may fall below 0. If you require use of a
community's {\bf resources} (see §9.2.2) your actions will be subject to
a {\bf penalty} equal to the resource rank.

\subsection[title={Bolstering
Resources},reference={bolstering-resources}]

Your GM may offer you the opportunity to {\bf bolster} a community
{\bf resources} ahead of need by seeking out and overcoming relevant
{\bf story obstacles}. If you succeed, the community resource improves a
{\bf rank}. Your GM will set the {\bf resistance} for the {\bf bolster}.
The community's higher {\bf ranked} resources should have higher
{\bf resistances} to {\bf bolstering}. As a default, use the current
{\bf rank} as the resistance to {\bf bolstering}.

For clarity, a {\bf resource} rated at +M can be bolstered to +M2.

{\bf Bolstering} lasts until the {\bf resource} is used. When your GM
depletes a {\bf bolstered resource} following usage, they remove only
the additional {\bf rank} from {\bf bolstering}.

If a {\bf resource} is already suffering from a {\bf penalty},
bolstering removes that {\bf penalty} instead of improving the
{\bf rank}.

\subsection[title={Background Events},reference={background-events}]

In the background all sorts of other events periodically alter the
community's prosperity. These include the actions of other community
members, who are {\bf depleting and bolstering resources} all the time,
as well as the unexpected intrusion of outside forces.

Your GM may decide that the community's {\bf rank} in a {\bf resource}
is temporarily at a higher or lower {\bf rank} due to these outside
events. Your GM decides when the {\bf resource} returns to normal. For a
lower {\bf rank}, this may require you to overcome a {\bf story
obstacle}.

\chapter[title={Appendix},reference={appendix}]

\section[title={Glossary of Terms},reference={glossary-of-terms}]

\startdescription{{\bf Ability}}
  Anything you can apply to solve a problem or overcome an obstacle.
\stopdescription

\startdescription{{\bf Advance}}
  A package of improvements to your {\bf abilities} and {\bf keywords}
  earned through {\bf experience points} or {\bf milestone advancement}.
\stopdescription

\startdescription{{\bf Advantage Point (AP)}}
  A measure of advantage in an {\bf extended contest}.
\stopdescription

\startdescription{{\bf Ally}}
  A {\bf supporting character} of roughly equal ability to your own.
\stopdescription

\startdescription{{\bf AP}}
  Abbreviation for Advantage Point.
\stopdescription

\startdescription{{\bf AP Gifting}}
  When you help another character, whilst uninvolved in a {\bf contest},
  by giving them {\bf advantage points} in an {\bf extended contest}.
\stopdescription

\startdescription{{\bf AP Lending}}
  When you help another character, whilst engaged in a {\bf contest}, by
  lending them {\bf advantage points}, in an {\bf extended contest}.
\stopdescription

\startdescription{{\bf Asymmetrical Exchange}}
  In a {\bf extended contest}, where you are pressed by an opponent, but
  want to do something other than contend directly for the {\bf prize}.
\stopdescription

\startdescription{{\bf Asymmetrical Round}}
  In a {\bf scored contest}, where you are pressed by an opponent, but
  want to do something other than contend directly for the {\bf prize}.
\stopdescription

\startdescription{{\bf Assist}}
  In a {\bf scored contest}, if you are unengaged you may use an
  {\bf assist} to reduce the {\bf resolution points} scored against
  another character.
\stopdescription

\startdescription{{\bf Augment}}
  Using one {\bf ability} to help another {\bf ability}.
\stopdescription

\startdescription{{\bf Assured Contest}}
  You have an appropriate {\bf ability} and the GM feels {\bf failure}
  is not interesting, or makes the PC looks un-heroic.
\stopdescription

\startdescription{{\bf Background Event}}
  An off-stage {\bf bonus} or {\bf penalty} applied to a {\bf resource}.
\stopdescription

\startdescription{{\bf Base resistance}}
  The {\bf TN} for a {\bf moderate resistance class}, from which all
  other {\bf resistance classes} are figured as a {\bf bonus} or
  {\bf penalty}.
\stopdescription

\startdescription{{\bf Benefit of Victory}}
  Long term positive modifier, because you won a {\bf contest}, against
  a challenging opponent (not −6 or less than your {\bf ability}).
  Usually a {\bf state of fortune}.
\stopdescription

\startdescription{{\bf Bid}}
  Also an {\bf AP Bid} or {\bf advantage point bid} is your wager in an
  {\bf extended contest}.
\stopdescription

\startdescription{{\bf Bolster}}
  A {\bf story obstacle} to apply a bonus to a community {\bf resource}
\stopdescription

\startdescription{{\bf Bonus}}
  A positive modifier.
\stopdescription

\startdescription{{\bf Boost}}
  Spending points after a {\bf group simple contest outcome}, to improve
  the victory.
\stopdescription

\startdescription{{\bf Bump}}
  An increment of the {\bf result} of a roll, up or down. So a bump up
  moves a {\bf fumble}, to a {\bf failure}, to a {\bf success} to a
  {\bf critical}, a bump down moves a {\bf critical}, to a
  {\bf success}, to a {\bf failure} to a {\bf fumble}. One step is moved
  per {\bf bump}. It is usually the impact of a {\bf story point} or
  {\bf mastery}.
\stopdescription

\startdescription{{\bf Burn}}
  Using a {\bf story point} as a bump. The {\bf story point} is lost
  after burning.
\stopdescription

\startdescription{{\bf Catch-Up}}
  When you cross a {\bf mastery} threshold you can increase lesser used
  {\bf abilities} to ensure they keep pace.
\stopdescription

\startdescription{{\bf Climax}}
  A {\bf long contest} {\bf story obstacle} that provides the conclusion
  to a story.
\stopdescription

\startdescription{{\bf Contact}}
  A {\bf supporting character} who shares an {\bf occupation} or
  interest with your character.
\stopdescription

\startdescription{{\bf Contest}}
  Where there is uncertainty as to whether a PC can overcome a
  {\bf story obstacle} or discover a secret, then your GM can call for a
  contest to determine if the PC succeeds or fails. A contest may be
  {\bf simple} (one roll) of {\bf long} (a series of rolls).
\stopdescription

\startdescription{{\bf Consequences}}
  Long term negative modifier, because you lost a contest. Usually a
  {\bf state of adversity}.
\stopdescription

\startdescription{{\bf Contest Framing}}
  Setting the stakes of the {\bf contest}, what is this conflict about.
  Often not the immediate aftermath of victory.
\stopdescription

\startdescription{{\bf Credibility Test}}
  Is it possible to perform the action without an {\bf ability}, with an
  ordinary {\bf ability}, or only with a {\bf extraordinary ability}?
\stopdescription

\startdescription{{\bf Crisis Test}}
  Used to determine if a {\bf resource} that has a {\bf penalty} creates
  a crisis.
\stopdescription

\startdescription{{\bf Defeat}}
  Your {\bf result} is worse than the {\bf resistance's} result.
\stopdescription

\startdescription{{\bf Defensive Response}}
  In a {\bf scored contest} you can choose a defensive {\bf tactic}
  which reduces the {\bf resource points} you lose on a negative
  {\bf result}.
\stopdescription

\startdescription{{\bf Dependent}}
  A {\bf supporting character} who depends on your PC.
\stopdescription

\startdescription{{\bf Depletion}}
  Use of a community {\bf resource} leads to its depletion.
\stopdescription

\startdescription{{\bf Directed Improvement}}
  When your GM grants you a new {\bf ability}, or an increase to an
  existing one, to recognize a story event.
\stopdescription

\startdescription{{\bf Distinguishing Characteristic}}
  The dominant personality {\bf ability} that others recognize in a
  character.
\stopdescription

\startdescription{{\bf Dying}}
  A {\bf state of adversity}, where the character's {\bf defeat} will
  end their participation.
\stopdescription

\startdescription{{\bf Edge}}
  In an {\bf extended contest} adds to the {\bf APs} lost or transferred
  when you win an {\bf exchange}.
\stopdescription

\startdescription{{\bf Exchange}}
  In an {\bf extended contest} a round is divided into two
  {\bf exchanges} where both aggressor and defender act. In a {\bf group
  extended contest} a round consists of a sequence of {\bf exchanges}
  where everyone acts in turn. The GM determines the order of action.
\stopdescription

\startdescription{{\bf Extended Contest}}
  A type of {\bf long contest} in which you track the relative advantage
  one opponent has over another using {\bf advantage points}.
\stopdescription

\startdescription{{\bf Experience Points (XP)}}
  When you experience {\bf defeat}, or a {\bf flaw} you may gain an
  {\bf experience point}, which accumulate between sessions.
\stopdescription

\startdescription{{\bf Extraordinary ability}}
  Certain genres allow player characters to have {\bf abilities} that
  exceed human norms, these are {\bf extraordinary abilities}. A genre
  pack normally outlines what is possible as part of its extraordinary
  powers framework.
\stopdescription

\startdescription{{\bf Failure}}
  Rolling over your {\bf target number}. It can be a {\bf fumble} or
  just a plain {\bf failure}.
\stopdescription

\startdescription{{\bf Final Action}}
  A last action by a {\bf dying} character
\stopdescription

\startdescription{{\bf Flaw}}
  An {\bf ability} that penalizes you instead of helping you.
\stopdescription

\startdescription{{\bf Fumble}}
  The worst {\bf failure} {\bf result}, a notable failure either due to
  incompetence or bad luck.
\stopdescription

\startdescription{{\bf Follower}}
  A {\bf supporting character} under your control. Either a
  {\bf sidekick} or {\bf retainer}
\stopdescription

\startdescription{{\bf Framing the contest}}
  You and your GM agree on the {\bf prize} for the victor, and your
  tactic in trying to win it.
\stopdescription

\startdescription{{\bf Group Chained Contest}}
  A {\bf chained contest} in which more than a pair of opponents contend
  for the {\bf prize}
\stopdescription

\startdescription{{\bf Group Extended Contest}}
  An {\bf extended contest} in which more than a pair of opponents
  contend for the {\bf prize}
\stopdescription

\startdescription{{\bf Group Scored Contest}}
  A {\bf scored contest} in which more than a pair of opponents contend
  for the {\bf prize}
\stopdescription

\startdescription{{\bf Group Simple Contest}}
  A {\bf simple contest} where one side has multiple participants.
\stopdescription

\startdescription{{\bf Graduated Goals}}
  When a contestant has a {\bf primary} and {\bf secondary} goal, and
  may have to choose between them if you have the same result as your
  opponent but a better roll.
\stopdescription

\startdescription{{\bf Handicap}}
  In an {\bf extended contest} subtracts from the {\bf APs} lost or
  transferred when you win an {\bf exchange}.
\stopdescription

\startdescription{{\bf Story Point}}
  Allows you to alter fate for a player character, either by a
  {\bf bump} to their {\bf result} or a {\bf plot edit}.
\stopdescription

\startdescription{{\bf Hurt}}
  A state of adversity, a flesh wound or injured pride, heals at the end
  of a session.
\stopdescription

\startdescription{{\bf Keyword}}
  A single {\bf ability} that encompasses a range of abilities within
  it, such as an {\bf occupation} or culture. An {\bf ability} within an
  {\bf umbrella keyword} is a {\bf break-out ability}, an {\bf ability}
  within a {\bf package keyword} is a {\bf stand-alone ability}.
\stopdescription

\startdescription{{\bf Long Contest}}
  A {\bf contest} where we drill-down to the individual exchanges that
  resolve the conflict. We support {\bf scored}, {\bf extended}, and
  {\bf chained contests}
\stopdescription

\startdescription{{\bf Milestone Advancement}}
  A method for improving a character where the GM declares when you
  receive an {\bf advance}.
\stopdescription

\startdescription{{\bf Modifiers}}
  Adjustments to a {\bf target number} due to circumstance.
\stopdescription

\startdescription{{\bf Mastery}}
  An {\bf ability} {\bf score} that rises above 20 is said to have a
  {\bf mastery}. {\bf Masteries} cancel each other out in
  {\bf contests}. {\bf Masteries} that are not cancelled provide a
  {\bf bump}.
\stopdescription

\startdescription{{\bf Mismatched Goals}}
  When the opposing sides in a {\bf contest} want different
  {\bf prizes}.
\stopdescription

\startdescription{{\bf Occupation}}
  An {\bf ability} that indicates the profession, or primary area of
  expertise, of your character.
\stopdescription

\startdescription{{\bf Outcome}}
  A {\bf contest} has an {\bf outcome}, described as a {\bf victory} or
  {\bf defeat} in obtaining the {\bf prize} that was agreed in
  {\bf contest framing} for any PCs involved.
\stopdescription

\startdescription{{\bf Parting Shot}}
  An attempt to make your opponent's {\bf defeat} worse in a {\bf long
  contest} ({\bf scored} or {\bf extended}), by \quote{finishing them
  off}.
\stopdescription

\startdescription{{\bf Patron}}
  A {\bf supporting character} with superior assets.
\stopdescription

\startdescription{{\bf Penalty}}
  A negative modifier.
\stopdescription

\startdescription{{\bf Prize}}
  What is at stake in the {\bf contest}, decided during {\bf framing}.
\stopdescription

\startdescription{{\bf Rating}}
  An ability has a {\bf rating}, between 1 and 20, indicating how likely
  a character is to succeed at using it.
\stopdescription

\startdescription{{\bf Resistance}}
  The forces opposing the PC in a conflict, or concealing a secret that
  must be overcome by using an {\bf ability} in a {\bf contest}. One of:
  {\bf Extreme}, {\bf Huge}, {\bf Very High}, {\bf High}, {\bf Raised},
  {\bf Moderate}, {\bf Low}, {\bf Very Low}, {\bf Tiny},
  {\bf Rock-bottom}.
\stopdescription

\startdescription{{\bf Resistance Class}}
  The {\bf bonus} or {\bf penalty} to the {\bf resistance} {\bf TN},
  depending on the GM's interpretation of how {\em dramatically} hard
  the {\bf story obstacle} is.
\stopdescription

\startdescription{{\bf Resolution Point (RP)}}
  In a {\bf scored contest} an {\bf RP} tracks the advantage one
  contestant has over the other.
\stopdescription

\startdescription{{\bf Resource}}
  A community {\bf ability} that your PC may draw on.
\stopdescription

\startdescription{{\bf Result}}
  The {\bf outcome} of a die roll against a {\bf TN}. One of
  {\bf critical}, {\bf success}, {\bf failure}, and {\bf fumble}.
\stopdescription

\startdescription{{\bf Retainer}}
  A {\bf follower} of your PC who is not \quote{fleshed out} and cannot
  act independently.
\stopdescription

\startdescription{{\bf Rising Action}}
  A {\bf scored contest} where the {\bf story obstacle} is a step
  towards the final {\bf story obstacle} of this story.
\stopdescription

\startdescription{{\bf Risky Gambit}}
  In a {\bf long contest} you can take an action that puts you at more
  risk on defeat, but enhances victory.
\stopdescription

\startdescription{{\bf Round}}
  A {\bf long contest} is broken into a series of rounds, each of which
  is an attempt to obtain the {\bf prize}. In an {\bf extended contest}
  a round is further broken into a number of {\bf exchanges} in which
  all participants have the chance to act.
\stopdescription

\startdescription{{\bf Score}}
  A {\bf score} consists of a {\bf rating} and, if it is above 20, one
  or more {\bf masteries}
\stopdescription

\startdescription{{\bf Second Chance}}
  An attempt by {\bf defeated}, but unengaged, PCs to re-enter an
  {\bf extended contest}.
\stopdescription

\startdescription{{\bf Scored Contest}}
  A {\bf long contest} where we track the relative advantage one
  contestant has over another using {\bf resolution points}
\stopdescription

\startdescription{{\bf Sidekick}}
  A fleshed out {\bf follower} of your PC who can act independently.
\stopdescription

\startdescription{{\bf Supporting Characters}}
  Additional characters under the player's control that play a
  supporting role to their PC.
\stopdescription

\startdescription{{\bf Simple Contest}}
  A one roll resolution method, the default {\bf contest} type, used
  when learning the {\bf outcome} matters more than the breakdown of how
  you achieved it.
\stopdescription

\startdescription{{\bf Stand Alone Ability}}
  An {\bf ability} raised separately to a {\bf keyword}. It may have
  been added to the character as part of a {\bf package keyword}, or on
  its own.
\stopdescription

\startdescription{{\bf Story Obstacle}}
  Something that prevents you from getting what you want, the
  {\bf prize}. A {\bf story obstacle} is the trigger for a
  {\bf contest}.
\stopdescription

\startdescription{{\bf Stretch}}
  A {\bf penalty} applied to an {\bf ability} because it is stretches
  credibility that it is a reasonable {\bf tactic}.
\stopdescription

\startdescription{{\bf Success}}
  Rolling under your {\bf target number}. It can be a {\bf critical} or
  just a plain {\bf success}.
\stopdescription

\startdescription{{\bf Tactic}}
  How you intend to use one of your {\bf abilities} to overcome a
  {\bf story obstacle}
\stopdescription

\startdescription{{\bf Target Number (TN)}}
  The number, either an {\bf ability} {\bf rating}, or a
  {\bf resistance}, to roll under or equal to in order to {\bf succeed}.
\stopdescription

\startdescription{{\bf TN}}
  Abbreviation for {\bf Target Number}
\stopdescription

\startdescription{{\bf Unrelated Action}}
  An action when you are disengaged in a {\bf long contest} that does
  not relate to your attempt to win the {\bf prize}.
\stopdescription

\startdescription{{\bf Victory}}
  Your {\bf result} is a better roll than the {\bf resistance}.
\stopdescription

\section[title={Version Changes},reference={version-changes}]

\subsection[title={Version 3.0},reference={version-3.0}]

These are the major changes in this version of the rules

\startitemize[packed]
\item
  Split hero points into story points (bumps) and experience points
  (character improvement). Flaws generate experience points as do
  failures.
\item
  Moved the Degree of Victory to an Appendix. We now recommend that the
  GM just uses victory and defeat and adjudicates a suitable bonus or
  penalty if needed.
\item
  Added ranks to codify the +3, +6, +9, \ldots{} progression used
  throughout.
\item
  For Degree of Victory, clarified that contest results are only
  reciprocal between PCs. When the contest is against a resistance set
  by the GM, the results indicate whether the PC gains the prize, and
  the GM narrates the result for the resistance based on this.
\item
  Rephrased the Degree of Victory outcomes to emphasize: Yes, No,
  And\ldots{}, But\ldots{}, This change is designed to dissuade GMs from
  misunderstanding that the prize is obtained on a marginal victory, one
  of the most common result types, and instead encourage GMs to allow
  PCs to fail forward on such a result by introducing downstream
  complications.
\item
  Provided clarity that consequences of defeat and benefit of victory
  are optional and the GM should focus on using the prize to narrate the
  outcome of a contest, only applying mechanical benefits if they make
  sense.
\item
  For use with Degrees of Victory, added States of Fortune to mirror
  States of Adversity. Overall mirrored benefits and consequences more
  closely
\item
  Specific Ability Bonuses are dropped. They were hard for the GM to
  adjudicate and the same intent is better served by using a stretch on
  a broad ability when contesting against a PC with a more specific
  ability.
\item
  A winning group in a Group Simple Contest does not suffer a
  Consequence of Defeat as a result of a low RP difference victory any
  more, the GM should narrate consequences from the level of victory, if
  appropriate.
\item
  Dropped the negative consequences for the winner in an Extended
  Contest during the Rising Action. If the winner is a PC the different
  results suggest additional consequences. So this rule is an
  over-complication.
\item
  Made it clear that only a PC should use a parting shot, not the
  resistance.
\item
  Long contests include both extended contest and scored contests.
  Between version 1 and version 2 extended contests switched to scored
  contests, this approach restores both variants, but requires changing
  the generic name to a long contest.
\item
  Dropped edges and handicaps from extended contests - we use a
  resistance not stats, so makes no sense to have edges and handicaps
\item
  Added alternate mechanisms for determining if resistance advances and
  when
\item
  Added story-based resistance mechanics
\item
  Added story-based improvements
\item
  Added Mythic Russia's Plot Edits
\item
  Added Mythic Russia's Pyrrhic Victories for Extended Contests but as
  Climatic Contests
\item
  Changed degree of success and failure, to degree of victory and
  defeat, as success and failure are for individual rolls, victory and
  defeat once compared.
\item
  Simplified how multiple opponents are handled
\item
  Clarified contest outcomes for long contests, and how to determine the
  overall winner in a long contest
\item
  Do not allow transfers in an extended contest where the abilities
  differ by 6 or more. Consistent with benefits of victory and prevents
  \quote{loading up on mooks} as a strategy.
\stopitemize

\section[title={Outcomes},reference={outcomes}]

Prior versions of the {\em QuestWorlds} rules determined a {\bf degree
of victory} by comparing PC and {\bf resistance} {\bf results}. We now
recommend just having a {\bf victory} or {\bf defeat} an narrating from
the individual {\bf results} as faster and simpler in play. But for
those who prefer the older approach, or want to maintain compatibility
with it, we present those rules in this appendix.

\subsection[title={Degree of Victory or
Defeat},reference={degree-of-victory-or-defeat}]

Often all you need to know to interpret the {\bf outcome} of a
resolution is whether you gained {\bf victory} or suffered a
{\bf defeat}.

Sometimes, you'll want to know how great a {\bf victory} you won, or how
bad a {\bf defeat} you endured. This may be important in providing
{\bf consequences or benefits} that drive further story.

All of the resolution methods have an option to yield the {\bf Degree of
Victor or Defeat} for the PC. The possible {\bf Degree of Victory or
Defeat}, from least to greatest, are: {\bf marginal}, {\bf minor},
{\bf major}, {\bf complete}. {\bf Ties} are also possible.

If you struggle against NPCs or abstract forces, the interpretation of
the {\bf outcome} reveals whether you overcome the {\bf story obstacle},
and any {\bf consequences or benefits}; your GM narrates the fate of the
NPCs or other forces depending on what makes sense. However, when you
and another PC engage in a {\bf contest} then a {\bf victory} for one
contestant means a corresponding {\bf defeat} for the loser.

So whilst in a PC vs.~PC duel the PC would only be killed on a
{\bf complete defeat}, an NPC, described as a {\bf resistance}, might be
killed on any {\bf victory}, depending on how the {\bf contest} was
framed.

{\bf Tie}: Tie means no {\bf outcome}. Effort was expended, but the net
{\bf outcome} is that nothing consequential occurs, or else both sides
lose or gain equally. If this is confusing, and you are not contending
with another PC, your GM can rule that you gain a {\bf marginal
victory}.

{\bf Marginal Victory}: Yes, but\ldots{} You get what you want, the
{\bf prize}, but there are complications, the effect is more limited
than you desired, or you have to make a hard choice between benefits or
accept a loss to get one

{\bf Minor Victory}: Yes\ldots{} You get exactly what they want
i.e.~whatever was the {\bf prize} in the {\bf contest}.

{\bf Major & Complete Victory}: Yes, and\ldots{} You get the
{\bf prize}, and something else. You gain something, stealing a
possession, gaining a new {\bf follower}, or become renowned in song. If
you want to distinguish a {\bf complete} the effect is often permanent
and no new {\bf contests} should be framed for this {\bf story
obstacle}.

{\bf Marginal Defeat}: No, but\ldots{} You don't get what you want, you
lose the {\bf prize}, but it's not a total loss. You are able to salvage
something from the {\bf defeat}, a little more if you sacrifice
something other than the {\bf prize} to your opponent, that the opponent
agrees to take instead.

{\bf Minor Defeat}: No\ldots{} You don't get what you want, you lose the
{\bf prize}. Any consequences or complications such as injury or loss of
influence are short term and easily shrugged off. Just take the loss and
rest up.

{\bf Major & Complete Defeat}: No and\ldots{} You don't get what you
want, you lose the {\bf prize}, and there are long-term consequences.
The situation might grow worse or more complicated or you might suffer
adverse consequences that will require other conflicts to resolve: an
injury that needs a healer, an insult that requires a formal apology, a
loss of influence with the community that requires a triumph to win
their trust again etc. You might be dead, or as good as. The {\bf prize}
is likely lost to you permanently. Or perhaps you lose something, an
item is taken from you, a {\bf follower} deserts you, your reputation
lies in ruins as poets mock your defeat. If you want to distinguish, a
{\bf complete} should be bigger loss than a {\bf major}, but you can
often ignore this distinction.

Your GM will use the {\bf degree of success} to determine any
{\bf benefits and consequences}, but be sure to describe the
{\bf success} in narrative terms.

If you are using a {\bf stretch}, see §2.4.1, then {\bf major or
complete victories} you obtain are instead treated as {\bf minor
victories}.

\subsection[title={Simple Contest},reference={simple-contest-2}]

In a {\bf simple contest}, using the table below (§9.2.1.2) to determine
the {\bf degree of victory or defeat} in the {\bf outcome}

\startplacetable[title={10.3.2.1 SIMPLE CONTEST TABLE}]
\startxtable
\startxtablehead[head]
\startxrow
\startxcell[align=middle,width={0.12\textwidth}] Roll \stopxcell
\startxcell[align=middle,width={0.21\textwidth}] Critical \stopxcell
\startxcell[align=middle,width={0.22\textwidth}] Success \stopxcell
\startxcell[align=middle,width={0.22\textwidth}] Failure \stopxcell
\startxcell[align=middle,width={0.22\textwidth}] Fumble \stopxcell
\stopxrow
\stopxtablehead
\startxtablebody[body]
\startxrow
\startxcell[align=middle,width={0.12\textwidth}] Critical \stopxcell
\startxcell[align=middle,width={0.21\textwidth}] Better roll = Marginal
Victory, else tie \stopxcell
\startxcell[align=middle,width={0.22\textwidth}] Minor
Victory \stopxcell
\startxcell[align=middle,width={0.22\textwidth}] Major
Victory \stopxcell
\startxcell[align=middle,width={0.22\textwidth}] Complete
Victory \stopxcell
\stopxrow
\startxrow
\startxcell[align=middle,width={0.12\textwidth}] Success \stopxcell
\startxcell[align=middle,width={0.21\textwidth}] Minor
Victory \stopxcell
\startxcell[align=middle,width={0.22\textwidth}] Better roll = Marginal
Victory, else tie \stopxcell
\startxcell[align=middle,width={0.22\textwidth}] Minor
Victory \stopxcell
\startxcell[align=middle,width={0.22\textwidth}] Major
Victory \stopxcell
\stopxrow
\startxrow
\startxcell[align=middle,width={0.12\textwidth}] Failure \stopxcell
\startxcell[align=middle,width={0.21\textwidth}] Major
Victory \stopxcell
\startxcell[align=middle,width={0.22\textwidth}] Minor
Victory \stopxcell
\startxcell[align=middle,width={0.22\textwidth}] Better roll = Marginal
Victory, else tie \stopxcell
\startxcell[align=middle,width={0.22\textwidth}] Minor
Victory \stopxcell
\stopxrow
\stopxtablebody
\startxtablefoot[foot]
\startxrow
\startxcell[align=middle,width={0.12\textwidth}] Fumble \stopxcell
\startxcell[align=middle,width={0.21\textwidth}] Complete
Victory \stopxcell
\startxcell[align=middle,width={0.22\textwidth}] Major
Victory \stopxcell
\startxcell[align=middle,width={0.22\textwidth}] Minor
Victory \stopxcell
\startxcell[align=middle,width={0.22\textwidth}] Tie* \stopxcell
\stopxrow
\stopxtablefoot
\stopxtable
\stopplacetable

* In a {\bf group simple contest} (see below), your GM may declare that
both contestants suffer a {\bf marginal defeat} to indicate that,
although their {\bf results} cancel out with respect to each other,
their situation worsens compared to other contestants.

\subsection[title={Group Simple
Contest},reference={group-simple-contest-1}]

In a {\bf group simple contest} rather than overall {\bf victory} going
to the side with the plurality of {\bf victories}, each side scores a
number of {\bf outcome points} (OPs) for their side on a {\bf victory}.
The number of {\bf RPs} is determined by the table below (see §9.2.3.1).
After all the exchange have been concluded, your GM uses the the
difference in {\bf outcome points} between the two groups and table
§9.2.3.2 to determine the {\bf degree of victory or defeat}

Depending on which approach seems to grow from the story, your GM may
assign {\bf consequences} or {\bf benefits} from {\bf group simple
contests} to the entire group, or to individual members who performed
either especially well, or especially poorly. Your GM should default to
rewarding or penalizing everyone. Your GM should resort to
individualized repercussions or benefits only when a group reward defies
dramatic credibility, or when competition within the group is a pivotal
dramatic issue.

\startplacetable[title={10.3.3.1 GROUP SIMPLE CONTEST TABLE}]
\startxtable
\startxtablehead[head]
\startxrow
\startxcell[align=middle]  \stopxcell
\startxcell[align=middle] Critical \stopxcell
\startxcell[align=middle] Success \stopxcell
\startxcell[align=middle] Failure \stopxcell
\startxcell[align=middle] Fumble \stopxcell
\stopxrow
\stopxtablehead
\startxtablebody[body]
\startxrow
\startxcell[align=middle] Critical \stopxcell
\startxcell[align=middle] 1 \stopxcell
\startxcell[align=middle] 2 \stopxcell
\startxcell[align=middle] 3 \stopxcell
\startxcell[align=middle] 5 \stopxcell
\stopxrow
\startxrow
\startxcell[align=middle] Success \stopxcell
\startxcell[align=middle] 2 \stopxcell
\startxcell[align=middle] 1 \stopxcell
\startxcell[align=middle] 2 \stopxcell
\startxcell[align=middle] 3 \stopxcell
\stopxrow
\startxrow
\startxcell[align=middle] Failure \stopxcell
\startxcell[align=middle] 3 \stopxcell
\startxcell[align=middle] 2 \stopxcell
\startxcell[align=middle] 1 \stopxcell
\startxcell[align=middle] 2 \stopxcell
\stopxrow
\stopxtablebody
\startxtablefoot[foot]
\startxrow
\startxcell[align=middle] Fumble \stopxcell
\startxcell[align=middle] 5 \stopxcell
\startxcell[align=middle] 3 \stopxcell
\startxcell[align=middle] 2 \stopxcell
\startxcell[align=middle] 0 \stopxcell
\stopxrow
\stopxtablefoot
\stopxtable
\stopplacetable

\startplacetable[title={10.3.3.2 DEGREE OF VICTORY TABLE}]
\startxtable
\startxtablehead[head]
\startxrow
\startxcell[align=middle] Difference Between OPs \stopxcell
\startxcell[align=middle] Winning Group's Degree of Victory \stopxcell
\stopxrow
\stopxtablehead
\startxtablebody[body]
\startxrow
\startxcell[align=middle] 1 \stopxcell
\startxcell[align=middle] Marginal \stopxcell
\stopxrow
\startxrow
\startxcell[align=middle] 2 \stopxcell
\startxcell[align=middle] Minor \stopxcell
\stopxrow
\startxrow
\startxcell[align=middle] 3--4 \stopxcell
\startxcell[align=middle] Major \stopxcell
\stopxrow
\stopxtablebody
\startxtablefoot[foot]
\startxrow
\startxcell[align=middle] 5+ \stopxcell
\startxcell[align=middle] Complete \stopxcell
\stopxrow
\stopxtablefoot
\stopxtable
\stopplacetable

\subsection[title={Boosting Outcomes},reference={boosting-outcomes}]

Because they average together the {\bf outcomes} of multiple
participants, {\bf group simple contests} tend to flatten
{\bf outcomes}, making {\bf victories} more likely to be {\bf marginal}
or {\bf minor} than {\bf major} or {\bf complete}.

To overcome this flattening effect, if the outcome of a {\bf group
simple contest} is a {\bf tie} or {\bf victory}, you may spend one or
more {\bf story points} to purchase a {\bf boost}; a {\bf boost} assures
a clearer victory.

The cost varies by the number of PCs participating:

\startitemize[packed]
\item
  1 {\bf story point} for 1--3 PCs.
\item
  2 {\bf story points} for 4--6 PCs.
\item
  3 {\bf story points} for 7--9 PCs.\crlf
\item
  and so on\ldots{}
\stopitemize

You may spend twice as many {\bf story points} as required to gain a
{\bf double boost}. The points may be spent by any combination of
players. They remain spent no matter how the {\bf contest} resolves. You
may continue to spend {\bf story points} to {\bf bump} your individual
{\bf result}.

The {\bf boost} increases the collective {\bf victory} level by one
step. A {\bf double boost} increases it by two steps.

\subsection[title={Scored Contest},reference={scored-contest-1}]

In a {\bf scored contest} you compare the difference between the
winner's {\bf resolution points} and the loser's {\bf resolution points}
to determine the {\bf outcome}. You use one of two tables {\bf rising
action} (see §9.2.) or {\bf climax} depending on the dramatic arc of
your story.

\subsubsection[title={Rising Action},reference={rising-action}]

{\bf Rising action} refers to all of the many plot events and
complications that occur between the beginning and the climax of a
story. During this phase of your GM's story, they will use the
{\bf rising action} consequence table to assess {\bf outcomes}.

Find the difference between you and your opponent's {\bf resolution
point} scores at the {\bf contest}'s conclusion. Your GM then determines
your {\bf outcome} by cross-referencing with the following table to find
your {\bf benefits} or {\bf consequences}.

Note, you may suffer a {\bf state of adversity}, even if you win the
{\bf prize}.

\startplacetable[title={10.3.4.2 RISING ACTION CONTEST TABLE}]
\startxtable
\startxtablehead[head]
\startxrow
\startxcell[align=middle,width={0.21\textwidth}] Difference Between
RPs \stopxcell
\startxcell[align=middle,width={0.30\textwidth}] Negative Consequences
for Loser \stopxcell
\startxcell[align=middle,width={0.30\textwidth}] Consequences/Benefit
for Winner \stopxcell
\startxcell[align=middle,width={0.19\textwidth}] Victory/Defeat
Level \stopxcell
\stopxrow
\stopxtablehead
\startxtablebody[body]
\startxrow
\startxcell[align=middle,width={0.21\textwidth}] 1 \stopxcell
\startxcell[align=middle,width={0.30\textwidth}] Hurt \stopxcell
\startxcell[align=middle,width={0.30\textwidth}] Hurt \stopxcell
\startxcell[align=middle,width={0.19\textwidth}] Marginal \stopxcell
\stopxrow
\startxrow
\startxcell[align=middle,width={0.21\textwidth}] 2 \stopxcell
\startxcell[align=middle,width={0.30\textwidth}] Hurt \stopxcell
\startxcell[align=middle,width={0.30\textwidth}] Fresh \stopxcell
\startxcell[align=middle,width={0.19\textwidth}] Marginal \stopxcell
\stopxrow
\startxrow
\startxcell[align=middle,width={0.21\textwidth}] 3 \stopxcell
\startxcell[align=middle,width={0.30\textwidth}] Impaired \stopxcell
\startxcell[align=middle,width={0.30\textwidth}] Pumped \stopxcell
\startxcell[align=middle,width={0.19\textwidth}] Minor \stopxcell
\stopxrow
\startxrow
\startxcell[align=middle,width={0.21\textwidth}] 4 \stopxcell
\startxcell[align=middle,width={0.30\textwidth}] Impaired \stopxcell
\startxcell[align=middle,width={0.30\textwidth}] Pumped \stopxcell
\startxcell[align=middle,width={0.19\textwidth}] Minor \stopxcell
\stopxrow
\startxrow
\startxcell[align=middle,width={0.21\textwidth}] 5 \stopxcell
\startxcell[align=middle,width={0.30\textwidth}] Injured \stopxcell
\startxcell[align=middle,width={0.30\textwidth}] Invigorated \stopxcell
\startxcell[align=middle,width={0.19\textwidth}] Major \stopxcell
\stopxrow
\startxrow
\startxcell[align=middle,width={0.21\textwidth}] 6 \stopxcell
\startxcell[align=middle,width={0.30\textwidth}] Injured \stopxcell
\startxcell[align=middle,width={0.30\textwidth}] Invigorated \stopxcell
\startxcell[align=middle,width={0.19\textwidth}] Major \stopxcell
\stopxrow
\startxrow
\startxcell[align=middle,width={0.21\textwidth}] 7 \stopxcell
\startxcell[align=middle,width={0.30\textwidth}] Dying \stopxcell
\startxcell[align=middle,width={0.30\textwidth}] Heroic \stopxcell
\startxcell[align=middle,width={0.19\textwidth}] Complete \stopxcell
\stopxrow
\startxrow
\startxcell[align=middle,width={0.21\textwidth}] 8 \stopxcell
\startxcell[align=middle,width={0.30\textwidth}] Dead \stopxcell
\startxcell[align=middle,width={0.30\textwidth}] Heroic \stopxcell
\startxcell[align=middle,width={0.19\textwidth}] Complete \stopxcell
\stopxrow
\stopxtablebody
\startxtablefoot[foot]
\startxrow
\startxcell[align=middle,width={0.21\textwidth}] 9 \stopxcell
\startxcell[align=middle,width={0.30\textwidth}] Dead \stopxcell
\startxcell[align=middle,width={0.30\textwidth}] Heroic \stopxcell
\startxcell[align=middle,width={0.19\textwidth}] Complete \stopxcell
\stopxrow
\stopxtablefoot
\stopxtable
\stopplacetable

\subsubsection[title={Climax},reference={climax}]

For the final, climactic confrontation that wraps up your GM's story,
you may suffer a {\bf state of adversity}, even if the {\bf outcome}
shows that you won the {\bf prize}. This represents that at the
{\bf climax} you may triumph, but be laid low by the effort.

First, determine your {\bf outcome} for the {\bf contest} as for rising
action, but in addition, if the outcome show that you gained the
{\bf prize} your GM now cross-references the {\bf resolution points}
scored against you by your opponent on the {\bf climactic state of
adversity} table to determine the {\bf state of adversity} you suffered
in winning that {\bf victory}. If you lost the {\bf prize} use the
{\bf RPs} scored against you to determine your {\bf outcome} as per the
{\bf rising action} table above.

\startplacetable[title={10.3.4.4 CLIMACTIC STATE OF ADVERSITY TABLE}]
\startxtable
\startxtablehead[head]
\startxrow
\startxcell[align=middle] Total Resolution Points Scored Against
PC \stopxcell
\startxcell[align=middle] State of Adversity \stopxcell
\stopxrow
\stopxtablehead
\startxtablebody[body]
\startxrow
\startxcell[align=middle] 0 \stopxcell
\startxcell[align=middle] Unharmed \stopxcell
\stopxrow
\startxrow
\startxcell[align=middle] 1 \stopxcell
\startxcell[align=middle] Dazed \stopxcell
\stopxrow
\startxrow
\startxcell[align=middle] 2 \stopxcell
\startxcell[align=middle] Hurt \stopxcell
\stopxrow
\startxrow
\startxcell[align=middle] 3 \stopxcell
\startxcell[align=middle] Hurt \stopxcell
\stopxrow
\startxrow
\startxcell[align=middle] 4 \stopxcell
\startxcell[align=middle] Impaired \stopxcell
\stopxrow
\startxrow
\startxcell[align=middle] 5 \stopxcell
\startxcell[align=middle] Impaired \stopxcell
\stopxrow
\startxrow
\startxcell[align=middle] 6 \stopxcell
\startxcell[align=middle] Injured \stopxcell
\stopxrow
\startxrow
\startxcell[align=middle] 7 \stopxcell
\startxcell[align=middle] Injured \stopxcell
\stopxrow
\startxrow
\startxcell[align=middle] 8 \stopxcell
\startxcell[align=middle] Dying \stopxcell
\stopxrow
\stopxtablebody
\startxtablefoot[foot]
\startxrow
\startxcell[align=middle] 9 \stopxcell
\startxcell[align=middle] Dead \stopxcell
\stopxrow
\stopxtablefoot
\stopxtable
\stopplacetable

\subsection[title={Group Scored Contest
Outcomes},reference={group-scored-contest-outcomes-1}]

In a {\bf group scored contest} the side that has the last undefeated
contestant gains the {\bf prize}.

If the PCs won, determine the group's overall {\bf outcome} by using the
second-best {\bf outcome} obtained by the PCs, or if there is only one
opponent, the {\bf outcome}. If the PCs lost, determine the group's
overall {\bf outcome} by using the second-worst outcome obtained by the
PCs, or if there is only one PC, the {\bf outcome}.

{\em For example, your PC Lieutenant Jackson of the Royal Navy has led a
shore-action against a French outpost. Lieutenant Jackson and two other
PCs have {\bf victory} {\bf outcomes} at the end of the contest, so the
Royal Navy wins the day. To determine how well the Royal Navy has done
your GM looks at the three {\bf victorious} {\bf outcomes} for the Royal
Navy, a {\bf major victory}, a {\bf minor victory} and a {\bf marginal
victory}. The second best outcome is a {\bf minor victory} so your GM
declares that the Royal Navy have a {\bf minor victory} and have overrun
the French outpost, but gained little else.}

{\em Later you lead your men in a spirited defense against a French
boarding action of your ship. However, the French win the day, with
Lieutenant Jackson and the other PCs suffering {\bf defeat}
{\bf outcomes} at the end of the {\bf contest}. Looking at your PCs
{\bf outcomes} there is a {\bf major defeat}, two {\bf minor defeats}
and a {\bf marginal defeat}. The French win the day with a {\bf minor
defeat} for your Royal Navy crew.}

To determine individual {\bf consequences} or {\bf benefits}, in
{\bf rising action}, even if you engage multiple opponents in a
{\bf rising action scored contest}, only use the last opponent you
engaged to determine your individual {\bf outcome}. In a climatic
contest total the {\bf resolution points} scored against you by all your
opponents. If you engage more than one opponent, be sure to add the
{\bf resolution points} scored against you by all of them. If you lost,
add 1 to your total. Your GM cross-references the total against the
{\bf climactic state of adversity} table.

\subsection[title={Extended Contest
Outcomes},reference={extended-contest-outcomes-1}]

At the end of the contest the {\bf APs} of the loser determine the
{\bf benefits} for the winner or {\bf consequences} for the loser. As
with all {\bf contests}, if the contest involved a {\bf resistance}, and
not another PC, we care about your {\bf outcome}, win or lose, and the
GM should feel free to narrate the {\bf outcome} for the
{\bf resistance} depending on their interpretation of your
{\bf outcome}, which may not be symmetrical. For example, if the
{\bf benefit of victory} for your PC is {\bf pumped} the GM should feel
free to interpret what this means for the {\bf resistance}: in a melee
they might be dead, in a social contest they might be exiled, or they
might surrender in the melee or cede ground in a social contest. Your GM
should focus on the {\bf prize} that was agreed when deciding how to
narrate the resolution of the contest.

\startplacetable[title={10.3.6.1 EXTENDED CONTEST TABLE}]
\startxtable
\startxtablehead[head]
\startxrow
\startxcell[align=middle] Final AP Total \stopxcell
\startxcell[align=middle] Level of Defeat \stopxcell
\startxcell[align=middle] Consequence for Loser \stopxcell
\startxcell[align=middle] Benefit for Winner \stopxcell
\stopxrow
\stopxtablehead
\startxtablebody[body]
\startxrow
\startxcell[align=middle] 0 to --10 AP \stopxcell
\startxcell[align=middle] Marginal \stopxcell
\startxcell[align=middle] Hurt \stopxcell
\startxcell[align=middle] Fresh \stopxcell
\stopxrow
\startxrow
\startxcell[align=middle] --11 to --20 AP \stopxcell
\startxcell[align=middle] Minor \stopxcell
\startxcell[align=middle] Impaired \stopxcell
\startxcell[align=middle] Pumped \stopxcell
\stopxrow
\startxrow
\startxcell[align=middle] --21 to --30 AP \stopxcell
\startxcell[align=middle] Major \stopxcell
\startxcell[align=middle] Injured \stopxcell
\startxcell[align=middle] Invigorated \stopxcell
\stopxrow
\stopxtablebody
\startxtablefoot[foot]
\startxrow
\startxcell[align=middle] --31 or fewer AP \stopxcell
\startxcell[align=middle] Complete \stopxcell
\startxcell[align=middle] Dying \stopxcell
\startxcell[align=middle] Heroic \stopxcell
\stopxrow
\stopxtablefoot
\stopxtable
\stopplacetable

\subsection[title={Group Extended Contest
Outcomes},reference={group-extended-contest-outcomes-1}]

In a {\bf group extended contest} the side that has the last undefeated
contestant gains the {\bf prize}.

If the PCs won, determine the group's overall {\bf outcome} by using the
second-best {\bf outcome} obtained by the PCs, or if there is only one
opponent, the {\bf outcome}. If the PCs lost, determine the group's
overall {\bf outcome} by using the second-worst outcome obtained by the
PCs, or if there is only one PC, the {\bf outcome}.

{\em For example, your PC Lieutenant Jackson of the Royal Navy has led a
shore-action against a French outpost. Lieutenant Jackson and two other
PCs have {\bf victory} {\bf outcomes} at the end of the contest, so the
Royal Navy wins the day. To determine how well the Royal Navy has done
your GM looks at the three {\bf victorious} {\bf outcomes} for the Royal
Navy, a {\bf major victory}, a {\bf minor victory} and a {\bf marginal
victory}. The second best outcome is a {\bf minor victory} so your GM
declares that the Royal Navy have a {\bf minor victory} and have overrun
the French outpost, but gained little else.}

{\em Later you lead your men in a spirited defense against a French
boarding action of your ship. However, the French win the day, with
Lieutenant Jackson and the other PCs suffering {\bf defeat}
{\bf outcomes} at the end of the {\bf contest}. Looking at your PCs
{\bf outcomes} there is a {\bf major defeat}, two {\bf minor defeats}
and a {\bf marginal defeat}. The French win the day with a {\bf minor
defeat} for your Royal Navy crew.}

To determine individual {\bf consequences} or {\bf benefits}, use the
{\bf AP} of last opponent you engaged to determine your individual
{\bf outcome}.

\subsection[title={Chained Contests},reference={chained-contests}]

In a {\bf chained contest} determine the {\bf state of adversity} (see
§9.4) to the loser of an exchange from the following table (see
9.3.8.1).

\startplacetable[title={10.3.8.1 CHAINED CONTEST TABLE}]
\startxtable
\startxtablehead[head]
\startxrow
\startxcell[align=middle,width={0.13\textwidth}] Roll \stopxcell
\startxcell[align=middle,width={0.22\textwidth}] Critical \stopxcell
\startxcell[align=middle,width={0.22\textwidth}] Success \stopxcell
\startxcell[align=middle,width={0.22\textwidth}] Failure \stopxcell
\startxcell[align=middle,width={0.22\textwidth}] Fumble \stopxcell
\stopxrow
\stopxtablehead
\startxtablebody[body]
\startxrow
\startxcell[align=middle,width={0.13\textwidth}] Critical \stopxcell
\startxcell[align=middle,width={0.22\textwidth}] Worse roll is
{\bf hurt}. If tied, no effect. \stopxcell
\startxcell[align=middle,width={0.22\textwidth}] Opponent {\bf hurt}. If
already {\bf hurt} in this contest, Injured. If already injured,
Dying \stopxcell
\startxcell[align=middle,width={0.22\textwidth}] Opponent Injured. If
already Injured in this contest, Dying \stopxcell
\startxcell[align=middle,width={0.22\textwidth}] Opponent Dying: player
has them at complete mercy. Contest is over. \stopxcell
\stopxrow
\startxrow
\startxcell[align=middle,width={0.13\textwidth}] Success \stopxcell
\startxcell[align=middle,width={0.22\textwidth}] PC is {\bf hurt}. If
already {\bf hurt} in this contest, Injured. If already injured,
Dying \stopxcell
\startxcell[align=middle,width={0.22\textwidth}] Worse roll is
{\bf hurt}. If tied, no effect \stopxcell
\startxcell[align=middle,width={0.22\textwidth}] Opponent {\bf hurt}. If
already {\bf hurt} in this contest, Injured. If already injured,
Dying \stopxcell
\startxcell[align=middle,width={0.22\textwidth}] Opponent Injured. If
already Injured in this contest, Dying \stopxcell
\stopxrow
\startxrow
\startxcell[align=middle,width={0.13\textwidth}] Failure \stopxcell
\startxcell[align=middle,width={0.22\textwidth}] PC is Injured. If
already Injured in this contest, Dying \stopxcell
\startxcell[align=middle,width={0.22\textwidth}] PC is {\bf hurt}. If
already {\bf hurt} in contest, Injured. If injured, Dying \stopxcell
\startxcell[align=middle,width={0.22\textwidth}] Worse roll is
{\bf hurt}. If tied, no effect \stopxcell
\startxcell[align=middle,width={0.22\textwidth}] Opponent {\bf hurt}. If
already {\bf hurt} in this contest, Injured. If already injured,
Dying \stopxcell
\stopxrow
\stopxtablebody
\startxtablefoot[foot]
\startxrow
\startxcell[align=middle,width={0.13\textwidth}] Fumble \stopxcell
\startxcell[align=middle,width={0.22\textwidth}] PC Dying: opponent has
them at complete mercy. Contest is over. \stopxcell
\startxcell[align=middle,width={0.22\textwidth}] PC is Injured. If
already Injured in this contest, Dying \stopxcell
\startxcell[align=middle,width={0.22\textwidth}] PC is {\bf hurt}. If
already {\bf hurt} contest, Injured. If already injured,
Dying \stopxcell
\startxcell[align=middle,width={0.22\textwidth}] Both make a mistake. No
effect on contest. Side effects at GM's discretion \stopxcell
\stopxrow
\stopxtablefoot
\stopxtable
\stopplacetable

\subsection[title={Group Chained Contest
Outcomes},reference={group-chained-contest-outcomes-1}]

In a {\bf group chained contest} the side that has the last undefeated
contestant gains the {\bf prize}.

If the PCs won, determine the group's overall {\bf outcome} by using the
second-best {\bf outcome} obtained by the PCs, or if there is only one
opponent, the {\bf outcome}. If the PCs lost, determine the group's
overall {\bf outcome} by using the second-worst outcome obtained by the
PCs, or if there is only one PC, the {\bf outcome}.

{\em For example, your PC Lieutenant Jackson of the Royal Navy has led a
shore-action against a French outpost. Lieutenant Jackson and two other
PCs have {\bf victory} {\bf outcomes} at the end of the contest, so the
Royal Navy wins the day. To determine how well the Royal Navy has done
your GM looks at the three {\bf victorious} {\bf outcomes} for the Royal
Navy, a {\bf major victory}, a {\bf minor victory} and a {\bf marginal
victory}. The second best outcome is a {\bf minor victory} so your GM
declares that the Royal Navy have a {\bf minor victory} and have overrun
the French outpost, but gained little else.}

{\em Later you lead your men in a spirited defense against a French
boarding action of your ship. However, the French win the day, with
Lieutenant Jackson and the other PCs suffering {\bf defeat}
{\bf outcomes} at the end of the {\bf contest}. Looking at your PCs
{\bf outcomes} there is a {\bf major defeat}, two {\bf minor defeats}
and a {\bf marginal defeat}. The French win the day with a {\bf minor
defeat} for your Royal Navy crew.}

Individual {\bf consequences} or {\bf benefits} will have already been
determined by the {\bf chained contest} {\bf outcomes} on each
{\bf round}.

\section[title={Benefits and
Consequences},reference={benefits-and-consequences-1}]

Prior versions of the {\em QuestWorlds} rules used the {\bf degree of
victory or defeat} (see §9.3) to determine consequences. Whilst we now
recommend that your GM determines consequences using the {\bf scale}
from their interpretation of the {\bf outcome}. We find this simpler and
faster in play. But for those who prefer the older approach, or want to
maintain compatibility with it, we present those rules in this appendix.

You should note that these rules allow the possibility that a PC will
end up dying either literally or figuratively and will be removed from
the game without intervention. Under the main rules this to be a player
choice in response to an {\bf outcome} not something that should come
randomly from the dice.

\subsection[title={The Consequences of
Defeat},reference={the-consequences-of-defeat}]

When you lose a {\bf contest}, you may suffer {\bf consequences}:
literal or metaphorical injuries which make it harder for you to use
related {\bf abilities}.

From the least to the most punishing, the five {\bf states of adversity}
are: {\bf hurt}, {\bf impaired}, {\bf injured}, {\bf dying}, and
{\bf dead}. The first four are possible {\bf consequences} of any
{\bf contest}. {\bf Dying} PCs become {\bf dead}, unless they receive
intervention of some sort.

Although the levels refer to physical {\bf states of adversity}, the
consequences can be emotional, social, spiritual, magical, and so on.

\subsubsection[title={Hurt},reference={hurt}]

If you are {\bf hurt}, you show signs of adversity and find it harder to
succeed at {\bf contests} related to your {\bf defeat}. Either your
flesh or pride may be bruised. Until you recover, you suffer a --3
{\bf penalty} to all related {\bf abilities}.

You may suffer multiple {\bf hurts} to the same {\bf ability}. These are
cumulative until recovery occurs.

Unless your GM has a dramatic reason to decide otherwise, your
{\bf hurts} vanish at the end of a session, after one day of rest per
accumulated {\bf hurt}, or when in-game events justify their removal.

\subsubsection[title={Impaired},reference={impaired}]

If you are {\bf impaired}, you have taken a jarring blow, physically,
socially, or emotionally, and you are much likelier to fail when
attempting similar actions in the future. You suffer a --6 {\bf penalty}
to all related {\bf abilities}. Impairments combine with {\bf hurts} and
with other impairments.

As bad as your condition may be, there's nothing wrong with you that
some prolonged inactivity won't fix. A single {\bf impairment} goes away
after one week of rest, or when an in-game event (like miraculous or
extraordinary treatment) occurs to make their removal seem believable.

\subsubsection[title={Injured},reference={injured}]

If you are {\bf injured}, you have suffered a debilitating blow which
leaves you reeling. Physically you may have lost the use of a limb or
sense, socially you may be shunned, and emotionally you may in shock.
Although you should heal with time, you suffer a −9 {\bf penalty} to all
related {\bf abilities}. Injuries combine with impairments and
{\bf hurts}.

A single {\bf injury} goes away after a month's rest, or by miraculous
intervention, as above.

\subsubsection[title={Dying},reference={dying}]

If you are {\bf dying} you will, without rapid and appropriate
intervention, expire. To save you, the other PCs must overcome a
{\bf story obstacle}. Their attempt must be credible, using medicine or
magic, as defined by your genre. Your GM should use a {\bf very high
resistance} for this {\bf contest}, unless the story suggests otherwise.
According to the conventions of dramatic storytelling, the character
typically has just enough time left for the other characters to make
this one attempt.

Successful intervention leaves the PC {\bf injured}. Depending on the
narrative circumstances, a {\bf complete victory} on the intervention
attempt may leave them merely {\bf impaired}.

If intervention fails, you will die, but not necessarily immediately.
Although irrevocably doomed, your GM may rule that the story suggests
that you survive long enough to take one final, heroic, action.

To even take that {\bf final action} if the GM offers you the chance,
you must succeed at a prior {\bf contest of wherewithal} to rouse
yourself to action. Appropriate abilities for the {\bf contest of
wherewithal} include:

\startitemize[packed]
\item
  Physical action: Endurance, High Pain Threshold, Grim Determination,
  etc.
\item
  Intellectual activity: Concentration, Iron Will, Love of Country (if
  action to be attempted is patriotic), etc.
\item
  Social humiliation: Savoir Faire, Unflappable, Stoic Dignity
\stopitemize

A {\bf contest of wherewithal} faces a {\bf moderate resistance}. Even
if you succeed at the {\bf contest of wherewithal}, you take an
automatic {\bf bump} down {\bf penalty} whenever you use any related
{\bf ability} in a {\bf contest}. (The {\bf bump} down does not apply to
the {\bf contest of wherewithal} itself.) Where it seems apt, your GM
may choose to ignore the {\bf bump} down if you score a {\bf major or
complete victory} on the {\bf contest of wherewithal}.)

Any active {\bf hurts} or {\bf impairments} continue to be counted
against you as well.

Your {\bf final action} cannot reverse the {\bf outcome} of the
{\bf contest} that you lost, it must involve a new {\bf story obstacle}.
Your GM will rule if your action is allowable.

Like other {\bf states of adversity}, {\bf dying} may be literal or
metaphorical. Your standing in society, business or politics may be on
the brink of permanent extinction. You may be facing mental death --- a
permanent lapse into madness or senility.

\subsubsection[title={Dead},reference={dead}]

If you die as a consequence of physical injuries, you are gone from the
game, period.

Death from a non-physical {\bf contest} will likely be metaphorical. If
you die in an economic, social, spiritual, or artistic {\bf contest},
you permanently lose abilities.

Even only metaphorically dead, your GM may declare that you have
undergone changes so dire as to make your PC unplayable. You may be
incurably insane, or be so socially shamed that you retire to a life of
obscurity or religious meditation. You may be shunned by all around you,
sent into permanent exile, or sentenced to long-term imprisonment with
no hope of escape.

\startplacetable[title={10.4.1.7 CONSEQUENCES OF DEFEAT TABLE}]
\startxtable
\startxtablehead[head]
\startxrow
\startxcell[align=middle,width={0.12\textwidth}] Defeat Level \stopxcell
\startxcell[align=middle,width={0.19\textwidth}] State of
Adversity \stopxcell
\startxcell[align=middle,width={0.69\textwidth}] Penalty \stopxcell
\stopxrow
\stopxtablehead
\startxtablebody[body]
\startxrow
\startxcell[align=middle,width={0.12\textwidth}] Marginal \stopxcell
\startxcell[align=middle,width={0.19\textwidth}] Hurt \stopxcell
\startxcell[align=middle,width={0.69\textwidth}] --3 penalty to
appropriate abilities \stopxcell
\stopxrow
\startxrow
\startxcell[align=middle,width={0.12\textwidth}] Minor \stopxcell
\startxcell[align=middle,width={0.19\textwidth}] Impaired \stopxcell
\startxcell[align=middle,width={0.69\textwidth}] --6 penalty to
appropriate abilities \stopxcell
\stopxrow
\startxrow
\startxcell[align=middle,width={0.12\textwidth}] Major \stopxcell
\startxcell[align=middle,width={0.19\textwidth}] injured \stopxcell
\startxcell[align=middle,width={0.69\textwidth}] --9 penalty to
appropriate abilities \stopxcell
\stopxrow
\stopxtablebody
\startxtablefoot[foot]
\startxrow
\startxcell[align=middle,width={0.12\textwidth}] Complete \stopxcell
\startxcell[align=middle,width={0.19\textwidth}] Dying \stopxcell
\startxcell[align=middle,width={0.69\textwidth}] No actions allowed. If
\quote{final action}, automatic {\bf bump} down on uses of appropriate
{\bf ability} \stopxcell
\stopxrow
\stopxtablefoot
\stopxtable
\stopplacetable

\subsection[title={Benefits of Victory},reference={benefits-of-victory}]

Just as when you experience {\bf defeat} you can suffer ongoing ill
effects in addition to the loss of the {\bf prize} at hand, when you win
you can gain benefits from that {\bf victory}.

A {\bf benefit of victory} gives you a bonus on the selected
{\bf abilities}, or in the specified situation, as determined by your
{\bf victory} level.

A PC may apply {\bf bonuses} from multiple {\bf benefits} to a single
{\bf contest}.

From the least to the most robust the four {\bf states of fortune} are:
{\bf fresh}, {\bf pumped}, {\bf invigorated}, and {\bf heroic}.

\subsubsection[title={Fresh},reference={fresh}]

If you are {\bf fresh}, you are lively and find it easier to succeed at
{\bf contests} related to your {\bf victory}. You are on a roll and feel
confident and able. Until you are {\bf defeated}, you gain a +3
{\bf bonus} to all related abilities.

You may be refreshed multiple times on the same {\bf ability}. These are
cumulative until {\bf defeat} occurs.

Unless your GM has a dramatic reason to decide otherwise, your
{\bf freshness} vanishes at the end of a session, after one day of
idleness, or when in-game events justify their removal.

\subsubsection[title={Pumped},reference={pumped}]

If you are {\bf pumped}, you are energized, physically, socially, or
emotionally, and you are much likelier to succeed when attempting
similar actions in the future. You gain a +6 {\bf bonus} to all related
abilities. {\bf Pumped} combines with {\bf fresh} and {\bf pumped}.

As good as your condition may be, an extended period of idleness will
cause you to lose your edge. A single {\bf pumped} goes away after one
week of idleness, or when an in-game event (like long drunken party)
occurs to make their removal seem believable.

\subsubsection[title={Invigorated},reference={invigorated}]

If you are {\bf invigorated}, you are pulsing with hormones, mentally
focused, or exuding confidence. Physically you can push your body to new
personal bests of achievement, socially confidant and exuding charisma,
and emotionally you are in touch with your feelings and resonate with
those of others. Although this will fade with time, you gain a +9
{\bf bonus} to all related abilities. {\bf Invigorated} combines with
{\bf pumped} and {\bf fresh}.

Being {\bf invigorated} goes away after a month's idleness, or an
in-game event, as above.

\subsubsection[title={Heroic},reference={heroic}]

If you are {\bf heroic}, you have become unstoppable, physically at peak
performance, socially, everyone wants to be you or be with you, and
emotionally you have gained new insights into yourself and others around
you. Although this will fade with time, you gain a {\bf bump}
{\bf bonus} to all related abilities. Being {\bf heroic} combines with
{\bf invigorated}, {\bf pumped} and {\bf fresh}.

Being {\bf heroic} goes away after a season's idleness, or an in-game
event, as above.

\startplacetable[title={10.4.2.5 BENEFITS OF VICTORY TABLE}]
\startxtable
\startxtablehead[head]
\startxrow
\startxcell[align=middle,width={0.14\textwidth}] Victory
Level \stopxcell
\startxcell[align=middle,width={0.15\textwidth}] State of
Fortune \stopxcell
\startxcell[align=middle,width={0.70\textwidth}] Benefit \stopxcell
\stopxrow
\stopxtablehead
\startxtablebody[body]
\startxrow
\startxcell[align=middle,width={0.14\textwidth}] Marginal \stopxcell
\startxcell[align=middle,width={0.15\textwidth}] Fresh \stopxcell
\startxcell[align=middle,width={0.70\textwidth}] +3 \stopxcell
\stopxrow
\startxrow
\startxcell[align=middle,width={0.14\textwidth}] Minor \stopxcell
\startxcell[align=middle,width={0.15\textwidth}] Pumped \stopxcell
\startxcell[align=middle,width={0.70\textwidth}] +6 \stopxcell
\stopxrow
\startxrow
\startxcell[align=middle,width={0.14\textwidth}] Major \stopxcell
\startxcell[align=middle,width={0.15\textwidth}] Invigorated \stopxcell
\startxcell[align=middle,width={0.70\textwidth}] +9 \stopxcell
\stopxrow
\stopxtablebody
\startxtablefoot[foot]
\startxrow
\startxcell[align=middle,width={0.14\textwidth}] Complete \stopxcell
\startxcell[align=middle,width={0.15\textwidth}] Heroic \stopxcell
\startxcell[align=middle,width={0.70\textwidth}] You receive an
automatic {\bf bump} up on uses of an appropriate
{\bf ability} \stopxcell
\stopxrow
\stopxtablefoot
\stopxtable
\stopplacetable

\subsubsection[title={Clearly Inferior
Opponents},reference={clearly-inferior-opponents}]

Defeating clearly inferior opponents neither teaches you anything nor
significantly enhances your reputation; you are ineligible for a
{\bf benefit of victory} if the {\bf resistance} you used in the
{\bf contest} exceeded the {\bf resistance} by 6 or more. If, in the
case of a {\bf long contest}, you or your opponent used multiple
{\bf abilities}, compare the best {\bf ability} you used to their worst.

\subsection[title={Recovery and
Healing},reference={recovery-and-healing-1}]

Consequences of {\bf injured} or less lapse on their own with the
passage of time. However, you'll often want to remove them ahead of
schedule, with the use of {\bf abilities}.

\subsubsection[title={Healing
Abilities},reference={healing-abilities-1}]

The {\bf ability} used to bring about recovery from a {\bf state of
adversity} must relate to the type of harm.

You can heal physical injuries with medical or extraordinary
{\bf abilities}.

You can remove mental traumas, including those of confidence and morale,
with mundane psychology or through {\bf extraordinary abilities}. You
might also remove them through a dramatic confrontation between the
victim and the source of the psychic injury.

You use social abilities to heal social injuries. You probably have to
make a public apology of some sort, often including a negotiation with
the offended parties and the payment of compensation, either in
disposable wealth or something more symbolic.

You can fix damage to items and equipment with some sort of repair
{\bf ability}. If you want to fix an extraordinary item, you may require
genre-specific expertise: a broken magic ring may require a ritual to
reforge.

Your GM should almost always resolve healing attempts as {\bf simple
contests}. An exception might be a medical drama, in which surgeries
would comprise the suspenseful set-piece sequences of the game, and your
GM might chose a {\bf long contest}.

\subsubsection[title={Healing
Resistances},reference={healing-resistances-1}]

Default {\bf resistances} to remove states of adversity are as follows:

\startplacetable[title={10.4.3.3 HEALING RESISTANCES TABLE}]
\startxtable
\startxtablehead[head]
\startxrow
\startxcell[align=middle] Consequence of Defeat \stopxcell
\startxcell[align=middle] Difficulty \stopxcell
\stopxrow
\stopxtablehead
\startxtablebody[body]
\startxrow
\startxcell[align=middle] Hurt \stopxcell
\startxcell[align=middle] Moderate \stopxcell
\stopxrow
\startxrow
\startxcell[align=middle] Impaired \stopxcell
\startxcell[align=middle] Raised \stopxcell
\stopxrow
\startxrow
\startxcell[align=middle] Injured \stopxcell
\startxcell[align=middle] High \stopxcell
\stopxrow
\stopxtablebody
\startxtablefoot[foot]
\startxrow
\startxcell[align=middle] Dying \stopxcell
\startxcell[align=middle] Very High \stopxcell
\stopxrow
\stopxtablefoot
\stopxtable
\stopplacetable

\subsubsection[title={Outcomes of
Healing},reference={outcomes-of-healing}]

When you make a successful healing attempt, you remove one level of
{\bf adversity} for each level of {\bf victory}. A {\bf major defeat}
increases the subject's {\bf consequences} by 1; a {\bf complete defeat}
adds an additional 2 levels to the {\bf state of adversity}.

\section[title={Augments},reference={augments-1}]

If you are using {\bf degrees of victory or defeat}, use the table below
to interpret the {\bf outcome} of the {\bf simple contest} for an
augment. Note that {\bf penalties} for {\bf defeat} when attempting an
{\bf augment} are much lessened compared to a regular {\bf contest}.

\startplacetable[title={10.5.1 AUGMENT TABLE}]
\startxtable
\startxtablehead[head]
\startxrow
\startxcell[align=middle] Contest Outcome \stopxcell
\startxcell[align=middle] Modifier \stopxcell
\stopxrow
\stopxtablehead
\startxtablebody[body]
\startxrow
\startxcell[align=middle] Complete Victory \stopxcell
\startxcell[align=middle] +M \stopxcell
\stopxrow
\startxrow
\startxcell[align=middle] Major Victory \stopxcell
\startxcell[align=middle] +9 \stopxcell
\stopxrow
\startxrow
\startxcell[align=middle] Minor Victory \stopxcell
\startxcell[align=middle] +6 \stopxcell
\stopxrow
\startxrow
\startxcell[align=middle] Marginal Victory \stopxcell
\startxcell[align=middle] +3 \stopxcell
\stopxrow
\startxrow
\startxcell[align=middle] Marginal Defeat \stopxcell
\startxcell[align=middle] 0 \stopxcell
\stopxrow
\startxrow
\startxcell[align=middle] Minor Defeat \stopxcell
\startxcell[align=middle] 0 \stopxcell
\stopxrow
\startxrow
\startxcell[align=middle] Major Defeat \stopxcell
\startxcell[align=middle] 0 \stopxcell
\stopxrow
\stopxtablebody
\startxtablefoot[foot]
\startxrow
\startxcell[align=middle] Complete Defeat \stopxcell
\startxcell[align=middle] −3 \stopxcell
\stopxrow
\stopxtablefoot
\stopxtable
\stopplacetable

\section[title={Community Rules},reference={community-rules}]

The community rules in earlier versions were complex and have been
simplified in the current rules. They are presented here mainly to help
in understanding older material, or for those who prefer it's more
complex tracking approach.

\subsection[title={Defining Resources},reference={defining-resources-1}]

Communities can have a type of {\bf ability} called a {\bf resource}
that your GM defines. Your PC can try to draw on their community's
{\bf resources} to use them as {\bf abilities}. Your GM should focus on
no more than five or so broadly-labeled {\bf resource} types, so that
the PCs can care about (and have a chance of successfully managing) all
of them.

\subsection[title={Specify an interval},reference={specify-an-interval}]

Your GM chooses a suitable interval to mark changes in {\bf resources}.
For genres bound by the agricultural season, this is usually a season,
for a military genre it might be a campaign, for a ship a voyage.

\subsection[title={Assigning Ability
Scores},reference={assigning-ability-scores-1}]

Your GM distributes the following {\bf scores} between the five
abilities: 12W, 9W, 18, 18, and 12. Note that the size of the group
doesn't affect the {\bf scores}.

\subsection[title={Resource Notation},reference={resource-notation}]

Your GM will keep track of {\bf modifiers} to community {\bf resources}
with a copy of the following record sheet. They will use a pencil,
because the numbers will fluctuate.

Your GM lists the names and {\bf scores} of your chosen {\bf resources}
in the first row. Under the total column for each, your GM will list the
total current modifier. Under the PC column, your GM lists {\bf bonuses}
resulting from PC activities (as opposed to un-cemented {\bf background
events}.) When PCs {\bf cement a background benefit}, your GM adds its
bonus to the PC column.

When PC activity reduces a {\bf penalty} but does not eliminate it, your
GM will alter the entry under the Total column to reflect the reduction,
but leave the PC column blank.

\startplacetable[title={10.6.4.1 RESOURCE NOTATION TABLE}]
\startxtable
\startxtablehead[head]
\startxrow
\startxcell Total \stopxcell
\startxcell PC \stopxcell
\startxcell Total \stopxcell
\startxcell PC \stopxcell
\startxcell Total \stopxcell
\startxcell PC \stopxcell
\startxcell Total \stopxcell
\startxcell PC \stopxcell
\startxcell Total \stopxcell
\startxcell PC \stopxcell
\stopxrow
\stopxtablehead
\startxtablebody[body]
\stopxtablebody
\startxtablefoot[foot]
\startxrow
\startxcell ~ \stopxcell
\startxcell ~ \stopxcell
\startxcell ~ \stopxcell
\startxcell ~ \stopxcell
\startxcell ~ \stopxcell
\startxcell ~ \stopxcell
\startxcell ~ \stopxcell
\startxcell ~ \stopxcell
\startxcell ~ \stopxcell
\startxcell ~ \stopxcell
\stopxrow
\stopxtablefoot
\stopxtable
\stopplacetable

\subsection[title={Drawing on
Resources},reference={drawing-on-resources-1}]

You can use community {\bf resources} as {\bf abilities} after
convincing the community to let you expend precious assets. This
requires a preliminary {\bf contest} using a social {\bf ability}, most
likely your community relationship. Your GM will use a {\bf moderate
resistance} as the baseline, with higher {\bf resistance}s when your
proposals seem selfish or likely to fail, and lower ones when everyone
but the dullest dolt would readily see their collective benefits. Your
GM may increase {\bf resistance}s if your group draws constantly on
community {\bf resources} without replenishing them.

The lobbying effort and the actual resource use require framing, a clear
description of what you are doing, and other details to bring them to
fictional life. You can use {\bf resource abilities} directly, or to
{\bf augment} your own {\bf abilities}.

Unlike character abilities, each use of community {\bf resources}
temporarily {\bf depletes} it.

On a {\bf victory}, you win the {\bf prize} specified by {\bf contest
framing}, and a {\bf penalty} is applied to subsequent uses of the
{\bf resource}.

On a {\bf defeat}, you lose the {\bf prize} and an even more severe
{\bf penalty} is applied to subsequent {\bf resource} uses. If you fail
to secure the {\bf prize} you were seeking, the depletion {\bf penalty}
is also applied to your social and community {\bf abilities} when
interacting with members of your community. This reflects community
displeasure at your fruitless expenditure.

{\bf Penalties} from the Resource Depletion Table replace standard
{\bf penalties} for {\bf defeat}, not add to them.

Like other {\bf modifiers} to {\bf resources}, depletion {\bf penalties}
end at the end of the current interval. These include {\bf depletion
penalties} applied to character {\bf abilities}. However, a
{\bf depletion penalty} left unattended at the end of the interval can
result in a permanent drop in the relevant {\bf resource}.

If your GM wants resource depletion to lead to longer-lasting social
{\bf penalties}, at the cost of some extra bookkeeping, they can have
the characters shed a 3-point {\bf penalty} at the end of each interval.

\startplacetable[title={10.6.5.1 RESOURCE DEPLETION TABLE}]
\startxtable
\startxtablehead[head]
\startxrow
\startxcell[align=middle] Contest Outcome \stopxcell
\startxcell[align=middle] Depletion Penalty \stopxcell
\stopxrow
\stopxtablehead
\startxtablebody[body]
\startxrow
\startxcell[align=middle] Complete Victory \stopxcell
\startxcell[align=middle] 0 \stopxcell
\stopxrow
\startxrow
\startxcell[align=middle] Major Victory \stopxcell
\startxcell[align=middle] −3 \stopxcell
\stopxrow
\startxrow
\startxcell[align=middle] Minor Victory \stopxcell
\startxcell[align=middle] −3 \stopxcell
\stopxrow
\startxrow
\startxcell[align=middle] Marginal Victory \stopxcell
\startxcell[align=middle] −3 \stopxcell
\stopxrow
\startxrow
\startxcell[align=middle] Marginal Defeat \stopxcell
\startxcell[align=middle] −6 \stopxcell
\stopxrow
\startxrow
\startxcell[align=middle] Minor Defeat \stopxcell
\startxcell[align=middle] −6 \stopxcell
\stopxrow
\startxrow
\startxcell[align=middle] Major Defeat \stopxcell
\startxcell[align=middle] −6 \stopxcell
\stopxrow
\stopxtablebody
\startxtablefoot[foot]
\startxrow
\startxcell[align=middle] Complete Defeat \stopxcell
\startxcell[align=middle] −9 \stopxcell
\stopxrow
\stopxtablefoot
\stopxtable
\stopplacetable

\subsection[title={Required Resource
Use},reference={required-resource-use-1}]

As part of your GM's setting design, they may specify that certain
actions in a setting always require the use of a community
{\bf resource}. Because the {\bf resource} use is obligatory, it need
not meet the usual criteria for entertainment value. Also, when the
resource is used as an {\bf augment}, you can also add a second
{\bf augment} from some other {\bf ability}, adjudicated according to
the standard rules, including entertainment value criteria. (This way,
the required {\bf resource} use doesn't penalize you by forcing you to
{\bf augment} with a low-rated {\bf resource} when you could otherwise
use a higher-rated {\bf ability}.)

\subsection[title={Penalties to
Resources},reference={penalties-to-resources}]

Threats to community {\bf resources} act as a spur to PC action. Your GM
may rule that the {\bf penalty} from any {\bf outcome} may be applied to
a {\bf resource}. (It might at the same time be applied to one or more
PC {\bf abilities}.)

When choosing a {\bf penalty} arising from a player {\bf defeat} in a
{\bf simple contest}, your GM will use the {\bf consequences of defeat}
table. For a {\bf group contest}, the {\bf penalty} corresponds to the
second worst {\bf state of adversity} suffered by a defeated group
member.

If your group voluntarily concede a {\bf contest} by withdrawing, your
community suffers {\bf resource depletion} equivalent to a {\bf major
defeat}.

\subsection[title={Bolstering
Resources},reference={bolstering-resources-1}]

You can add {\bf bonuses} to {\bf bolster} community {\bf resources} by
seeking out and overcoming relevant {\bf story obstacles}, specifying in
the {\bf contest framing} that the proceeds of {\bf victory} go the
community. If you succeed, {\bf bonuses} from the {\bf benefits of
victory} table are applied to a resource instead of one or more
character abilities. (Your GM may rule that the bonus also applies to
you in social situations that involve community members, reflecting
gratitude for their efforts on behalf of the community.)

\subsection[title={Background Events},reference={background-events-1}]

Your changes to {\bf resources} take center stage in a series, but in
the background all sorts of other events periodically alter the
community's prosperity. These include the actions of other community
members, who are {\bf depleting and bolstering resources} all the time,
as well as the unexpected intrusion of outside forces.

At the beginning of each interval, one of your group should perform a
{\bf simple contest} of each {\bf resource} against a {\bf resistance}
equal to the average value of all {\bf resources}. These {\bf contests}
simulate {\bf background events} outside of your control or influence;
they can't be {\bf augmented} or {\bf bumped} up with {\bf hero points}.

The {\bf outcome} of the {\bf contest} may apply a {\bf modifier} to a
{\bf resource}, as per the following table:

\startplacetable[title={10.6.9.1 RESOURCE FLUCTUATION TABLE}]
\startxtable
\startxtablehead[head]
\startxrow
\startxcell[align=middle] Outcome \stopxcell
\startxcell[align=middle] Depletion Penalty \stopxcell
\stopxrow
\stopxtablehead
\startxtablebody[body]
\startxrow
\startxcell[align=middle] Complete Victory \stopxcell
\startxcell[align=middle] +9 \stopxcell
\stopxrow
\startxrow
\startxcell[align=middle] Major Victory \stopxcell
\startxcell[align=middle] +6 \stopxcell
\stopxrow
\startxrow
\startxcell[align=middle] Minor Victory \stopxcell
\startxcell[align=middle] +3 \stopxcell
\stopxrow
\startxrow
\startxcell[align=middle] Marginal Victory \stopxcell
\startxcell[align=middle] 0 \stopxcell
\stopxrow
\startxrow
\startxcell[align=middle] Marginal Defeat \stopxcell
\startxcell[align=middle] 0 \stopxcell
\stopxrow
\startxrow
\startxcell[align=middle] Minor Defeat \stopxcell
\startxcell[align=middle] −3 \stopxcell
\stopxrow
\startxrow
\startxcell[align=middle] Major Defeat \stopxcell
\startxcell[align=middle] −6 \stopxcell
\stopxrow
\stopxtablebody
\startxtablefoot[foot]
\startxrow
\startxcell[align=middle] Complete Defeat \stopxcell
\startxcell[align=middle] −9 \stopxcell
\stopxrow
\stopxtablefoot
\stopxtable
\stopplacetable

Except where your group is exceptionally keen on tracking
{\bf resources}, your GM should skip the {\bf background events} process
when the PCs are long absent from home. Your GM should rejigger them to
serve their plot purposes when they return. The GM may also want to
shuffle this process offstage when the PCs are occupied by epic events.
This prevents them from having to flee from a climactic plot development
to go home and tend to the beet crop.

\subsection[title={Crisis Tests},reference={crisis-tests}]

When {\bf resources} endure {\bf penalties}, you conduct a {\bf crisis
test} at the beginning of each game session to see if trouble strikes
the community. A high but {\bf penalized score} can still lead to
crisis, because people have adjusted to the equilibrium it offers and
feel squeezed when it shifts on them.

A {\bf crisis test} is a {\bf simple contest} (one for each
{\bf penalized ability}) of the {\bf resource score} against a
{\bf resistance} equal to the average of all {\bf resource scores}. Like
{\bf background event} checks, these can't be {\bf augmented} or
{\bf bumped} up by player action. On any {\bf defeat}, the community
starts to visibly suffer.

Your GM invents the specific reasons for each fluctuation and narrates
them to you.

{\bf Crisis tests} should spur you to action, challenging you to find
ways to {\bf bolster} the affected {\bf resources} (see above). When
{\bf bolstered}, the {\bf crisis} is reversed. If you neglect your
duties or fail, the {\bf crisis} worsens.

Your GM will call for {\bf crisis tests} only as needed, as a tool to
generate story. If your group already has enough story on its hands,
your GM will suspend them until you next need a new plot hook.

\subsection[title={Cementing Benefits of Background
Events},reference={cementing-benefits-of-background-events}]

{\bf Bonuses} from {\bf background events} are temporary, unless you
take steps to {\bf cement your benefits}. Doing so requires you to
overcome a major {\bf story obstacle}, perhaps taking focus for an
evening's worth of play. If you succeed, the {\bf background event
bonus} may, as per the next section, later solidify into a permanent
increase in the {\bf resource's score}.

When you {\bf cement a background bonus}, your GM changes their notation
of that {\bf bonus}.

\subsection[title={Changes to Resource
Scores},reference={changes-to-resource-scores}]

At the end of your GM's chosen interval, they review the Resource
Notation Table.

Any {\bf resource} with a {\bf bonus} of 3 or more in its PC column
increases by 1 for each 3 points of {\bf bonus}, for a maximum increase
of 3.

Any {\bf resource} with a {\bf penalty} in its Total column decreases by
1 for each 3 points of {\bf penalty}, for a maximum loss of 2.

Any remaining {\bf modifiers} are now reduced to 0.

The GM now start a new Resource Notation Table, with {\bf resource
scores} altered to reflect any changes from the above process.

Having made permanent changes to the community's {\bf resource scores},
your GM then restarts the cycle by again testing for a new set of
{\bf background events}.

\subsection[title={Changes from Plot
Events},reference={changes-from-plot-events}]

Your GM may decide that certain remarkable triumphs or horrifying
catastrophes may directly alter a {\bf resource score}, independent of
the resource tracking system given here. The possibility of a dramatic
swing in community fortunes should be made clear by your GM during
{\bf contest framing}, so that you know the {\bf prize} and can pull out
all the stops to secure {\bf victory} or stave off {\bf defeat}.

\section[title={Additional Terms},reference={additional-terms}]

The following terms are for rules in the appendix that are no longer
used in the main rules.

\startdescription{{\bf Contest of Wherewithal}}
  A {\bf contest} that allows a {\bf dying} character to complete one
  {\bf final action}.
\stopdescription

\startdescription{{\bf Complete Defeat}}
  No, and\ldots{}. You have lost, and the impact is long-lasting, maybe
  even fatal or terminal.
\stopdescription

\startdescription{{\bf Complete Victory}}
  Yes, and\ldots{} You have won, and the impact is long-lasting,
  possibly a permanent change in your favor.
\stopdescription

\startdescription{{\bf Degree of Victory or Defeat}}
  How well did you triumph, or how badly did you fail: {\bf marginal},
  {\bf minor}, {\bf major}, {\bf complete}
\stopdescription

\startdescription{{\bf Major Defeat}}
  No, and. You have lost, and the impact is long-lasting.
\stopdescription

\startdescription{{\bf Major Victory}}
  Yes, and. You have won, and the impact is long-lasting.
\stopdescription

\startdescription{{\bf Marginal Defeat}}
  No, but\ldots{} You don't get what you want, but the damage may be
  mitigated.
\stopdescription

\startdescription{{\bf Marginal Victory}}
  Yes, but\ldots{} You get what you want, but you may have to make a
  hard choice.
\stopdescription

\startdescription{{\bf Minor Defeat}}
  No\ldots{} you don't get the agreed {\bf prize}.
\stopdescription

\startdescription{{\bf Minor Victory}}
  Yes\ldots{} you get the agreed {\bf prize}.
\stopdescription

\startdescription{{\bf Outcome Point}}
  A point scored in favor one side in a {\bf group simple contest}
\stopdescription

\startdescription{{\bf State of Adversity}}
  How \quote{banged up} a PC is, physically or metaphorically, following
  a {\bf defeat}: {\bf Hurt}, {\bf Injured}, {\bf Impaired}, {\bf Dying}
  and {\bf Dead}
\stopdescription

\startdescription{{\bf State of Fortune}}
  A \quote{boost} to the PC which may be physical or metaphorical.
\stopdescription

\stoptext
